\documentclass[11pt, a4paper]{article}
\input{preamble.tex}
\switchlang{ru}
\begin{document}

\title{1984 "--- Tektronix 4952 joystick}
\date{}
\maketitle
\selectlanguage{russian}

Джойстик Tektronix 4952 (рис. \ref{fig:TektronixJoystickPic}) был разработан для текстовых и графических компьютерных терминалов серии 4010 и аналогичных настольных компьютеров серии 4050 на основе технологии запоминающих электронно-лучевых трубок, созданной Tektronix, чтобы обеспечивать высокое разрешение экрана (до 1024x780) без использования видеопамяти \cite{wiki}. Такие устройства производились Tektronix в конце 1970-х — начале 1980-х годов, до появиления более дешевых рабочих станций UNIX. О пополнении линейки периферийных устройств джойстиком было объявлено в 1974 году \cite{adv}, но известные экземпляры документации датируются следующим годом.

\begin{figure}[h]
   \centering
    \includegraphics[scale=0.45]{1975_Tektronix_4952_Joystick/pic_30.jpg}
    \caption{Джойстик Tektronix 4952}
    \label{fig:TektronixJoystickPic}
\end{figure}

Джойстик имеет резиновые ножки и металлический корпус. На верхней стороне (рис. \ref{fig:TektronixJoystickTopAndBottom}) находятся два триммера \cite{manual} и небольшая рукоятка («рычаг управления» или «control lever» в терминологии производителя). Внешний вид соответствует терминалам и компьютерам Tektronix "--- моноблокам с дисплеем, клавиатурой, процессором и стримером в общем напольном корпусе \cite{wiki}.

\begin{figure}[h]
    \centering
    \includegraphics[scale=0.3]{1975_Tektronix_4952_Joystick/top_30.jpg}
    \includegraphics[scale=0.3]{1975_Tektronix_4952_Joystick/bottom_30.jpg}
    \caption{Джойстик Tektronix 4952 joystick, вид сверху и снизу}
    \label{fig:TektronixJoystickTopAndBottom}
\end{figure}

На лицевой стороне корпуса расположены две подписанные кнопки, а также название компании и модель устройства.
Корпус устройства весьма крупный (рис. \ref{fig:TektronixJoystickSize}).

\begin{figure}[h]
    \centering
    \includegraphics[scale=0.42]{1975_Tektronix_4952_Joystick/size_30.jpg}
    \caption{Джойстик Tektronix 4952 на размерном коврике с шагом сетки 1~см}
    \label{fig:TektronixJoystickSize}
\end{figure}

Рукоятку джойстика достаточно удобно двигать, обхватив пальцами и опираясь кистью на корпус. Размер корпуса не позволяет дотянуться до кнопок на передней панели, не убирая с него руку (рис. \ref{fig:TektronixJoystickHand}). Тем не менее, это было проблемой с учетом специфики использования устройства.

\begin{figure}[h]
    \centering
    \includegraphics[scale=0.42]{1975_Tektronix_4952_Joystick/hand_05.jpg}
    \caption{Джойстик Tektronix 4952 с моделью руки человека}
    \label{fig:TektronixJoystickHand}
\end{figure}

\begin{itemize}
\item \verb!SELECT! это кнопка с фиксацией, необходимая только при использовании джойстика с графическими терминалами 4010: в зависимости от положения она позволяла перемещать <<перекрестие курсора>> (``cross-hair cursor'') либо с помощью джойстика, либо с помощью встроенных в терминал регуляторов колесного типа \cite{manual, manual2}. 

\item \verb!X-Y ZERO! это кнопка, нажатие на которую устанавливает на выходах джойстика $X$ и $Y$ нулевое напряжение, что приводит к немедленному перемещению курсора в центр экрана \cite{manual}.
\end{itemize}

Наклон рукоятки влияет на движение курсора предсказуемым образом: направление движения определяется направлением наклона, а угол наклона пропорционален скорости движения. Триммеры используются для регулировки дрейфа путем выставления нулевого напряжения на выходах $X$ и $Y$ при вертикальном положении рукоятки.

При работе пользователь смещает курсор к нужной точке на дисплее, отклоняя рукоятку джойстика, а затем программное обеспечение (например, система САПР) может получить в нужный момент координаты перекрестия курсора \cite{price}.

Джойстик подключался к системе с помощью длинного толстого кабеля. К компьютерам серии 4050 периферийные устройства подключались с помощью параллельной шины GPIB, а у терминалов 4010 был специальный адаптер, смонтированный в подставке терминала \cite{manual, manual2}.


В разобранном виде джойстик показан на рис. \ref{fig:TektronixJoystickInside}. 

\begin{figure}[h]
    \centering
    \includegraphics[scale=0.8]{1975_Tektronix_4952_Joystick/inside_30.jpg}
    \caption{Джойстик Tektronix 4952 в разобранном виде}
    \label{fig:TektronixJoystickInside}
\end{figure}

Смещение триммеров обеспечивает контроль дрейфа путем механического поворота потенциометров $X$ и $Y$. 

Сборный узел джойстика, включая стержень рукоятки и его крепление, а также потенциометры и конструкцию поворотных узлов корректировки дрейфа, является типовым: в дальнейшем он встречается в неизменном виде, вплоть до полной взаимозаменяемости, во многих аналоговых джойстиках, выпускаемых для промышленных нужд.

\begin{thebibliography}{9}
\bibitem {wiki} Tektronix 4050 -- Wikipedia \url{https://en.wikipedia.org/wiki/Tektronix_4050}
\bibitem {adv} Canadian Information Processing Society (CIPS) Computer Magazine - Vol. 5, Iss. 1-11, 1974. - p. 29
\bibitem {manual} TEKTRONIX 4952 JOYSTICK. Tektronix, Inc., JAN 1975. \url{http://www.bitsavers.org/pdf/tektronix/401x/070-1826-01_4952_Joystick_Jan75.pdf}
\bibitem {manual2} TEKTRONIX 4952 JOYSTICK OPTION 2. Instruction manual Tektronix, Inc., FEB 1976. \url{http://www.bitsavers.org/pdf/tektronix/405x/070-2098-00_4952_Joystick_Feb76.pdf}
\bibitem {price} Stanley J. Tektronix 4952 - Electronixandmore Vintage Electronics and Beyond. -- \url{https://electronixandmore.com/misc/index.php?item=56}
\end{thebibliography}
\end{document}
