\documentclass[11pt, a4paper]{article}
\input{preamble.tex}
\switchlang{en}
\begin{document}

\title{1995 -- Kensington Mouse}
\date{}
\maketitle
\selectlanguage{english}

Kensington Mouse, marketed and advertised as ``Mouse in a Box'' due to its cube-shaped packaging, was released by the American company Kensington Computer Products Group as part of a line of several similarly shaped mice \cite{kensingtonfamily}. In 1995, the manufacturer, which had been using the word ``mouse'' in the names of its trackballs (Turbo Mouse and Expert Mouse) for ten years by then, expanded its product line to include devices that were mice in more than just name. Like the company's trackballs, the Kensington Mouse was available in a version for PCs (fig. \ref{kensingtonmousepic}) and for Apple computers with an ADB interface \cite{mouseinabox, denver}.
   
    \begin{figure}[h]
        \centering
    \includegraphics[scale=0.7]{1995_kensington_mouse_in_a_box/pic_30.jpg}
        \caption{Kensington Mouse}
        \label{kensingtonmousepic}
    \end{figure}

The concept of minimalism and the use of geometric abstraction are clearly visible in the design of the mouse. The mouse body is made of white plastic and has an ovoid base (it is symmetrical about the longitudinal axis and slightly tapered toward the front, i.e., the side furthest from the user). A pair of buttons are located on the front of the body (on the ADB version of the mouse, they are usually combined into a single button, but the company has experimented extensively with these models, so two-button ADB models can also be found on \cite{lowendmac}). The buttons are symmetrical, integrated into the shape of the mouse body. They occupy a third of the mouse's length, are separated by a visually distinguishable gap, and are also separated from the rest by an arched gap. The rest of the top of the mouse is devoid of any elements other than the company logo.(fig. \ref{kensingtonmousetopbottom}). The logo is dark gray, matching the color of the cable and the plain washer that protects the cable from damage where it exits the body. The bottom of the mouse shows a label with technical specifications, five low-friction round ``feet'' that ensure a smooth glide across the desk surface, and a rotating latch ring for removing the ball and cleaning the mouse.

A search of the FCC ID code in the Federal Communications Commission database dates the mouse to 1995.

    \begin{figure}[h]
        \centering
    \includegraphics[scale=0.76]{1995_kensington_mouse_in_a_box/top_15.jpg}
    \includegraphics[scale=0.76]{1995_kensington_mouse_in_a_box/bottom_15.jpg}
        \caption{Kensington Mouse, top and bottom views}
        \label{kensingtonmousetopbottom}
        \end{figure}

The mouse is of medium size, the ball is located exactly in the center of the body and has a fairly small diameter (obviously, its size is limited by the height of the body, which is also quite small, as can be seen from fig. \ref{kensingtonmousesize}). Its developers carefully followed the principles of mathematical design, which is also declared on some versions of the mouse packaging, where one can see a drawing of the case with auxiliary construction lines showing the circles and arcs used to make its shape \cite{lowendmac}.

     \begin{figure}[h]
        \centering
    \includegraphics[scale=0.46]{1995_kensington_mouse_in_a_box/size_15.jpg}
        \caption{Kensington Mouse on a graduated pad with a grid step of 1~cm}
        \label{kensingtonmousesize}
    \end{figure}


Thanks to its symmetrical design, the Kensington Mouse is equally suitable for left-handed and right-handed users. (fig. \ref{kensingtonmousehand}). According to the company's advertising, the mouse fits hands of all sizes because it ``fits the natural shape of the human hand better'' \cite{mouseinabox}. Indeed, the palm position on the mouse body feels very natural (fig. \ref{kensingtonmousehand}).

Thanks to the included Kensington branded adapter, the mouse could be connected to both a serial port (which was its native interface) and a PS/2 port.

Inside the Kensington Mouse there is a classic optomechanical system typical of mice from the first half of the 1990s (fig. \ref{kensingtonmouseinside}), except all electronic components mounted on the back of the board.

As seen on the photo, the text on the board indicates that the mouse was manufactured by Mitsumi Electric, which often acted as a contract developer of mice for other companies.

    \begin{figure}[h!]
        \centering
    \includegraphics[scale=0.68]{1995_kensington_mouse_in_a_box/hand_30.jpg}
        \caption{Kensington Mouse with a human hand model}
        \label{kensingtonmousehand}
    \end{figure}

    \begin{figure}[h!]
        \centering
    \includegraphics[width=\textwidth]{1995_kensington_mouse_in_a_box/inside_30.jpg}
        \caption{Kensington Mouse disassembled}
        \label{kensingtonmouseinside}
    \end{figure}

\begin{thebibliography}{9}
    \bibitem{kensingtonfamily} The first family in mice // PC Magazine, February 10, 1998 -- P. 270 \url{https://books.google.by/books?id=fFrjSBw0w14C&lpg=PA270&dq=kensington%20mouse%20in%20a%20box&hl=ru&pg=PA270#v=onepage&q&f=false}
    \bibitem {mouseinabox} Kensington: Mouse in a Box -- kensington.com. January 06, 1997 \url{https://web.archive.org/web/19970106170908/http://www.kensington.com/prod/mice/mice3d.html}
    \bibitem {denver} Kensinton 1 button ADB mouse -- Apple rescue of Denver \url{https://applerescueofdenver.com/products-page/macintosh-to-powerpc/keyboards-mice-joysticks-macintosh/kensinton-1-button-adb-mouse/} 
    \bibitem {lowendmac} Knight D. 2-Button Kensington Mouse ADB -- Low End Mac \url{https://lowendmac.com/1998/2-button-kensington-mouse-adb/}
\end{thebibliography}

\end{document}
