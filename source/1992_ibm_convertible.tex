\documentclass[11pt, a4paper]{article}
\input{preamble.tex}

\begin{document}

\title{1992 "--- Трекбол/мышь IBM PS/2 Track Ball}
\date{}
\maketitle
В 1992 году Компания IBM выпустила нестандартный трекбол-перевёртыш, который мог работать в двух режимах: собственно трекбола (рис. \ref{fig:IBMConvertibleTrackball}) и обычной мыши (\ref{fig:IBMConvertibleMouse}).

\begin{figure}[h]
    \centering
    \includegraphics[scale=0.5]{1992_ibm_convertible/picball_60}
    \caption{Изображение IBM PS/2 Track Ball в режиме трекбола}
    \label{fig:IBMConvertibleTrackball}
\end{figure}

\begin{figure}[h]
    \centering
    \includegraphics[scale=0.5]{1992_ibm_convertible/picmouse_60}
    \caption{Изображение IBM PS/2 Track Ball в режиме мыши}
    \label{fig:IBMConvertibleMouse}
\end{figure}

Перевод устройства из одного режима в другой выполняется нажатием на пару пластиковых защёлок, меняющих положение верхней "--- либо, в зависимости от режима работы, нижней "--- стенки корпуса, в результате чего шар и расположенные рядом с ним кнопки в большей или в меньшей степени выступают из корпуса \cite{mouses}.

\begin{figure}[h]
    \centering
    \includegraphics[scale=0.5]{1992_ibm_convertible/top_60.jpg}
    \includegraphics[scale=0.5]{1992_ibm_convertible/bottom_60.jpg}
    \caption{Изображение IBM PS/2 Track Ball: слева "--- вид трекбола, справа "--- вид мыши}
    \label{fig:IBMConvertibleTopAndBottom}
\end{figure}

Со стороны трекбола на устройстве имеется 4 клавиши (рис. \ref{fig:IBMConvertibleTopAndBottom}): 2 крупные клавиши являются соответственно левой и правой кнопками мыши, 2 маленькие кнопки - это защёлки, нажатие на которые отвечает за блокирование клавиш с противоположной стороны устройства. Со стороны мыши присутствуют логотип IBM и две крупные кнопки. Само по себе устройство является компактным, пригодно для портативного применения (рис. \ref{fig:IBMConvertibleSize}).

\begin{figure}[h]
    \centering
    \includegraphics[scale=0.5]{1992_ibm_convertible/size_30.jpg}
    \caption{Изображение BM PS/2 Track Ball на размерном коврике с шагом сетки 1~см}
    \label{fig:IBMConvertibleSize}
\end{figure}

С точки зрения анатомического строения кисти, устройство имеет довольно эргономичную  форму и крупные клавиши, которые удобно нажимать пальцами (рис. \ref{fig:IBMConvertibleHand}). Однако из-за гладкости шара, использование манипулятора в качестве мыши на большинстве поверхностей является  затруднительным. Также оказывается проблемным и его использование в качестве трекбола, поскольку в этом режиме устройство опирается на две выступающие клавиши мыши, что отрицательно сказывается на его устойчивости \cite{IBM}.

\begin{figure}[h]
    \centering
    \includegraphics[scale=0.45]{1992_ibm_convertible/hand_60.jpg}
    \caption{Изображение BM PS/2 Track Ball с моделью руки человека}
    \label{fig:IBMConvertibleHand}
\end{figure}

Изучение разобранного трекбола показывает, что он также выполнен по стандартной оптомеханической схеме, и имеет надёжные металлические ролики с подшипниками качения (\ref{fig:IBMConvertibleInside}). Для сопряжения данного устройства с компьютером использовался стандартный порт PS/2.

\begin{figure}[h]
    \centering
    \includegraphics[scale=0.7]{1992_ibm_convertible/inside_60.jpg}
    \caption{Изображение BM PS/2 Track Ball изнутри}
    \label{fig:IBMConvertibleInside}
\end{figure}

Однако в конструкции не предусмотрено способа открыть трекбол для чистки, за исключением отклеивания круглой заглушки, закрывающей доступ к крепежному винту (её можно видеть на рис. \ref{fig:IBMConvertibleTopAndBottom} слева). Учитывая, что попадание мелкого мусора внутрь корпуса механических мышей и трекболов является практически неизбежным, вопрос о длительной эксплуатации данного устройства дополняет его спорные эргономические характеристики.

\begin{thebibliography}{9}
\bibitem {mouses} Quain J.R. IBM PS/2 trackpoint // PC Magazine. October 15, 1991. p. 126 \url{https://books.google.by/books?id=tSLe3yMjc-AC&lpg=PP1&pg=PT123#v=onepage&q&f=false}
\bibitem {IBM} IBM PS/2 L40SX "Convertible" Pointing Device \url{https://www.youtube.com/watch?v=-OSXeNVM3UI&ab_channel=uxwbill}
\end{thebibliography}
\end{document}
