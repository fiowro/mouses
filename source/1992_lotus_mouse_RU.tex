\documentclass[11pt, a4paper]{article}
\input{preamble.tex}
\switchlang{ru}
\begin{document}

\title{1992 "--- Lotus Mouse}
\date{}
\maketitle
\selectlanguage{russian}
Lotus Mouse является характерным примером оптомеханического манипулятора 90-х годов. Внешний вид данной мыши можно видеть на рис. \ref{fig:LotusPic}.

\begin{figure}[h]
    \centering
    \includegraphics[scale=0.6]{1992_lotus_mouse/pic_30.jpg}
    \caption{Мышь Lotus Mouse}
    \label{fig:LotusPic}
\end{figure}

Мышь имеет асимметричный корпус эргономичной формы и предназначена для использования правой рукой (рис. \ref{fig:LotusHand}). На верхней стороне корпуса присутствуют две кнопки большого размера и надпись Lotus. Поверхность главной кнопки мыши сделана рельефной, чтобы упростить ее тактильную идентификацию.

\begin{figure}[h]
    \centering
    \includegraphics[scale=0.3]{1992_lotus_mouse/hand_30.jpg}
    \caption{Изображение Lotus Mouse с моделью руки человека}
    \label{fig:LotusHand}
\end{figure}

\begin{figure}[h]
    \centering
    \includegraphics[scale=0.3]{1992_lotus_mouse/top_30.jpg}
    \includegraphics[scale=0.3]{1992_lotus_mouse/bottom_30.jpg}
    \caption{Lotus Mouse, вид сверху и снизу}
    \label{fig:LotusTopBottom}
\end{figure}

Перевернув мышь (рис. \ref{fig:LotusTopBottom}), можно увидеть обрезиненный шарик, ножки для скольжения по поверхности, выполненные из низкофрикционного полимера, а также маркировку мыши.

Мышь подключается к компьютеру через последовательный порт.

\begin{figure}[h]
    \centering
    \includegraphics[scale=0.3]{1992_lotus_mouse/size_30.jpg}
    \caption{Изображение Lotus Mouse на размерном коврике с шагом сетки 1~см}
    \label{fig:LotusSize}
\end{figure}

Можно сказать, что Lotus Mouse в определенной степени предвосхищает форму мыши Microsoft Mouse 2.0, которая была выпущена годом позже, а затем в свою очередь подарила форму самой известной мыши Microsoft "--- IntelliMouse.

\begin{figure}[h]
    \centering
    \includegraphics[scale=0.7]{1992_lotus_mouse/inside_30.jpg}
    \caption{Lotus Mouse в разобранном состоянии}
    \label{fig:LotusInside}
\end{figure}

Внутреннее устройство данного манипулятора показано на рис. \ref{fig:LotusInside}, что позволяет классифицировать его как традиционную оптомеханическую конструкцию. Изучение кода FCC ID по базе данных Федеральной комиссии по связи США показывает, что мышь была изготовлена в 1992 году на заводе в Сингапуре тайваньской компанией BMC Micro Industries.
\end{document}
