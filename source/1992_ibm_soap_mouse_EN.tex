\documentclass[11pt, a4paper]{article}
\input{preamble.tex}
\switchlang{en}
\begin{document}

\title{1992 "--- IBM ``Soap'' mouse}
\date{}
\maketitle
\selectlanguage{english}

This mouse, officially known by the hard-to-remember name ``model 13H6690'' (and almost identical ``model 33G5430'') was apparently released to the market in 1992 as an update to the 1987 IBM PS/2 mouse.
In addition to the obvious association with a well-rounded bar of soap, this model was also unofficially called ``IBM Fat Mouse''.

\begin{figure}[h]
    \centering
    \includegraphics[scale=0.6]{1992_ibm_soap_mouse/pic_30.jpg}
    \caption{IBM Soap mouse}
    \label{fig:IBMSoapPic}
\end{figure}

Despite its wide distribution, this mouse is practically not represented in press reviews. IBM supplied it with some models of PS/1 and PS/ValuePoint computers (however, along with this mouse, you can find promotional materials for these computers with the previous IBM PS/2 Mouse model).

\begin{figure}[h]
    \centering
    \includegraphics[scale=0.5]{1992_ibm_soap_mouse/top_60.jpg}
    \includegraphics[scale=0.5]{1992_ibm_soap_mouse/bottom_60.jpg}
    \caption{IBM Soap mouse, top and bottom views}
    \label{fig:IBMSoapTopBottom}
\end{figure}

The mouse was produced in several color options: a solid beige body, and two-color versions with dark buttons (fig. \ref{fig:IBMSoapTopBottom}) or a dark lower part of the body \cite{hugold}. On the top side there are two large buttons and an engraved IBM logo; in general, the case is minimalistic and, apart from that, does not contain any additional elements. On the bottom you can see a ball and a swivel ring that allows you to remove it for cleaning.

\begin{figure}[h]
    \centering
    \includegraphics[scale=0.34]{1992_ibm_soap_mouse/size_30.jpg}
    \caption{IBM Soap mouse on a graduated pad with a grid step of 1~cm}
    \label{fig:IBMSoapSize}
\end{figure}

The mouse is compact in size (fig. \ref{fig:IBMSoapSize}) and does not provide wrist support. However, as can be seen in fig. \ref{fig:IBMSoapHand}, the curved shape of the body allows you to comfortably rest your palm on it and press the buttons in a natural hand position. At the same time, the mouse is symmetrical and is equally suitable for left-handers and right-handers. In addition to ergonomics, the mouse was also successful in terms of reliability \cite{usage}, which apparently made it popular and caused a fairly long mass production.

\begin{figure}[h]
    \centering
    \includegraphics[scale=0.34]{1992_ibm_soap_mouse/hand_30.jpg}
    \caption{IBM Soap mouse with a human hand model}
    \label{fig:IBMSoapHand}
\end{figure}

The internals of the IBM Soap mouse are shown in fig. \ref{fig:IBMSoapInside}. This is an optomechanical encoder device manufactured by Logitech. The heavily slotted encoder disks, the low number of discrete PCB elements, and the plastic rollers are typical of mid-90s rather than early 90s mice, making the IBM Soap mouse one of the pioneers of this trend.

\begin{figure}[h]
    \centering
    \includegraphics[scale=0.7]{1992_ibm_soap_mouse/inside_60.jpg} 
    \caption{IBM Soap mouse disassembled}
    \label{fig:IBMSoapInside}
\end{figure}

\begin{thebibliography}{9}
\bibitem {usage} Wendt P.H. Mice \& other stuff \url{http://www.mcamafia.de/mycomp/mycomp06.htm}
\bibitem {hugold} Przytul grata. IBM Mouse PS/2. \url{http://hugold.pl/gratym0126/ibm33g5430.html}
\end{thebibliography}
\end{document}
