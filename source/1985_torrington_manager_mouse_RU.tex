\documentclass[11pt, a4paper]{article}
\input{preamble.tex}
\switchlang{ru}
\begin{document}

\title{1985 "--- Torrington Manager Mouse}
\date{}
\maketitle
\selectlanguage{russian}

Manager Mouse была выпущена в 1985 году компанией Torrington, начавшей свой бизнес в середине XIX века в роли производителя инновационных швейных игл под именем Excelsior Needle Company. За сто с небольшим лет, предшествовавших выпуску Manager Mouse, компания пережила несколько этапов расширения ассортимента продукции на смежные области, такие как велосипедных спиц и игольчатых подшипников. Незадолго до 1985 Torrington стала автономным дочерним предприятием компании Ingersoll Rand Inc., полностью отказалась от производства игл, будучи на тот момент известна в первую очередь как крупный международный производитель подшипников \cite{torrington} "--- и, очевидно, продолжала активно экспериментировать с новыми рынками, результатом чего и стало появление данной компьютерной мыши.

\begin{figure}[h]
   \centering
    \includegraphics[width=\textwidth]{1985_torrington_manager_mouse/pic_30.jpg}
    \caption{Manager Mouse}
    \label{fig:ManagerMousePic}
\end{figure}

Мышь появилась на рынке в двух вариантах, с проводным (модель 1001C) и с беспроводным инфракрасным (модель 1001С-IR) подключением. Помимо того, что Manager Mouse стала первой беспроводной мышью в широкой продаже, она привлекала внимание также и необычным способом регистрации движения на основе колес: поэтому ей было посвящено несоклько обзоров в прессе, а в последующие годы беспроводной вариант Manager Mouse можно было встретить под брэндом других компаний --- например, Maynard Electronics \cite{admaynard} и Numonics Corporation \cite{adnumonics} (в 1989 году Numonics пошла дальше и выпустила собственный вариант Manager Mouse в более современном обтекаемом корпусе).

Датировку мыши и ее оригинального разработчика можно определить по коду FCC ID, который, согласно базе данных Федеральной комиссии по связи США, показывает, что мышь была выпущена на рынок в 1985 году Компанией Torrington в Коннектикуте.

Мышь выполнена в бежевом корпусе, демонстрирующем строгий индустриальный дизайн и минимализм.

\begin{figure}[h]
    \centering
    \includegraphics[scale=0.9]{1985_torrington_manager_mouse/top_30.jpg}
    \includegraphics[scale=0.9]{1985_torrington_manager_mouse/bottom_30.jpg}
    \caption{Manager Mouse, вид сверху и снизу}
    \label{fig:ManagerMouseTopAndBottom}
\end{figure}

Корпус имеет практически прямоугольную форму, если не считать скошенную вперёд часть верхней грани, на которой находятся три вытянутые прямоугольные кнопки более тёмного оттенка. Кабель проводного варианта мыши не снабжён ограничительной втулкой, которая защищала бы его от механических повреждений в месте выхода из корпуса. На верхней стороне в зоне запястья пользователя находится табличка с надписью <<Manager mouse>>. На нижней стороне (рис. \ref{fig:ManagerMouseTopAndBottom}) можно видеть наклейку с информационными данными, а также два маленьких колеса в форме усечённого конуса, предназначенные для регистрации движения вместо более традиционного для механических мышей шара. Это решение восходит к первым мышам Дугласа Энгельбарта: колеса расположены ортогонально друг другу, чтобы при продольном и поперечном движении одно из них вращалось, а другое проскальзывало. Но, по сравнению с мышью Энгельбарта, колеса Manager Mouse существенно уменьшены в диаметре, а их оси расположены под углом к горизонтальной плоскости, что должно обеспечивать регистрацию диагональных перемещений.

Корпус имеет размер, типичный для первой половины 1980-х годов (рис. \ref{fig:ManagerMousePic}, \ref{fig:ManagerMouseSize}).

\begin{figure}[h]
    \centering
    \includegraphics[scale=0.51]{1985_torrington_manager_mouse/size_30.jpg}
    \caption{Manager Mouse на размерном коврике с шагом сетки 1~см}
    \label{fig:ManagerMouseSize}
\end{figure}

По сильному визуальному сходству можно предположить, что форма корпуса была в течение последующих двух лет позаимствована тайваньской компанией KYE Systems для первого поколения своих мышей Genius Mouse (начиная с модели GM-3 и заканчивая моделью GM-5 в <<старом>> дизайне).

Версии Manager Mouse c интерфейсом RS-232 (выпускались варианты и с 9, и с 25-контактным разъёмом) работают по протоколу Mouse Systems. В \cite{maynard} упоминается также вариант c адаптером, вставляемым в системную шину.

В плане эргономики мышь не может похвастаться существенными достоинствами. Пользовательский опыт очевидно страдает от суровой прямоугольности корпуса, с учетом того, что он имеет значительную высоту и не может обеспечить существенной поддержки ладони (рис. \ref{fig:ManagerMouseHand}). Кроме того, как отмечено в \cite{managerwheels}, конструкция колес все же имеет недостатки: время от времени они проскальзывают (особенно на гладких поверхностях), а диагональное движение курсора несмотря на наклонные оси колёс всё равно выглядит неровным.


\begin{figure}[h]
    \centering
    \includegraphics[scale=0.51]{1985_torrington_manager_mouse/hand_30.jpg}
    \caption{Manager Mouse с моделью руки человека}
    \label{fig:ManagerMouseHand}
\end{figure}

Внутреннее устройство мыши показано на рисунке \ref{fig:ManagerMouseInside}. Как можно видеть, она представляет собой механическую конструкцию, в которой диски механичекого энкодера закреплены непосредственно на осях колес. В рекламе производителя отмечается \cite{buxton}, что это "--- <<запатентованная система независимой подвески, которая обеспечила широкое признание всем остальным мышам линейки Torrington Manager Mouse>>, которая <<обеспечивает плавное движение мыши практически на любой поверхности и под любым углом>>, а кроме того, сообщается, что при движении мыши полиуретановые колеса <<фактически выталкивают мусор в стороны>>. Согласно \cite{buxton}, благодаря использованию колёс мышь действительно засорялась несколько меньше, чем механические и оптомеханические мыши на основе шара. Кроме того, механический энкодер требовал меньше электропитания по сравнению с оптомеханическим: до появления в 1986 году низкопотребляющих светодиодов, все оптомеханические мыши и трекболы с интерфейсом RS-232 нуждались в дополнительном источнике питания.

В обзоре \cite{buxton} упоминается, что <<два маленьких пластиковых колеса, расположенных под прямым углом друг к другу, вместо оптического детектора или трекбола>> обеспечивают разрешение 100 точек на дюйм (что, например, раза меньше разрешения мышей Logitech и Microsoft).

 \begin{figure}[h]
    \centering
    \includegraphics[width=\textwidth]{1985_torrington_manager_mouse/inside_30.jpg}
    \caption{Manager Mouse в разобранном виде}
    \label{fig:ManagerMouseInside}
\end{figure}

\begin{thebibliography}{9}

\bibitem {torrington} The Torrington Company / International Directory of Company Histories. ed. by T. Grant. Vol. 13. - St.James Press, USA. 1996. P. 521 \url{https://archive.org/details/internationaldir0013unse/page/520/mode/2up}

\bibitem {admaynard} This Little Fella Means Business [adv.] // PC Magazine, Vol. 5, No. 1, January 14, 1986. P. 37. \url{https://archive.org/details/PC-Mag-1986-01-14/page/n71/mode/2up}

\bibitem {adnumonics} Don't let your mouse get hung-up on itself [adv.] // PC Magazine, Vol. 7, No. 6, March 29, 1988. P. 282 // \url{https://archive.org/details/PC-Mag-1988-03-29/page/n277/mode/2up}

\bibitem {maynard} Hart G. Building a better mouse interface. Maynard mouse // PC Magazine, Vol. 5, No. 4, February 25, 1986. - p. 170--172 \url{https://archive.org/details/PC-Mag-1986-02-25/page/n177/mode/2up}

\bibitem {managerwheels} Numonics corp. Manager mouse, Manager mouse cordless. //  PC Magazine, V. 7, No. 3, February 14, 1989. - p. 268. \url{https://archive.org/details/PC-Mag-1989-02-14/page/n267/mode/2up}

\bibitem {buxton} Manager Mouse Cordless \url{https://www.billbuxton.com/inputTimelineAssets/TorringtonBrochure.pdf}

\end{thebibliography}
\end{document}
