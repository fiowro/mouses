\documentclass[11pt, a4paper]{article}
\input{preamble.tex}
\switchlang{ru}
\begin{document}

\title{1994 "--- Memorex trackball}
\date{}
\maketitle
\selectlanguage{russian}
Трекбол Memorex, показанный на рис. \ref{fig:MemorexPic}, выпускался калифорнийской компанией Memtek Products inc. При этом фактическое производство типичным для середины 90-х годов образом велось на Тайвани.

\begin{figure}[h]
    \centering
    \includegraphics[scale=0.5]{1994_memorex_trackball/pic_30.jpg}
    \caption{Вид трекбола Memorex}
    \label{fig:MemorexPic}
\end{figure}

Корпус трекбола является асимметричным, выполнен в минималистичном стиле из глянцево-белого пластика с коричневой вставкой.

\begin{figure}[h]
    \centering
    \includegraphics[scale=0.45]{1994_memorex_trackball/top_30.jpg}
    \includegraphics[scale=0.45]{1994_memorex_trackball/bottom_30.jpg}
    \caption{Easy Options, вид сверху и снизу}
    \label{fig:MemorexTopBottom}
\end{figure}

Как можно видеть на рис. \ref{fig:MemorexSize}, этот манипулятор имеет сравнительно небольшие размеры.
На упаковке трекбола и в руководстве пользователя использован подзаголовок "Stationary Mouse", очевидно позаимствованный у выпущенной годом ранее модели трекбола Logitech TrackMan Stationary Mouse, и подчеркивается удобство его использования в условиях ограниченного рабочего пространства.

\begin{figure}[h]
    \centering
    \includegraphics[scale=0.35]{1994_memorex_trackball/size_30.jpg}
    \caption{Трекбол Memorex на размерном коврике с шагом сетки 1 см}
    \label{fig:MemorexSize}
\end{figure}

Трекбол рассчитан на управление правой рукой, а его форма в значительной степени инспирирована другим продуктом Logitech - LOGiTECH, в который также рассчитан на горизонтальное расположение кисти руки и вращение шара большим пальцем (рис. \ref{fig:MemorexHand}). По замыслу Logitech, данная форма является лучше соответствует анатомическому строению кисти (в рекламе форма классических симметричных трекболов противопоставлялась подобной форме как <<устройство для инопланетян>> \cite{adv}).

\begin{figure}[h]
    \centering
    \includegraphics[scale=0.2]{1994_memorex_trackball/hand_30.jpg}
    \caption{Трекбол Memorex в комплекте с моделью руки человека}
    \label{fig:MemorexHand}
\end{figure}

Внутреннее устройство трекбола показано на рис. \ref{fig:MemorexInside}, что позволяет классифицировать его как типичное оптомеханическое устройство середины 90-х годов, как в плане реализации энкодера, так и в плане компоновки, оставлявшей значительный объем пустого пространства внутри корпуса.

\begin{figure}[h]
    \centering
    \includegraphics[scale=0.6]{1994_memorex_trackball/inside_30.jpg}
    \caption{Memorex в разобранном виде}
    \label{fig:MemorexInside}
\end{figure}

\begin{thebibliography}{9}
\bibitem {adv} Not every kind of pointing device fits your kind of hand (LOGiTECH TrackMan tadvertising). // PC Magazine. V.~8, No.~19. November, 1989, pp. 360-361. \url{https://archive.org/details/PC-Mag-1989-11-14/page/n361}
\end{thebibliography}
\end{document}
