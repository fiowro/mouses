\documentclass[11pt, a4paper]{article}
\input{preamble.tex}

\begin{document}

\title{1986 "--- Commodore C300 mouse}
\date{}
\maketitle

Мышь Commodore mouse C300, подписанная на коробке как Joystick Mouse, является одной из реинкарнаций Kempston Mouse, выпущенной в 1986 году для компьютеров ZX Spectrum \cite{SinclairUser}. Внешне мышь является достаточно типичной, с двумя кнопками на верхней стороне и шаром снизу (рис. \ref{fig:C300Pic}).

\begin{figure}[h]
    \centering
    \includegraphics[scale=0.7]{1986_commodore_c300_mouse/cmnirm_30.jpg}
    \caption{Commodore C300 Mouse}
    \label{fig:C300Pic}
\end{figure}

Согласно прилагаемому описанию, данная мышь работает в двух режимах работы: в режиме джойстика и пропорциональном режиме. Режим работы определяется при включении компьютера: при нажатии правой кнопки мыши мышь переходит в режим джойстика, в противном случае (по умолчанию) в пропорциональный режим. В режиме джойстика левая кнопка мыши задействована как кнопка включения джойстика, а правая кнопка эквивалентна движению джойстика вверх. Режим джойстика позволяет использовать мышь с любым программным обеспечением, совместимым с джойстиком, а также играет роль <<резерва>> на случай, когда используемое программное обеспечение не поддерживает пропорциональный режим.


\begin{figure}[h]
    \centering
    \includegraphics[scale=0.7]{1986_commodore_c300_mouse/3verh_60.jpg}
    \includegraphics[scale=0.7]{1986_commodore_c300_mouse/3niz_60.jpg}
    \caption{Commodore C300 Mouse, вид сверху и снизу}
    \label{fig:C300TopAndBottom}
\end{figure}

Как иможно заметить на рис. \ref{fig:C300TopAndBottom}, плавное скольжение мыши по рабочей поверхности обеспечивается не тканевыми накладками с низким коэффициентом сопротивления или пластиковым <<ножками>>, а четырьмя металлическими шариками малого диаметра, расположенными в дополнительных отверстиях нижней стенки корпуса и свободно вращающимися при движении (решение, характерное для ряда механических мышей 80-х годов, выпускавшихся до ряда изменений типовой конструкции, направленных на удешевление изделий).

\begin{figure}[h]
    \centering
    \includegraphics[scale=0.3]{1986_commodore_c300_mouse/cmruka_30.jpg}
    \caption{Commodore C300 Mouse с моделью руки человека}
    \label{fig:C300Hand}
\end{figure}

\begin{figure}[h]
    \centering
    \includegraphics[scale=0.3]{1986_commodore_c300_mouse/cmblock_30.jpg}
    \caption{Адаптер Commodore, вид спереди}
    \label{fig:C300Block}
\end{figure}

\begin{figure}[h]
    \centering
    \includegraphics[scale=0.7]{1986_commodore_c300_mouse/cm4raz_30.jpg}
%    \includegraphics[scale=0.7]{1986_commodore_c300_mouse/cm41raz_30.jpg}
    \caption{Commodore C300 Mouse в разобранном виде}
    \label{fig:C300Inside}
\end{figure}

Разбор мыши показывает достаточно необычную конструкцию (рис. \ref{fig:C300Inside}): это оригинальное оптомеханическое указательное устройство, в котором нестандартно реализован оптический прерыватель. Вместо прохождения диска между ситочником и приёмником света, светодиод и фотодиод располагаются по одну и ту же сторону диска, а сам диск является сплошным. Вместо прорезей, в нём использованы радиальные металлические полосы, отражающие свет, в то время как черный матовый материал самого диска рассеивает его, приводя к отсутствию сигнала на фотоприёмнике. В электрическом плане подобная конструкция не отличается от других оптомеханических мышей, но визуально напоминает диск контактного энкодера (в случае которого при механическом контакте щёток с поверхностью диска происходило бы замыкание цепи в момент прохожедния металлической радиальной полосы и её размыкание когда щётка оказывалась между полосами).

Безусловно, такое решение не является дешёвым в сравнении с традиционной схемой, использующей пластиковый диск с прорезями без металлизации.

\begin{thebibliography}{9}
\bibitem {LittleMagick} The mouse and the C64 \url{https://www.c64-wiki.com/wiki/Mouse}
\bibitem {SinclairUser} Kempston mouse // Sinclair User, Iss. 56, November 1986. -- p. 29. \url{https://worldofspectrum.org/archive/magazines/sinclair-user/56/0/1986/11/0}
\bibitem {c64wiki} Mouse -- C64-Wiki \url{https://www.c64-wiki.com/wiki/Mouse}
\end{thebibliography}
\end{document}
