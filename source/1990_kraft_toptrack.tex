\documentclass[11pt, a4paper]{article}
\input{preamble.tex}

\begin{document}

\title{1990 "--- Трекбол Kraft TopTrak}
\date{}
\maketitle

Трекбол TopTrak имеет средние размеры и корпус с закруглёнными углами (рис. \ref{fig:TopTrakPic}, \ref{fig:TopTrakTopAndBottom}).

\begin{figure}[h]
    \centering
    \includegraphics[scale=0.45]{1990_kraft_toptrack/pic_60.jpg}
    \caption{Изображение мыши Kraft TopTrak}
    \label{fig:TopTrakPic}
\end{figure}

Устройство снабжено кабелем длиной 2.5 м, что заметно больше типового расстояния между пользователем и системным блоком. Трекбол имеет последовательный интерфейс сопряжения с компьютером.

\begin{figure}[h]
    \centering
    \includegraphics[scale=0.4]{1990_kraft_toptrack/2.9_60.jpg}
    \includegraphics[scale=0.4]{1990_kraft_toptrack/2.10_60.jpg}
    \caption{TopTrak, вид сверху и снизу}
    \label{fig:TopTrakTopAndBottom}
\end{figure}

Дополнительно в комплекте с трекболом идёт стальная ножная педаль (рис. \ref{fig:TopTrakPedal}), которая служит альтернативой левой кнопке мыши, имеет еще один аналогичный кабель и добавляет дополнительные полкилограмма веса устройству \cite{mouses}.

\begin{figure}[h]
    \centering
    \includegraphics[scale=0.45]{1990_kraft_toptrack/pedal_30.jpg}
    \caption{Изображение педали для мыши TopTrak}
    \label{fig:TopTrakPedal}
\end{figure}

%\begin{figure}[h]
%    \centering
%    \includegraphics[scale=0.4]{1990_kraft_toptrack/2.8.jpg}
%    \caption{Изображение педали для мыши TopTrak с моделью руки человека}
%    \label{fig:TopTrakPedalHand}
%\end{figure}

Kraft TopTrak является достаточно компактным устройством (рис. \ref{fig:TopTrakSize}), в особенности в сравнении с предыдущей моделью трекбола Kraft, выпущенной в 1989 году ~--- существенно большей по размеру, имевшей подчёркнуто угловатый корпус, и комплектовавшейся в точности такой же педалью.

\begin{figure}[h]
    \centering
    \includegraphics[scale=0.35]{1990_kraft_toptrack/2.6_30.jpg}
    \caption{Изображение TopTrak на размерном коврике с шагом сетки 1~см}
    \label{fig:TopTrakSize}
\end{figure}


Код FCC ID, присутствующий на корпусе TopTrak (рис. \ref{fig:TopTrakTopAndBottom}), показывает, что трекбол был разработан американской компанией Kraft Systems в 1990 году, всего через год после выхода предыдущей модели.

В плане эргономики TopTrak можно назвать существенным шагом вперед. Обтекаемая
форма корпуса, а также большие левая и правая кнопки, которые полностью занимают
верхние углы, обеспечивают достаточно удобное положение ладони (рис. \ref{fig:TopTrakHand}).


\begin{figure}[h]
    \centering
    \includegraphics[scale=0.45]{1990_kraft_toptrack/hand_60.jpg}
    \caption{Изображение TopTrak с моделью руки человека}
    \label{fig:TopTrakHand}
\end{figure}


%\begin{figure}[htpb]
%    \centering
%    \includegraphics[scale=0.4]{1990_kraft_toptrack/2.10.jpg}
%    \caption{TopTrak, вид снизу}
%    \label{fig:TopTrakBottom}
%\end{figure}

Изучение разобранного трекбола (рис. \ref{fig:TopTrakInside}) показывает, что он выполнен по стандартной оптомеханической схеме, а массивные металлические ролики с подшипниками качения показывают, что трекбол был задуман как достаточно долговечное устройство, не относящееся к нижнему ценовому диапазону.

\begin{figure}[h]
    \centering
    \includegraphics[scale=0.7]{1990_kraft_toptrack/inside_60.jpg}
    \caption{TopTrak изнутри}
    \label{fig:TopTrakInside}
\end{figure}

\begin{thebibliography}{9}
\bibitem {mouses} Berlin E. TopTrak // PC Magazine. October 15, 1991. p. 126-127 \url{https://books.google.by/books?id=tSLe3yMjc-AC&lpg=PP1&pg=PT123#v=onepage&q&f=false}
\end{thebibliography}
\end{document}
