\documentclass[11pt, a4paper]{article}
\input{preamble.tex}
\switchlang{ru}
\begin{document}

\title{1983 "--- Logitech LOGIMOUSE P5}
\date{}
\maketitle
\selectlanguage{russian}
Мышь LOGIMOUSE P5 была выпущена в 1983 году. Это второй по счету манипулятор фирмы Logitech (после P4, выпуск которого продолжился одновременно с P5). P5 был удешевленной моделью (отпускная цена различалась на \$100) и предназначался в первую очередь для поставок ОЕМ.
Но главной особенностью P5 является кардинально отличающийся внешний вид (рис. \ref{fig:LogimouseP5Pic}). 

\begin{figure}[h]
   \centering
    \includegraphics[scale=0.35]{1983_logitech_logimouse_p5/pic_30.jpg}
    \caption{LOGIMOUSE P5, вид спереди}
    \label{fig:LogimouseP5Pic}
\end{figure}

Мышь имеет вызывающе дизайнерскую наружность, хотя в 1983 году это понятие применительно к мышам еще не существовало: на черном призматическом корпусе с обратным наклоном диагонально располагаются три узкие белые кнопки. На нижней стороне присутствуют четыре белые опоры с низким коэффициентом трения (одновременно это крепления печатной платы), и шар, выполненный из матового металла (рис. \ref{fig:LogimouseP5TopAndBottom}). Можно отметить съемное поворотное кольцо на защелках, позволяющее извлечь шар для удаления собравшегося мусора и чистки роликов. В Logitech P4 и других мышах первой половины 80-х годов кольцо требовалось отвинчивать с помощью отвертки; таким образом, это один из первых случаев (наряду с мышью Apple Lisa того же года выпуска) применения поворотного кольца, ставшего впоследствии стандартом.

\begin{figure}[h]
    \centering
    \includegraphics[scale=0.4]{1983_logitech_logimouse_p5/top_30.jpg}
    \includegraphics[scale=0.4]{1983_logitech_logimouse_p5/bottom_30.jpg}
    \caption{LOGIMOUSE P5, вид сверху и снизу}
    \label{fig:LogimouseP5TopAndBottom}
\end{figure}

Судя по кабелю с двумя разъемами, это мышь от FutureNet "--- рабочей станции для САПР микроэлектроники. Компьютер представлял собой классическую IBM PC с монохромным текстовым видеоадаптером и дополнительной графической платой разрешением 640x360 пикселей, и пятнадцатиконтактный разъем на кабеле мыши использовался, чтобы соединить выход монохромного видеоадаптера IBM и графическую плату. Такое решение позволяло попеременно использовать дисплей для текстового вывода и для вывода графики редактором схем \cite{futurenet}.

\begin{figure}[h]
    \centering
    \includegraphics[scale=0.5]{1983_logitech_logimouse_p5/size_30.jpg}
    \caption{LOGIMOUSE P5 на размерном коврике с шагом сетки 1~см}
    \label{fig:LogimouseP5Size}
\end{figure}

Мышь имеет размеры, типичные для мышей 1980-х годов (рис. \ref{fig:LogimouseP5Size}). Фактически, ее габариты близки к более поздней модели Logitech P7 и ее наиболее массовой разновидности, C7. Что касается эргономики, то очевидно, что призматический корпус и диагональные узкие кнопки сказались на ней весьма негативно: при расположении пальцев вдоль кнопок ладонь рискует опереться на торчащий угол корпуса, а расположении ладони вдоль линий корпуса улучшает ситуацию лишь незначительно, взамен осложняя нажатие на и без того неудобные кнопки (рис. \ref{fig:LogimouseP5Hand}).

\begin{figure}[h]
    \centering
    \includegraphics[scale=0.5]{1983_logitech_logimouse_p5/hand_30.jpg}
    \caption{LOGIMOUSE P5 с моделью руки человека}
    \label{fig:LogimouseP5Hand}
\end{figure}

В электрическом плане P5 аналогична мыши Logitech P4, но имеет меньшее разрешение (200 DPI против 381 DPI). В качестве дополнительного аксессуара обе мыши могли комплектоваться специальным адаптером-переходником LogiMate, который позволял подключить мышь не к отдельному адаптеру с шинным интерфейсом, а в разрыв кабеля клавиатуры. При таком подключении перемещение мыши приводило к генерации кодов нажатий клавиш управления курсором: в стандартном режиме разрешение составляло 12 нажатий клавиш на дюйм по горизонтали и 6 нажатий по вертикали, что было рассчитано на работу в текстовом режиме 80x25 символов. В \cite{logimouse} отмечается, что использование мыши в стандартном режиме позволяло перемещать курсор в текстовом редакторе в семь раз быстрее, чем соответствующие клавиши на клавиатуре (однако требовалось привыкнуть к тому, что в результате небольшого промаха пользователя за крайней правой позицией курсор текстового редактора неизменно перескакивал на левую позицию следующей строки). По очевидным причинам для использования мыши не требовался драйвер; однако его применение давало возможность дополнительных настроек "--- например, позволяло выставлять разрешение переходника LogiMate в диапазоне 1--100 нажатий на дюйм, а также переназначить действие клавиш мыши (по-умолчанию генерировались коды клавиш F8, F9 и F10).

 \begin{figure}[h]
    \centering
    \includegraphics[scale=0.7]{1983_logitech_logimouse_p5/inside_30.jpg}
    \caption{LOGIMOUSE P5 в разобранном виде}
    \label{fig:LogimouseP5Inside}
\end{figure}

Внутреннее устройство мыши показано на рис. \ref{fig:LogimouseP5Inside}. В мыши использованы оптомеханические энкодеры. Оптопары и диск оптического прерывателя очень похожи на соответствующие детали в мышах 90-х годов (можно считать это одной из первых подобных реализаций). Отличительной особенностью энкодера (как и в модели P4) является дополнительная неподвижная маска, уменьшавшая площадь засветки.

\begin{thebibliography}{9}
\bibitem {logimouse} J. Taylor. Faster then a speeding cursor key. // PC Magazine, V. 3, No. 2, February 7, 1984. - p. 243-245 \url{https://archive.org/details/PC-Mag-1984-02-07/page/n243/mode/2up}
\bibitem {futurenet} M. Holley. Logitech Logimouse Cable 1983 \url{https://commons.wikimedia.org/wiki/File:Logitech_Logimouse_Cable_1983.jpg}
\end{thebibliography}
\end{document}
