\documentclass[11pt, a4paper]{article}
\input{preamble.tex}
\switchlang{ru}
\begin{document}

\title{1994 "--- Trinity Mariqua Mouse}
\date{}
\maketitle
\selectlanguage{russian}
Мышь Trinity Mariqua Mouse была выпущена в 1994 году для японского рынка компанией Miyuki Electronic Design. Вслед за семейством мышей и трекболов Prohance 1989 года, 
Mariqua Mouse следует концепции размещения на корпусе мыши дополнительных кнопок, которые, по замыслу разработчиков, избавляют пользователя от необходимости часто перемещать руку с мыши на клавиатуру и обратно.

Мышь Trinity Mariqua выпускалась в двух модификациях, отличавшихся цветом (рис. \ref{fig:TrinityPic}): в корпусе цвета красный металлик с черными элементами, либо в белом с серыми элементами.

\begin{figure}[h]
    \centering
    \includegraphics[scale=0.46]{1994_mariqua_mouse/pic_b_60.jpg}
    \includegraphics[scale=0.46]{1994_mariqua_mouse/pic_w_60.jpg}
    \caption{Colani Trackball}
    \label{fig:TrinityPic}
\end{figure}

На нижней стороне корпуса (рис. \ref{fig:TrinityTopBottom}) можно видеть поворотное кольцо, позволяющее извлечь шар для чистки мыши, ярлык с техническими данными, а также четыре накладки из низкофрикционного материала.
На верхней стороне расположены три достаточно крупные кнопки, заходящие на переднюю часть корпуса мыши, а также 16 миниатюрных круглых кнопок, выполняющих роль цифровой клавиатуры и функциональных клавиш.
На боковых стенках корпуса находятся еще две прямоугольные кнопки. На передней стороне корпуса присутствует ребристая муфта, предохраняющая кабель от механических повреждений в месте выхода из корпуса, а также два светодиода, расположенные по бокам от неё.

\begin{figure}[h]
    \centering
    \includegraphics[scale=0.5]{1994_mariqua_mouse/top_b_30.jpg}
    \includegraphics[scale=0.5]{1994_mariqua_mouse/bottom_b_30.jpg}
    \includegraphics[scale=0.4]{1994_mariqua_mouse/top_w_30.jpg}
    \includegraphics[scale=0.39]{1994_mariqua_mouse/bottom_w_30.jpg}

    \caption{Colani Trackball, вид сверху и снизу}
    \label{fig:TrinityTopBottom}
\end{figure}

Данный экземпляр мыши Trinity Mariqua подключается к компьютеру по последовательному интерфейсу, и три кнопки выполняют стандартную функцию левой, средней и правой кнопок мыши.
Однако существовали также варианты мыши с интерфейсом ADB, предназначенные для работы с компьютерами Apple: в этом случае левая кнопка играла роль основной (единственной в случае Apple) кнопки мыши,
правая кнопка фиксировала нажатие для облегчения перетаскивания (одно нажатие генерирует событие нажатия главной кнопки, второе — событие её отпускания), а нажатие средней кнопки генерировало
двойной клик главной кнопки мыши \cite{info_1}.

\begin{figure}[h]
    \centering
    \includegraphics[scale=0.5]{1994_mariqua_mouse/size_30.jpg}
    \caption{Colani Trackball на размерном коврике с шагом сетки 1 см}
    \label{fig:TrinitySize}
\end{figure}

Клавиатура мыши не активна по умолчанию, чтобы избежать случайных нажатий. Для активации цифровой клавиатуры необходимо нажать левую боковую кнопку, что также включает светодиод на передней панели устройства
(удобство данного способа индикации вызывает вопросы, так как при нормальном положении мыши пользователь не видит светодиоды).

Цифровая клавиатура включает 10 клавиш, представленных числами от 0 до 9, а также клавиши операций: «+», «-», «*», «/», «,». При каждом нажатии клавиши издается звуковой сигнал, что также может вызывать неудобства при частом использовании.

Для выхода из режима цифровой клавиатуры требуется повторное нажатие левой кнопки.

Правая боковая кнопка клавиатуры активирует вторую функцию клавиш (в этом варианте они работают как функциональные клавиши клавиатуры), и также позволяет переключаться между режимами. Важно отметить,
что нажатие правой боковой кнопки активирует этот режим, а нажатие любой из круглых кнопок на верхней стороне корпуса деактивирует его, возвращая устройство в режим мыши. Это связано с тем, что функциональные клавиши обычно не требуют
многократного использования подряд \cite{info_1}.

Рисунок \ref{fig:TrinityiHand} показывает типичное положение ладони на корпусе мыши. Следует признать, что за исключением миниатюрных круглых кнопок устройство является достаточно удобной компактной мышью с умеренно обтегаемой формой корпуса,
вписывающейся в тенденции середины 90-х годов.

\begin{figure}[h]
    \centering
     \includegraphics[scale=0.15]{1994_mariqua_mouse/hand_60.jpg}
    \caption{Colani Trackball в комплекте с моделью руки человека}
    \label{fig:TrinityiHand}
\end{figure}

Внутреннее устройство мыши показано на рисунке \ref{fig:TrinityInside}. Помимо дополнительной цифровой клавиатуры мышь является стандартным оптомеханическим устройством перемещения курсора, с узлом регистрации движения, характерным для середины 90-х годов.
Цифровой блок выполнен на отдельной плате, соединенной с основной при помощи гибкого шлейфа.

\begin{figure}[h]
    \centering
    \includegraphics[scale=0.8]{1994_mariqua_mouse/inside_30.jpg}
    \caption{Colani Trackball в разобранном виде}
    \label{fig:TrinityInside}
\end{figure}

\begin{thebibliography}{9}
    \bibitem {info_1} La souris ADB qui fait clavier avec bips et feux de position [Электронный ресурс] – Режим доступа: \url{https://www.journaldulapin.com/2022/06/06/souris-clavier-adb/}. Дата доступа: 16.12.2025.
\end{thebibliography}

\end{document}
