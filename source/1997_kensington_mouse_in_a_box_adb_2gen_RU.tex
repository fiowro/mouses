\documentclass[11pt, a4paper]{article}
\input{preamble.tex}
\switchlang{ru}
\begin{document}

\title{1997 "--- Kensington Mouse, ADB 2-nd generation}
\date{}
\maketitle
\selectlanguage{russian}

Однокнопочная мышь Kensington Mouse, продававшаяся и рекламировавшаяся как <<Mouse in a Box>> из-за упаковки в форме куба, была выпущена  американской компанией Kensington Computer Products Group в составе линейки из нескольких похожих по форме мышей \cite{kensingtonfamily, mouseinabox}. Данный экземпляр представляет собой версию Kensington Mouse второго поколения, предназначенную для компьютеров Apple с интерфейсом ADB (рис. \ref{mouseinaboxtopbottom}).

    \begin{figure}[h]
        \centering
    \includegraphics[scale=0.65]{1997_kensington_mouse_in_a_box_adb_2gen/pic_30.jpg}
        \caption{Внешний вид мыши Kensington Mouse второго поколения}
        \label{kensingtonmousepic}
    \end{figure}

Дизайн мыши следует концепции минимализма и активно использует геометрическую абстракцию. Корпус мыши аналогичен по форме Kensington Mouse первого поколения "--- он выполнен из белого пластика и имеет в основании овоид (симметричен относительно продольной оси и слегка заужен к передней, дальней от пользователя, стороне). На передней части корпуса присутствует единственная кнопка (у версии мыши для PC-совместимых компьютеров она разделена на две). Кнопка интегрирована в форму корпуса, занимает треть длины мыши и отделена от остальной части корпуса визуально различимым зазором в форме дуги, за которым присутствует логотип компании (рис. \ref{mouseinaboxtopbottom}). Логотип выполнен в тёмно-сером цвете, совпадающем с цветом кабеля и плоской шайбоподобной муфты, защищающей его от повреждения в месте выхода из корпуса. На нижней стороне корпуса можно найти ярлык с техническими данными мыши, две дугообразные низкофрикционные накладки (в отличие от круглых <<ножек>> Kensington mouse) и поворотное кольцо-защелку для извлечения шара и чистки мыши.

У данного экземпляра отсутствует код FCC ID, однако по аналогии с другими вариантами Kensington Mouse второго поколения можно датировать это устройство 1997 годом.

   \begin{figure}[h]
        \centering
    \includegraphics[scale=0.77]{1997_kensington_mouse_in_a_box_adb_2gen/top_30.jpg}
    \includegraphics[scale=0.77]{1997_kensington_mouse_in_a_box_adb_2gen/bottom_30.jpg}
        \caption{Изображение Kensington Mouse второго поколения, вид сверху и снизу} \label{mouseinaboxtopbottom}
    \end{figure}

Мышь имеет средние размеры, шар расположен строго в центре корпуса и имеет достаточно небольшой диаметр (очевидно, его размер ограничен высотой корпуса, также достаточно небольшой, как видно из рис. \ref{mouseinaboxsize}). Как отмечается в материалах компании, мышь <<подходит всем "--- левшам или правшам, с короткими пальцами или длинными>>. Действительно, как и в случае двухкнопочной Kensington Mouse, положение ладони на корпусе мыши выглядит очень естественным (рис. \ref{mouseinaboxhand}).

    \begin{figure}[h]
        \centering
    \includegraphics[scale=0.46]{1997_kensington_mouse_in_a_box_adb_2gen/size_15.jpg}
        \caption{Изображение Kensington Mouse второго поколения на размерном коврике с шагом сетки 1 см}
        \label{mouseinaboxsize}
    \end{figure}

Внутри Kensington Mouse можно увидеть классическую оптомеханическую систему, характерную для мышей середины 90-х годов (рис. \ref{mouseinaboxinside}). Плата незначительно отличается от более ранней версии в Kensington Mouse первого поколения. Однако, как и в первом поколении, в ней также реализован монтаж большей части элементов на нижнюю сторону. В качестве отличительной особенности внутреннего устройства мышей Kensington второго поколения можно отметить выполнение крепления роликов и <<купола>> шара как частей детали, представляющей собой дно корпуса: в моделях 1995 года этой цели служила дополнительная пластиковая деталь сложной формы, закрепленная непосредственно на плате, что, несмоненно, было менее экономным и менее технологичным решением.

    \begin{figure}[h!]
        \centering
    \includegraphics[scale=0.48]{1997_kensington_mouse_in_a_box_adb_2gen/hand_30.jpg}
        \caption{Изображение Kensington Mouse второго поколения c муляжом руки}
        \label{mouseinaboxhand}
    \end{figure}

    \begin{figure}[h!]
        \centering
    \includegraphics[width=\textwidth]{1997_kensington_mouse_in_a_box_adb_2gen/inside_30.jpg}
        \caption{Изображение Kensington Mouse второго поколения в разобранном виде}
        \label{mouseinaboxinside}
    \end{figure}

\begin{thebibliography}{9}
    \bibitem{kensingtonfamily} The first family in mice // PC Magazine, February 10, 1998 -- P. 270 \url{https://books.google.by/books?id=fFrjSBw0w14C&lpg=PA270&dq=kensington%20mouse%20in%20a%20box&hl=ru&pg=PA270#v=onepage&q&f=false}
    \bibitem {mouseinabox} Kensington: Mouse in a Box -- kensington.com. January 06, 1997 \url{https://web.archive.org/web/19970106170908/http://www.kensington.com/prod/mice/mice3d.html} 
\end{thebibliography}

\end{document}
