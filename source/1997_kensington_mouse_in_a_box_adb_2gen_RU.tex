\documentclass[11pt, a4paper]{article}
\input{preamble.tex}
\switchlang{ru}
\begin{document}

\title{1997 "--- Kensington Mouse in a box, ADB 2-nd generation version}
\date{}
\maketitle
\selectlanguage{russian}

Однокнопочная мышь Kensington Mouse in a Box является частью той же линейки мышей, что и Kensington mouse, но предназначена для работы не с IBM-совместимыми компьютерами, а с компьютерами Apple, к которым она подключается по шине ADB. Эта однокнопочная мышь имеет такой же стильный симметричный корпус из белого пластика овальной формы (рис. \ref{mouseinaboxtopbottom}), как и Kensington mouse. Причиной её специфического названия является упаковка: будучи округлой, эта мышь продавалась в картонной коробке, имевшей форму куба.

    \begin{figure}[h]
        \centering
    \includegraphics[scale=0.5]{1995_kensington_mouse_in_a_box_adb_2gen/pic_30.jpg}
        \caption{Внешний вид мыши Kensington Mouse in a box}
        \label{kensingtonmousepic}
    \end{figure}

    Дизайн мыши ещё более четко следует концепции минимализма. На верхней части корпуса можно увидеть вписанную в форму корпуса единственную кнопку, занимающую треть длины мыши и отделенную от остальной части визуально различимым дугообразным зазором, а также логотип компании. 
    
   \begin{figure}[h]
        \centering
    \includegraphics[scale=0.7]{1997_kensington_mouse_in_a_box_adb_2gen/top_30.jpg}
    \includegraphics[scale=0.7]{1997_kensington_mouse_in_a_box_adb_2gen/bottom_30.jpg}
        \caption{Изображение Mouse in a box, вид сверху и снизу} \label{mouseinaboxtopbottom}
    \end{figure}
    
    Как и у Kensington Mouse, логотип выполнен в тёмно-сером цвете, совпадающем с цветом кабеля и плоской шайбоподобной муфты, защищающей его от повреждения в месте выхода из корпуса.
    На нижней стороне корпуса можно видеть две дугообразные низкофрикционные накладки (в отличие от круглых <<ножек>> Kensington mouse) и поворотное кольцо-защелку, которое можно снять для извлечения шара и чистки мыши. Шар достаточно небольшого диаметра (очевидно, его размеры ограничены высотой корпуса), расположен в центре мыши. На ярлыке с информацией от производителя отсутствует код FCC ID; однако в \cite{denver} можно увидеть экземпляр аналогичной однокнопочной мыши, с нижней частью, идентичной Kensington mouse и кодом, также зарегистрированным в 1995 году. Таким образом, данный экземпляр является поздней ревизией мыши ADB "--- предположительно, он датируется 1997 годом.
    
    Мышь имеет средние размеры (рис. \ref{mouseinaboxsize}), и как отмечается в материалах компании, подходит всем "--- левшам или правшам, с короткими пальцами или длинными. Действительно, как и в случае двухкнопочной Kensington Mouse, положение ладони на корпусе мыши выглядит очень естественным (рис. \ref{mouseinaboxhand}).
    
    \begin{figure}[h]
        \centering
    \includegraphics[scale=0.4]{1997_kensington_mouse_in_a_box_adb_2gen/size_30.jpg}
        \caption{Изображение Mouse in a box на размерном коврике с шагом сетки 1 см}
        \label{mouseinaboxsize}
    \end{figure}   
    
    \begin{figure}[h]
        \centering
    \includegraphics[scale=0.4]{1997_kensington_mouse_in_a_box_adb_2gen/hand_30.jpg}
        \caption{Изображение Mouse in a box c муляжом руки на размерном коврике с шагом сетки 1 см}
        \label{mouseinaboxhand}
    \end{figure}

    Согласно информации на сайте \cite{mouseinabox}, мышь не требует установки программного обеспечения, благодаря использованию интерфейса ADB и прямого подключения к компьютеру через порт в корпусе клавиатуры.

    ADB (Apple Desktop Bus "--- англ. шина для настольных компьютеров Apple) — устаревший в настоящее время компьютерный порт и одноименная последовательная шина ввода-вывода данных. 
    
    Этот интерфейс был создан для подсоединения медленных устройств (таких, как компьютерная клавиатура и мышь) к компьютерам Apple Macintosh. Им комплектовались все настольные компьютеры Apple вплоть до 1999 года. 
    ADB была разработана Стивом Возняком, искавшим себе новый проект в середине 1980-х годов. Кто-то предложил, чтобы он создал новую систему связи для устройств типа мыши и клавиатуры, которая была бы недорогой и требовала бы только единственное цепочечное соединение кабеля, результатом чего и стало появление данного интерфейса.

    Первая система, в которой использовалась шина ADB, была Apple IIgs. ADB впоследствии использовалась на всех настольных (а также и мобильных) компьютерах Apple Macintosh, начиная с Macintosh II и Macintosh SE, прежде чем была заменена на USB, начиная с компьютеров iMac 1998 года выпуска. ADB также использовалась в ряде других микрокомпьютерах на базе процессора Motorola 680x0, выпускавшихся Sun, HP, NeXT и некоторыми другими производителями.

    Последними устройствами, использовавшими ADB (как внутренний интерфейс для встроенных клавиатуры и тачпада) были PowerBook и iBook, с февраля 2005 года окончательно перешедшие на USB. 

    Согласно философии промышленного дизайна Apple, ADB должна была быть максимально простой в использовании и недорогой для создания. Подходящий разъём был найден в форме 4-штырькового разъёма miniDIN. Разъёмы были малогабаритными, широко доступными (они также использовались для S-Video), и могли быть вставлены только в правильное положение из-за выемок на кольцевой части разъёма.

    Протокол ADB требовал единственный провод для передачи данных (он обзоначался «ADB»). Два других провода использовались для подачи питания (+5В и земля). 5-вольтовый провод допускал токи до 500 мА, но требовал, чтобы устройства использовали только по 100 мА каждое. ADB также включал провод «PSW», который был подключён непосредственно к блоку питания компьютера. Это было сделано для того, чтобы разрешить клавише на клавиатуре выводить компьютер из ждущего режима без использования программного обеспечения для интерпретации сигнала.

    Большинство последовательных цифровых интерфейсов используют отдельный тактовый провод, чтобы сигнализировать прибытие индивидуальных битов данных. Поскольку ADB была разработана, чтобы быть дешёвой, Возняк признал, что один единственный провод имел достаточную полосу пропускания, чтобы нести оба сигнала. Кроме того, было экономично декодировать тактовые сигналы и сигналы данных, чтобы использовать более дешёвые кабели. Декодирующий приёмопередатчик был доступен только по запросу производителя оборудования, поскольку Apple предпочитала тесное сотрудничество с вендорами устройств. Есть вероятность, что Apple продала эти аппаратные средства ниже их стоимости, чтобы поощрить развитие периферийных устройств. 

    Внутри Kensington Mouse In a Box можно увидеть классическую оптомеханическую систему, характерную для мышей первой половины 90-х годов (рис. \ref{mouseinaboxinside}). Плата незначительно отличается от более ранней версии, представленной в Kensington Mouse. Однако в ней также реализован монтаж большей части элементов на нижнюю сторону, а свободные посадочные места под микропереключатели, идентичные таковым у Kensington Mouse, показывают, что у этой ревизии также существовал двухкнопочный вариант.
   
    \begin{figure}[h]
        \centering
    \includegraphics[scale=0.9]{1997_kensington_mouse_in_a_box_adb_2gen/inside_30.jpg}
        \caption{Изображение Mouse in a box в разобранном виде}
        \label{mouseinaboxinside}
    \end{figure}

\begin{thebibliography}{9}
    
\end{thebibliography}

\end{document}
