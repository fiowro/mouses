\documentclass[11pt, a4paper]{article}
\input{preamble.tex}
\switchlang{en}
\begin{document}

\title{1986 -- Apple Desktop Bus Mouse}
\date{}
\maketitle
\selectlanguage{english}
The Apple Desktop Bus Mouse (fig. \ref{fig:pic}) was introduced by Apple in 1986, along with its new interface bus for peripherals, the Apple Desktop Bus (ADB). It was the first mouse officially designed in the ``Snow White'' design language, developed for Apple by Frog Design, the company of industrial designer Hartmut Esslinger \cite{apple1, apple2}. 
Along with the ``Snow White'' concept \cite{frog}, the mouse body, including the single button, has a solid light gray color (so-called ``platinum'', and not white, despite what the name suggests), while the cables and connectors have a darker ``smoke gray'' shade (fig. \ref{fig:pic}).

\begin{figure}[h]
    \centering
    \includegraphics[scale=0.5]{1986_apple_desktop_bus_mouse/pic_30.jpg}
    \caption{Apple Desktop Bus Mouse} \label{fig:pic}
\end{figure}

The Apple Desktop Mouse's body is rectangular at the base, with a wedge-shaped longitudinal cross-section, and looks rather unnatural (fig. \ref{fig:top}). The mouse's single button is large and rectangular. It fits within the contours of the body, but is sharply defined to make it more distinguishable for the user. Also there is the Apple logo on top, located in the left corner, from the side of the body closest to the user.  On the underside of the body there is a contrasting black rotating ring (in some other Apple Desktop Mouse variations it has the same color as the body), used to remove the cleaning ball for cleaning, a manufacturer's label with technical data, and low-friction pads to facilitate mouse movement (one of the two pads is located on the rotating ring).

\begin{figure}[h]
    \centering
    \includegraphics[scale=0.65]{1986_apple_desktop_bus_mouse/top_30.jpg}
    \includegraphics[scale=0.65]{1986_apple_desktop_bus_mouse/bottom_30.jpg}
    \caption{Apple Desktop Bus Mouse, top and bottom views} \label{fig:top}
\end{figure}

Overall, this model retained the rectangular shape of the previous Apple mouse, but was given a shorter height and the already mentioned ``wedge'' with a rib that visually separates the third of the body closest to the user (fig. \ref{fig:size}).

\begin{figure}[h]
    \centering
    \includegraphics[scale=0.4]{1986_apple_desktop_bus_mouse/size_30.jpg}
    \caption{Apple Desktop Bus Mouse on a graduated pad with a grid step of 1~cm} \label{fig:size}
\end{figure}

The eccentric technical design made the mouse instantly recognizable, but it had a negative impact on its ergonomics.
The angular body doesn't provide sufficient palm support (fig. \ref{fig:hand}), and overall, the only ergonomic advantage is the large button size.

\begin{figure}[h]
    \centering
    \includegraphics[scale=0.4]{1986_apple_desktop_bus_mouse/hand_30.jpg}
    \caption{Apple Desktop Bus Mouse with a human hand model} \label{fig:hand}
\end{figure}

The internal structure of the mouse is shown in fig. \ref{fig:inside}. As mentioned, the mouse has had several modifications, produced by several companies (ALPS Electric, Logitech, Mitsumi). Most mice are optomechanical, although a variant based on closed mechanical encoders from ALPS \cite{apple3} is also known. This model was also manufactured by ALPS, but its implementation uses optomechanical encoders in a closed, non-disassemblable design.

\begin{figure}[h]
    \centering
    \includegraphics[scale=0.6]{1986_apple_desktop_bus_mouse/inside_60.jpg}
    \caption{Apple Desktop Bus Mouse disassembled} \label{fig:inside}
\end{figure}


\begin{thebibliography}{9}
    \bibitem {apple1} Fuller M. Apple Desktop Bus (ADB) Mouse (A9M0331, 1986) -- mattjfuller.com \url{http://mattjfuller.com/apple-desktop-bus-adb-mouse-a9m0331-1986/}
    \bibitem {apple2} Phin C. When mice had balls: Remembering the Apple Desktop Bus Mouse -- macworld.com \url{https://www.macworld.com/article/225298/when-mice-had-balls-remembering-the-apple-desktop-bus-mouse.html}
    \bibitem {frog} Frog Design -- Wikipeda. \url{https://en.wikipedia.org/wiki/Frog_Design}
    \bibitem {apple3} Apple Desktop Bus Mouse Parts -- Apple Rescue of Denver \url{https://applerescueofdenver.com/products-page/macintosh-to-powerpc/keyboards-mice-joysticks-macintosh/apple-desktop-bus-mouse-parts/}
\end{thebibliography}

\end{document}
