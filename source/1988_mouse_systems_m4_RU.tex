\documentclass[11pt, a4paper]{article}
\input{preamble.tex}
\switchlang{ru}
\begin{document}

\title{1988 "--- Mouse Systems M4 PC mouse}
\date{}
\maketitle
\selectlanguage{russian}
Mouse Systems M4 mouse (рис. \ref{fig:mscM4Pic}) "--- характерный манипулятор компании Mouse Systems, созданной Стивеном Киршем в 1982 году для производства изобретенной им оригинальной оптической мыши. Данная версия "--- представитель четвертого (последнего и самого массового) поколения мышей Mouse Systems конструкции Кирша, адаптированный для работы с последовательным интерфейсом IBM PC, RS-232.

\begin{figure}[h]
    \centering
    \includegraphics[scale=0.5]{1988_mouse_systems_m4/pic_30.jpg}
    \caption{Мышь Mouse Systems M4}
    \label{fig:mscM4Pic}
\end{figure}

Как всем ранним оптическим мышам, данному манипулятору требуется коврик с отражающей сеткой (рис. \ref{fig:mscM4Pad}). Это традиционный металлический коврик Mouse Systems с решеткой из вертикальных и горизонтальных линий, соответствующих двум разным длинам волн (в оптическом и в инфракрасном диапазоне).

\begin{figure}[h]
    \centering
    \includegraphics[scale=0.4]{1988_mouse_systems_m4/pad_30.jpg}
    \caption{Мышь Mouse Systems M4 на комплектном коврике}
    \label{fig:mscM4Pad}
\end{figure}

Мышь выполнена в стандартном для Mouse Systems плоском прямоугольном корпусе с тремя вытянутыми закругленными кнопками. На верхней стороне корпуса присутствует табличка с названием компании MOUSE SYSTEMS, набранным курсивом, на нижней стороне "--- то же название и эмблема в виде двух треугольников, стилизованных под изображение мыши и экранного курсора (рис. \ref{fig:mscM4TopBottom}). В отличие от ранних моделей мышей Mouse Systems с интерфейсом RS-232 (например, мыши M2) Мышь M4 использует менее энергоемкие светодиоды и потому не требует дополнительного блока питания.

Изучение кода FCC ID по базе данных Федеральной комиссии по связи США показывает, что мышь была выпущена компанией Mouse Systems в 1988 году.

\begin{figure}[h]
    \centering
    \includegraphics[scale=0.6]{1988_mouse_systems_m4/top_30.jpg}
    \includegraphics[scale=0.6]{1988_mouse_systems_m4/bottom_30.jpg}
    \caption{Мышь Mouse Systems M4, вид сверху и снизу}
    \label{fig:mscM4TopBottom}
\end{figure}

Некрупный, подчеркнуто плоский корпус (рис. \ref{fig:mscM4Size}) наглядно иллюстрирует отсутствие шара и других механических элементов \cite{pcmag}; он оказался достаточно удачен и не претерпел изменений в размерах и форме начиная с первого поколения мышей компании \cite{old}.

\begin{figure}[h]
    \centering
    \includegraphics[scale=0.5]{1988_mouse_systems_m4/size_30.jpg}
    \caption{Mouse Systems M4 на размерном коврике с шагом сетки 1~см}
    \label{fig:mscM4Size}
\end{figure}

Мышь имеет достаточно эргономичную по меркам 80-х годов форму (рис. \ref{fig:mscM4Hand}) благодаря достаточно удачной форме с закругленными углами и большими кнопками. Отсутствие опоры под запястье отчасти компенсируется малой высотой корпуса мыши, а кнопки входят в число самых эргономичных среди манипуляторов восьмидесятых.

\begin{figure}[h]
    \centering
    \includegraphics[scale=0.35]{1988_mouse_systems_m4/hand_30.jpg}
    \caption{Mouse Systems M4 с моделью руки человека}
    \label{fig:mscM4Hand}
\end{figure}

Внутреннее устройство показано на рис. \ref{fig:mscM4Inside}.
В соответствии с придуманной Киршем схемой, излучаемый светодиодами свет отражается от коврика, затем от блестящей фольгированной пластины в верхней части корпуса мыши, и наконец попадает на две линейки фотоприемников, одна из которых расположена вдоль продольной оси, а другая "--- вдоль поперечной. Один из светодиодов излучает свет в оптическом диапазоне, в котором видны нанесенные на коврик поперечные линии, а другой "--- в инфракрасном, в котором считываются не поперечные, а продольные линии. В результате движения чередование линий считывается соответствующей линейкой фотоприемников. На рис. \ref{fig:mscM4Inside} можно увидеть две линейки фотоприемников (по 4 в каждой): это  усовершенствованная по сравнению с исходным вариантом Стивена Кирша конструкция (в первом поколении мышей Mouse Systems использовалась единственная матрица фотоприемников размером $2 \times 2$, недостатком которой был период калибровки мыши после каждого включения, во время которого микроконтроллер выполнял автоподбор корректных пороговых значений срабатывания фотодатчиков) \cite{bio}.

\begin{figure}[h]
    \centering
    \includegraphics[scale=0.6]{1988_mouse_systems_m4/inside_30.jpg}
    \caption{Mouse Systems M4 mouse в разобранном виде}
    \label{fig:mscM4Inside}
\end{figure}


\begin{thebibliography}{9}
\bibitem {pcmag} Why you should buy the mouse with no moving parts [adv.] // PC MAGAZINE, V. 7, No. 3, February 16, 1988. P. 205. \url{https://archive.org/details/PC-Mag-1988-02-16/page/n203/mode/2up}
\bibitem {old} Mouse Systems -- oldmouse.com  \url{https://web.archive.org/web/20211205082304/http://oldmouse.com/mouse/mousesystems/}
\bibitem {bio} Perry S.T. Steve Kirsch -- IEEE Spectrum. 01 AUG 2000 \url{https://spectrum.ieee.org/steve-kirsch}
\end{thebibliography}
\end{document}
