\documentclass[11pt, a4paper]{article}
\input{preamble.tex}
\switchlang{ru}
\begin{document}

\title{1986 "--- NEC Crystal mouse}
\date{}
\maketitle
\selectlanguage{russian}
В сентябре 1986 года корпорация NEC анонсировала рабочую станцию EWS 4800, работавшую под управлением ОС UNIX. Эта машина была спроектирована как рабочая станция для инженеров, для повышения эффективности решения таких задач, как разработка программного обеспечения, автоматизированное проектирование, научные и инженерные вычисления, сбор и анализ экспериментальных данных \cite{yt}. Она была оснащена графическим многооконным интерфейсом и большим 20-дюймовым дисплеем разрешения $1280 \times 1024 \times 256$. В комплекте с этой мощной рабочей станцией поставлялась мышь NEC Crystal Mouse (рис. \ref{fig:NECCrystalPic}).

\begin{figure}[h]
    \centering
    \includegraphics[scale=0.7]{1986_nec_crystal_mouse/necNorm_30.jpg}
    \caption{NEC Crystal Mouse}
    \label{fig:NECCrystalPic}
\end{figure}

Как всем ранним оптическим мышам, данному манипулятору требуется отражающий свет коврик с нанесенной на него сеткой. В отличие от традиционных <<зеркальных>> ковриков Mouse Systems с решеткой из вертикальных и горизонтальных линий, соответствующих двум разным длинам волн, металлический коврик Nec Crystal Mouse имеет темную глянцевую поверхность (рис. \ref{fig:NecCrystalPad}).

\begin{figure}[h]
    \centering
    \includegraphics[scale=0.5]{1986_nec_crystal_mouse/necPad_30.jpg}
    \caption{NEC Crystal Mouse на комплектном коврике}
    \label{fig:NecCrystalPad}
\end{figure}

На верхней стороне корпуса выделено название мыши <<Crystal Mouse>> с двумя треугольными стилизованными мышками в очертаниях и сплошных линиях. Нижняя сторона показывает, что это оптическая мышь (рис. \ref{NecCrystalTopAndBottom}), во многом повторяющая внешние конструктивные решения мышей Mouse Systems того же периода \cite{photo}.

\begin{figure}[h]
    \centering
    \includegraphics[scale=0.75]{1986_nec_crystal_mouse/nectop_60.jpg}
    \includegraphics[scale=0.75]{1986_nec_crystal_mouse/necbottom_60.jpg}
    \caption{NEC Crystal Mouse, вид сверху и снизу}
    \label{NecCrystalTopAndBottom}
\end{figure}

В плане размера манипулятор представляет собой типичное для 80-х годов оптическое устройство управления курсором (рис. \ref{fig:NecCrystalSize})

\begin{figure}[h]
    \centering
    \includegraphics[scale=0.45]{1986_nec_crystal_mouse/necSize_30.jpg}
    \caption{Изображение NEC Crystal Mouse на размерном коврике с шагом сетки 1~см}
    \label{fig:NecCrystalSize}
\end{figure}

В плане эргономики во внешнем виде Crystal Mouse прослеживается ярко выраженный индустриальный дизайн. При этом большое количество углов и плоских граней отчасти компенсируется закругленными стыками граней в ближней к пользователю части корпуса, и выпуклыми продолговатыми кнопками, удобно расположенными в зоне досягаемости пальцев (рис. \ref{fig:NecCrystalHand}).

\begin{figure}[h]
    \centering
    \includegraphics[scale=0.36]{1986_nec_crystal_mouse/necHand_30.jpg}
    \caption{NEC Crystal Mouse с моделью руки человека}
    \label{fig:NecCrystalHand}
\end{figure}

Внутреннее устройство манипулятора показано на рис. \ref{fig:NecCrystalInside}, где можно увидеть оригинальную конструкцию оптической мыши, отличающуюся от большинства оптических мышей 80-х годов, копировавших разработку компании Mouse Systems. Излучаемый светодиодами свет отражается от коврика, затем переотражается от зеркал внутри мыши (пара прямоугольных блоков с синей внешней стороной в центре конструкции), и наконец попадает на линейки фотоприемников, одна из которых расположена вдоль продольной оси, а другая --- вдоль поперечной. При этом у Nec Crystal Mouse светодиоды (а соответственно и зеркала) расположены под разными углами к поверхности коврика.

\begin{figure}[h]
    \centering
    \includegraphics[scale=0.75]{1986_nec_crystal_mouse/necraz_60.jpg}
    \caption{NEC Crystal Mouse в разобранном виде}
    \label{fig:NecCrystalInside}
\end{figure}

Коврик представляет собой металлическую пластину с полупрозрачным покрытием, скрывающим решетку. Продольные и поперечные линии решетки находятся на разной высоте, и их заметность зависит от того, под каким углом к поверхности расположена оптическая ось. Рисунок \ref{fig:NecCrystalPad} показывает Nec Crystal Mouse на коврике от более поздней мыши аналогичной конструкции, Kokuyo EAM-102; как можно судить по фотографии, приведенной в \cite{yt}, оригинальный коврик Nec был крупнее, и вероятно имел меньшее разрешение сетки.

Также рисунок \ref{fig:NecCrystalInside} позволяет заключить, что мышь подключается к компьютеру по разновидности последовательного интерфейса (об этом свидетельствует ее кабель из четырех проводов, из которых, согласно маркировке на печатной плате, два отвечают за прием и передачу битов данных, а два обеспечивают питание).

\begin{thebibliography}{9}
\bibitem {yt} NEC EWS4800 \url{http://museum.ipsj.or.jp/en/computer/work/0003.html}
\bibitem {photo} Graphics NEC EWS4800 \url{http://www.cs.ce.nihon-u.ac.jp/facility/exp-gazo.html}
\end{thebibliography}
\end{document}
