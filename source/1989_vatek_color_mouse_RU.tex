\documentclass[11pt, a4paper]{article}
\input{preamble.tex}
\switchlang{ru}
\begin{document}

\title{1985 "--- Vatek Color-Mouse}
\date{}
\maketitle
\selectlanguage{russian}

Мышь Vatek Color-Mouse была выпущена в продажу в 1989 году компанией Vatek USA Inc. "--- дистрибьютором компьютерного оборудования, зарегистрированным в Калифорнии годом ранее. По утверждению создателей мыши, в результате этого Vatek стала первой компанией, официально представившей на рынке мышь в многоцветном исполнении. Этот приоритет нельзя назвать бесспорным: в рекламных материалах Hawley Mouse House 1982 года красуется рисунок  Mark II X063X Mouse в десятках цветовых сочетаний. Но реально цветовых вариантов Mark II было выпущено не так уж много; кроме того, Hawley Mouse House не делала упор на многоцветность, а Vatek не только позиционировала свое изделие в этом качестве, но даже закрепила его в названии мыши. Наконец, утверждение Vatek можно понимать и буквально, в том смысле, что компания первая представила на рынке компьютерную мышь под названием <<color mouse>> (англ. <<цветная мышь>>), с чем действительно не приходится спорить.

\begin{figure}[h]
   \centering
    \includegraphics[scale=0.45]{1989_vatek_color_mouse/pic_30.jpg}
    \caption{Microsoft Gray-eyed Mouse}
    \label{fig:VatekColorPic}
\end{figure}

Корпус устройства изготавливался из пластика красного, зеленого, желтого или синего цвета, но иногда и из стандартного бежевого \cite{mouses}, и всегда оснащался двумя контрастными серыми кнопками. Как можно видеть (рис. \ref{fig:VatekColorPic}), данный экземпляр <<цветной>> мыши является именно бежевым.

\begin{figure}[h]
    \centering
    \includegraphics[scale=0.55]{1989_vatek_color_mouse/top_30.jpg}
    \includegraphics[scale=0.55]{1989_vatek_color_mouse/bottom_30.jpg}
    \caption{Microsoft Gray-eyed Mouse, вид сверху и снизу}
    \label{fig:VatekColorTopAndBottom}
\end{figure}

Форма мыши копирует вышедшую в 1985 году мышь Microsoft второго поколения (известную под названием <<сероглазой>> мыши из-за цвета кнопок). Верхняя часть корпуса Vatek Color-Mouse воспроизводит корпус мыши Microsoft вплоть до идентичности, за исключением более широких кнопок, смыкающихся друг с другом (рис. \ref{fig:VatekColorTopAndBottom}). При этом главная кнопка имеет продольный выступ для легкой тактильной идентификации ее края, что позаимствовано уже из третьего поколения мышей Microsoft.
Снизу можно видеть шар с резиновым покрытием, фиксирующее кольцо, сдвигаемое для извлечения шара и чистки мыши, четыре низкофрикционные накладки, а также наклейку с техническими данными (код FCC упоминается, но не приведен, однако встречаются и экземпляры данной мыши, промаркированные кодом <<E6Q5J8MOUSEX11>>). Нижняя сторона уже не копирует <<сероглазую>> мышь Microsoft (а точнее, не копирует лежащий в ее основе типовой дизайн ALPS), если не считать общего силуэта и положения шара в задней части корпуса.

\begin{figure}[h]
    \centering
    \includegraphics[scale=0.4]{1989_vatek_color_mouse/size_30.jpg}
    \caption{Microsoft Gray-eyed Mouse на размерном коврике с шагом сетки 1~см}
    \label{fig:VatekColorSize}
\end{figure}

По понятной причине размеры мыши позаимствованы у Microsoft (рис. \ref{fig:VatekColorSize}), в случае которой они определялись размерами типового узла ALPS. Помимо двухкнопочной Color-Mouse в то же время в таком же корпусе выпускался трехкнопочный вариант мыши: он продавался под различными брэндами, как известными так и нет, и различными названиями: Z-Nix Super Hi-Res Mouse, ProCorp Serial Mouse, SmarTEAM Smart Mouse и др.

Эргономика Color-Mouse также во многом повторяет особенности мыши Microsoft за счет того же самого высокого корпуса, удобного для захвата узкой ладонью, и угловых кнопок, немного отличаясь в лучшую сторону за счет того, что кнопки Vatek шире.
Такая угловая форма кнопок придумана Microsoft как логическое развитие мыши первого поколения, кнопки которой помещались на корпусе спереди и вызывали критику со стороны некоторых обозревателей из-за возможности нечаянно сдвинуть мышь, просто нажимая на них. В рекламных материалах <<сероглазой>> мыши Microsoft упоминается, что <<огибающие корпус командные кнопки спроектированы таким образом, чтобы естественно помещаться в ладони любого размера>> \cite{mouses}, но очевидно, такое двойное положение кнопок решало и проблему ошибочных перемещений, позволяя нажимать на них сверху тем, кому это удобнее (рис. \ref{fig:VatekColorHand}). Вслед за Microsoft, эта угловая форма кнопок появилась в некоторых других мышах и больше всего сходства по понятным причинам демонстрируют Vatek Сolor-Mouse и её трехкнопочный близнец, известный под именем ProCorp Serial Mouse и др.

\begin{figure}[h]
    \centering
    \includegraphics[scale=0.4]{1989_vatek_color_mouse/hand_30.jpg}
    \caption{Microsoft Gray-eyed Mouse с моделью руки человека}
    \label{fig:VatekColorHand}
\end{figure}

Color-Mouse подключается к компьютеру через последовательный порт с помощью 9-контактного разъема и имеет довольно умеренное разрешение "--- 250 точек на дюйм \cite{dpi}.

Изучение кода FCC ID по базе данных Федеральной комиссии по связи США показывает, что мышь была изготовлена компанией Jow Dian Enterprise Co Ltd., зарегистрированной в Калифорнии. Дата регистрации кода FCC ID "--- 1988 год, однако аналогичным кодом маркировалась и трехкнопочная мышь. Ранее выпуском двух поколений под одним кодом FCC ID отметилась Microsoft (самой первой <<зеленоглазой>> мыши и <<сероглазой>> мыши, у которой Jow Dian Enterprise позаимствовала форму корпуса). Сравнение маркировки печатных плат Color-Mouse и ProCorp Serial Mouse показывает, что Vatek Color-Mouse была второй ревизией. Поэтому, а также согласно \cite{mouses}, мышь датирована 1989 годом.

\begin{figure}[h]
    \centering
    \includegraphics[scale=0.8]{1989_vatek_color_mouse/inside_60.jpg}
    \caption{Microsoft Gray-eyed Mouse в разобранном виде}
    \label{fig:VatekColorInside}
\end{figure}

Внутреннее устройство мыши показано на рис. \ref{fig:VatekColorInside}. По фотографии видно, что она является мышью с механическим энкодером, как и ее прототип от Microsoft, однако вместо более надежных закрытых энкодеров ALPS применен бюджетный вариант - открытый дисковый энкодер, встречающийся в мышах конца 80-х годов, чаще предназначенных для домашних компьютеров того времени, более простых и дешевых, чем IBM PC.

\begin{thebibliography}{9}
\bibitem{mouses} Vatek Color Mouse // Home Office Computing, December, 1989. -- P. 59 \url{https://archive.org/details/home-office-computing-december-1989/page/58/mode/2up}
\bibitem{company} Vatek USA * RMI Company Profile. \url{https://web.archive.org/web/19981201090808/http://www.vatek.com:80/company.html}
\bibitem{dpi} Color-Mouse // Dataquest: DQ, India: Cyber Media, 1990. p. 41.
\end{thebibliography}
\end{document}
