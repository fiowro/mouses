\documentclass[11pt, a4paper]{article}
\input{preamble.tex}
\switchlang{ru}
\begin{document}

\title{1995 "--- Kensington Thinking Mouse}
\date{}
\maketitle
\selectlanguage{russian}

Мышь Thinking Mouse была выпущена в 1995 году американской компанией Kensington Computer Products Group в составе линейки из нескольких похожих по форме мышей \cite{kensingtonfamily}. Kensington Thinking Mouse --- как старшая модель в этой линейке --- как и другие мыши и трекболы компании, выпускалась в версии для PC (рис. \ref{thinkingmousepic}) и для компьютеров Apple с интерфейсом ADB \cite{thinkingmouse}.

    \begin{figure}[h]
        \centering
    \includegraphics[scale=0.65]{1995_kensington_thinking_mouse/pic_30.jpg}
        \caption{Внешний вид мыши Thinking Mouse}
        \label{thinkingmousepic}
    \end{figure}

В целом, Thinking Mouse оформлена в стиле остальных мышей Kensington, с незначительными модификациями.

Дизайн мыши следует концепции минимализма и активно использует геометрическую абстракцию. Корпус мыши --- белый сверху и серый снизу, имеет в основании овоид (симметричен относительно продольной оси и слегка заужен к передней, дальней от пользователя, стороне). На передней части корпуса расположены четыре кнопки: у края корпуса находятся большие по размеру левая и правая кнопки, а ближе к пользователю "--- пара дополнительных кнопок, в сумме образующих эллипс (сознательное использование математического дизайна подчеркивает упаковка Thinking mouse, где рисунок корпуса мыши снабжен дополнительными линиями построения, показывающими лежащие в его основе окружности и дуги \cite{box}). Кнопки занимают треть корпуса мыши, почти полностью интегрированы в его форму (за исключением того, что стык основной и дополнительной кнопки образует небольшой уступ) и отделены от остальной части и друг от друга визуально различимым зазором. Кроме них на верхней части корпуса нет никаких элементов, за исключением логотипа компании (рис. \ref{thinkingmousetopbottom}). Нижняя часть  мыши выполнена из серого пластика (рис. \ref{thinkingmousesize}), в тон логотипу, кабелю и плоской шайбоподобной муфте, защищающей кабель от повреждения в месте выхода из корпуса. Нижняя часть корпуса имеет такую же компоновку, что у представителей линейки Kensington Mouse (она содержит такой же ярлык с техническими данными, пять круглых накладок из низкофрикционного материала и поворотное кольцо-защелку для извлечения шара и чистки мыши). Однако, в отличие от младших моделей, нижняя часть корпуса Thinking Mouse изготовлена из серого <<прорезиненного>> материала <<Kensington EasyGrip>> \cite{thinkingmouse}, который во многом похож на современный пластик с покрытием софт-тач и, по заявлению производителя, минимизирует вероятность соскальзывания пальцев с боков мыши (рис. \ref{thinkingmousesize}).

    \begin{figure}[h]
        \centering
    \includegraphics[scale=0.76]{1995_kensington_thinking_mouse/top_30.jpg}
    \includegraphics[scale=0.76]{1995_kensington_thinking_mouse/bottom_30.jpg}
        \caption{Изображение Thinking Mouse Mouse вид сверху и снизу}
        \label{thinkingmousetopbottom}
    \end{figure}

Изучение кода FCC ID по базе данных Федеральной комиссии по связи США позволяет датировать Thinking Mouse 1995 годом.

     \begin{figure}[h]
        \centering
    \includegraphics[scale=0.44]{1995_kensington_thinking_mouse/size_15.jpg}
        \caption{Изображение Thinking Mouse на размерном коврике с шагом сетки 1 см}
        \label{thinkingmousesize}
    \end{figure}

Как и другие мыши Kensington, Thinking Mouse комплектовалась программным обеспечением Kensington Mouse Works. Программа позволяла выполнять тонкую настройку (вплоть до задания нелинейного ускорения мыши с помощью кривой или табличных данных), а также назначить различные функции кнопкам (по умолчанию две дополнительные кнопки дублируют основные, то есть выполняют точно такой же клик левой и правой кнопки мыши). Разработчики Mouse Works уделили большое внимание эргономике. В руководстве пользвоателя содержится раздел, подробно объясняющий правильное положение руки на мыши. Кроме того, среди назначаемых кнопкам функций присутствуют такие варианты, как одинарный клик с фиксацией (для облегчения перетаскивания объектов), а также генерация события двойного клика при однократном нажатии на кнопку \cite{italian}, что должно было быть полезным в первую очередь для облегчения работы людям с ограниченной подвижностью. Поэтому и при обычном использовании мышь сохраняет хорошую эргономичность, несмотря на то, что добавление двух дополнительных кнопок является скорее спорным решением (из-за них пропорционально уменьшена площадь основных кнопок, а кроме того, расположение дополнительных кнопок, хотя и минимизирует вероятность случайного нажатия, одновременно приводит к тому, что их не слишком удобно нажимать при нормальном положении руки на корпусе, как можно видеть на рис. \ref{thinkingmousehand}).

    \begin{figure}[h!]
        \centering
    \includegraphics[scale=0.46]{1995_kensington_thinking_mouse/hand_30.jpg}
        \caption{Изображение Thinking Mouse с муляжом руки}
        \label{thinkingmousehand}
    \end{figure}

    \begin{figure}[h!]
    \centering
    \includegraphics[width=\textwidth]{1995_kensington_thinking_mouse/inside_30.jpg}
        \caption{Изображение Thinking Mouse в разобранном виде}
        \label{thinkingmouseinside}
    \end{figure}


Мышь в разобранном виде показана на рисунке \ref{thinkingmouseinside}. Можно заметить необычное Г-образное положение микропереключателей на печатной плате, вызванное необходимостью как-то разместить четыре достаточно габаритных элемента без изменения формы корпуса. В остальном, мышь не отличается от младших моделей линейки, демонстрируя классическую оптомеханическую систему, характерную для мышей первой половины 90-х годов и монтаж электронных компонентов на обратной стороне платы. 

Как и в случае других мышей Kensington 1995 года, в роли контрактного разработчика Thinking Mouse выступала Mitsumi Electric.

\begin{thebibliography}{9}
    \bibitem{kensingtonfamily} The first family in mice // PC Magazine, February 10, 1998 -- P. 270 \url{https://books.google.by/books?id=fFrjSBw0w14C&lpg=PA270&dq=kensington%20mouse%20in%20a%20box&hl=ru&pg=PA270#v=onepage&q&f=false}
    \bibitem {thinkingmouse} Kensington: Thinking Mouse -- kensington.com. January 06, 1997 \url{https://web.archive.org/web/19970106170908/http://www.kensington.com/prod/mice/mice3c.html}
    \bibitem {box} Kensington Thinking Mouse package \url{https://github.com/fiowro/mouses/blob/main/source/OCR/thinking_mouse_box.pdf}
    \bibitem{italian} Truscelli M. Kensington Thinking Mouse // MCmicrocomputer, No. 152, giugno 1995. -- P. 214--215 \url{http://www.digitanto.it/mc-online/PDF/Articoli/152_214_215_0.pdf}
\end{thebibliography}

\end{document}
