\documentclass[11pt, a4paper]{article}
\input{preamble.tex}
\switchlang{ru}
\begin{document}

\title{1995 "--- Kensington Thinking Mouse}
\date{}
\maketitle
\selectlanguage{russian}

Мышь Thinking Mouse является завершающей частью той же линейки мышей, что и Kensington mouse. Она также предназначена для работы с IBM-совместимыми компьютерами, и следует тому же дизайну, с незначительными модификациями.

    \begin{figure}[h]
        \centering
    \includegraphics[scale=0.4]{1995_kensington_thinking_mouse/pic_30.jpg}
        \caption{Внешний вид мыши Thinking Mouse}
        \label{thinkingmousepic}
    \end{figure}

   	Зона кнопок на верхней части корпуса занимает ту же площадь, что у и Kensington mouse, но их количество увеличено до четырех. Кнопки также визуально отделены друг от друга и от остальной части корпуса визуально различимым  зазором, под которым также присутствует логотип компании (рис. \ref{thinkingmousetopbottom}). 

    \begin{figure}[h]
        \centering
    \includegraphics[scale=0.6]{1995_kensington_thinking_mouse/top_30.jpg}
    \includegraphics[scale=0.6]{1995_kensington_thinking_mouse/bottom_30.jpg}
        \caption{Изображение Thinking Mouse Mouse вид сверху и снизу}
        \label{thinkingmousetopbottom}
    \end{figure}
    
    Нижняя часть  мыши целиком выполнена из серого пластика (рис. \ref{kensingtonmousepic}), в тон логотипу и кабелю. На ней также присутствует ярлык с техническими данными мыши, содержащий код FCC ID. Изучение кода по базе данных Федеральной комиссии по связи США позволяет датировать мышь Thinking mouse 1995 годом.
    Кроме того, на нижней части корпуса в тех же местах что и у двухнопочного варианта располагаются 5 круглых накладок из низкофрикционного материала и такое же поворотное кольцо-защелка для извлечения шара. 

    
    В отличие от младших моделей семейства, нижняя часть корпуса мыши Thinking Mouse изготовлена из <<прорезиненного>> материала Kensington EasyGrip \cite{thinkingmouse} который во многом похож на современный пластик с покрытием софт-тач и минимизирует вероятность соскальзывания пальцев с боков мыши. 
    
     \begin{figure}[h]
        \centering
    \includegraphics[scale=0.3]{1995_kensington_thinking_mouse/size_15.jpg}
        \caption{Изображение Thinking Mouse на размерном коврике с шагом сетки 1 см}
        \label{thinkingmousesize}
    \end{figure}
    
    Как и Kensington Mouse, Thinking Mouse комплектовалась программным обеспечением Kensington Mouse Works. Программа позволяла выполнять тонкую настройку (вплоть до задания нелинейного ускорения мыши с помощью кривой или табличных данных), а также назначить различные функции кнопкам. По умолчанию две дальние от пользователя кнопки выполняют функции одинарного клика главной кнопкой и клика с фиксацией (для облегчения перетаскивания объектов), а ближние к пользователю "--- двойной щелчок и обычный щелчок правой кнопкой, обычно отображающий контекстное меню \cite{italian}.
    
    \begin{figure}[h]
        \centering
    \includegraphics[scale=0.3]{1995_kensington_thinking_mouse/hand_30.jpg}
        \caption{Изображение Thinking Mouse с муляжом руки на размерном коврике
с шагом сетки 1 см}
        \label{thinkingmousehand}
    \end{figure}

    Необходимо заметить, что добавление двух кнопок оказалось спорным решением. Помимо уменьшения площади кнопок для нажатия пальцами (крупные кнопки были одним из важных достоинств Kensington Mouse наряду с ее эргономичной формой), выбор функций по умолчанию мог бы быть востребован для облегчения работы людьми с ограниченными возможностями, но был избыточен для большинства пользователей.    
    
    \begin{figure}[h]
    \centering
    \includegraphics[scale=0.6]{1995_kensington_thinking_mouse/inside_30.jpg}
        \caption{Изображение Thinking Mouse в разобранном виде}
        \label{thinkingmouseinside}
    \end{figure}
    
    Мышь в разобранном виде показана на рисунке \ref{thinkingmouseinside}. Компоненты соответствуют Kensington Mouse, за исключением Т-образного положения микропереключателей на печатной плате, вызванной необходимостью как-то разместить эти 4 достаточно габаритных элемента без изменения формы корпуса.

\begin{thebibliography}{9}

\end{thebibliography}

\end{document}
