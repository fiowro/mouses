\documentclass[11pt, a4paper]{article}
\input{preamble.tex}

\begin{document}

\title{1993 "--- Colani Mouse}
\date{}
\maketitle

Apparently, the Colani Mouse was, along with the trackball of the same name, the first among the so-called “designer” mice - devices officially developed in collaboration with celebrities in the field of technical design. This mouse was named and shaped by Luigi Colani \cite{wiki}.

\textit{Colani is best known in the automotive industry, with about 40 concept cars; no less actively he designed furniture, household items, household appliances. And Luigi Colani's fruitful collaboration with the German company Vobis Microcomputer, the owner of the Highscreen trademark, led to the release of several personal computers, joysticks, mice and trackballs based on his design.}

\begin{figure}[h]
    \centering
    \includegraphics[scale=0.6]{1993_colani_mouse/pic_60.jpg}
    \caption{Colani Mouse}
    \label{fig:ColaniMousePic}
\end{figure}

The HIGHSCREEN Colani Mouse shown in figure \ref{fig:ColaniMousePic} is representative of this line. This mouse model (together with a similarly designed Colani Trackball from the same manufacturer) won the prestigious “IF Design Award” from iF International Forum Design GmbH in 1993 \cite{award}.

\begin{figure}[h]
    \centering
    \includegraphics[scale=0.5]{1993_colani_mouse/top_60.jpg}
    \includegraphics[scale=0.5]{1993_colani_mouse/bottom_30.jpg}
    \caption{Colani Mouse, top and bottom views}
    \label{fig:ColaniMouseTopBottom}
\end{figure}

Starting with concept cars that emphasized the aerodynamics of high speeds, Luigi Colani developed his own new style of industrial design, in the direction now known as bionics. The main principle of bionics is to give objects roundness and streamlined natural forms (figure \ref{fig:ColaniMouseTopBottom}).

\textit{According to the researchers of Luigi Colani's work, in general, the designer recognized the primacy of the object's function over its form. However, time after time, he could not resist the opportunity to make the product more “natural”, to provide it with curves, sometimes turning into each other in the most unexpected way, which suggests thoughts of rocks altered by the long work of wind or waves. In conservative industries, such as the automotive industry, this approach was perceived as too eccentric.}

However, with regard to cursor controls, the designer turned out to be the forerunner of the era of ergonomic mice that came after some time with a complex asymmetrical shape that provides the most comfortable position for the human hand. This trend was promoted both by the fact that such forms are easily reproduced when casting plastic bodies of manipulators (which cannot be said, for example, about car bodies), and the popularity of clay models with traces of a hand squeezing them, which allowed the developer to obtain an anatomically accurate and unusual outwardly product without special difficulties. However, judging by some signs, Luigi Colani himself did not resort to this method of shaping.

As you can see in figure \ref{fig:ColaniMouseSize}, the Colani Mouse is a very compact product that fits almost entirely in the palm of your hand.

\begin{figure}[h]
    \centering
    \includegraphics[scale=0.5]{1993_colani_mouse/size_30.jpg}
    \caption{Colani Mouse on a graduated pad with a grid step of 1~cm}
    \label{fig:ColaniMouseSize}
\end{figure}

In addition to the fact that the concave part of the body has an aesthetic and artistic function, it serves as a support for the thumb (figure \ref{fig:ColaniMouseHand}). However, figure \ref{fig:ColaniMouseSize} clearly shows that this design element does not reach the end of the body, resulting in a ledge, which reduces the comfort of holding the mouse. The ergonomic oddity of this ledge is increased by the fact that it is proudly signed with the designer's name. In addition, the body does not provide a wrist support to the user. However, in general, the position of the hand on it is comfortable, the keys are easy to reach and have sufficient area.

\begin{figure}[h]
    \centering
    \includegraphics[scale=0.45]{1993_colani_mouse/hand_30.jpg}
    \caption{Colani Mouse with a human hand model}
    \label{fig:ColaniMouseHand}
\end{figure}

Trackball internals are shown on figure \ref{fig:ColaniMouseInside}. This is an opto-mechanical device with toothed disks of encoders, typical for the late 80s and early 90s. Note that the choice of details is extremely budgetary: trackball contains a bare minimum of movable parts and a bare minimum of metal elements. The insulating tape, prominent in the photo, hides the place of fusion of the cable: these are the obvious consequences of the trend to damage the insulation of the cable at the place of its exit from the body - it is not the rarest situation when the mouse has no flexible protecting sleeve in this place.

\begin{figure}[h]
    \centering
    \includegraphics[scale=0.9]{1993_colani_mouse/inside_60.jpg}
    \caption{Colani Mouse disassembled}
    \label{fig:ColaniMouseInside}
\end{figure}

\begin{thebibliography}{9}
    \bibitem {wiki} Luigi Colani – Wikipedia \url{https://en.wikipedia.org/wiki/Luigi_Colani}
    \bibitem {award} iF – HIGHSCREEN Colani Mouse \url{https://ifdesign.com/en/winner-ranking/project/highscreen-colani-mouse/19312}
\end{thebibliography}

\end{document}
