\documentclass[11pt, a4paper]{article}
\input{preamble.tex}
\switchlang{ru}
\begin{document}

\title{1995 "--- Трекбол Logitech TrackMan Marble}
\date{}
\maketitle
\selectlanguage{russian}
Трекбол Trackman Marble (рис. \ref{fig:trackman}), выпущенный на рынок компанией Logitech в 1995 году, был первым трекболом, использовавшим полностью оптический принцип регистрации движения, без применения оптомеханического энкодера \cite{logitech25}.

\begin{figure}[h]
    \centering
    \includegraphics[scale=0.4]{1995_logitech_trackman/pic_60.jpg}
    \caption{Изображение Logitech TrackMan}
    \label{fig:trackman}
\end{figure}

Данный трекбол имеет 3 клавиши, отвечающие за стандартные функции кнопок мыши, и шар, предназначенный для вращения большим пальцем правой руки (рис. \ref{fig:trackmanTopAndBottom}). Регулярный узор из тёмных точек на поверхности шара вызван применением оптического датчика для считывания перемещений. Колесо прокрутки отсутствует на данной модели, однако драйвер позволял использовать для прокрутки вращение шара с зажатой средней кнопкой (при нажатии эта кнопка выполняет привычную функцию). Следует отметить, что существовала также модификация этого трекбола с колесом прокрутки в вырезе третьей кнопки. Трекбол подключается к компьютеру по интерфейсу PS/2.

\begin{figure}[h]
    \centering
    \includegraphics[scale=0.4]{1995_logitech_trackman/top_60.jpg}
    \includegraphics[scale=0.4]{1995_logitech_trackman/bottom_60.jpg}
    \caption{Изображение Logitech TrackMan, вид сверху и снизу}
    \label{fig:trackmanTopAndBottom}
\end{figure}

\begin{figure}[h]
    \centering
    \includegraphics[scale=0.4]{1995_logitech_trackman/size_30.jpg}
    \caption{Изображение Logitech TrackMan на размерном коврике с шагом сетки 1 см}
    \label{fig:trackmanSize}
\end{figure}

Трекбол имеет средние размеры (рис. \ref{fig:trackmanSize}). Большая часть корпуса трекбола выполнена с наклоном вправо, благодаря чему запястье руки пользователя оказывается в более естественном положении. В то время, как шар прокручивается большим пальцем, остальные пальцы работают так же, как при пользовании обычной мышью, что делает конструкцию более привлекательной для пользователя, привыкшего к мыши или попеременно работающего мышью и трекболом (рис. \ref{fig:trackmanHand}). Однако у такой компоновки есть и недостатки: подвижность большого пальца несколько меньше, что теоретически может отражаться на быстроте и точности позиционирования. К тому же благодаря такой форме устройство совершенно непригодно для левшей.

\begin{figure}[h]
    \centering
    \includegraphics[scale=0.6]{1995_logitech_trackman/hand_30.jpg}
    \caption{Изображение Logitech TrackMan с моделью руки человека}
    \label{fig:trackmanHand}
\end{figure}

Разбор трекбола (рис. \ref{fig:trackmanInside}) наглядно показывает причину того, что Logitech изменила расцветку шара. По факту, вместо традиционной схемы оптомеханической мыши, данное устройство оказалось первым, построенным как аналог оптической мыши, считывающей изменения яркости с помощью специального коврика с нанесенной на нём сеткой (в данном случае роль коврика играет рисунок на вращающемся шаре). По заверениям разработчика, распознавание движения реализовано системой на основе искусственной нейронной сети \cite{marbleAdv}.

\begin{figure}[h]
    \centering
    \includegraphics[scale=0.6]{1995_logitech_trackman/inside_30.jpg}
    \caption{Изображение Logitech TrackMan изнутри}
    \label{fig:trackmanInside}
\end{figure}

Маркировка на нижней части трекбола содержит код FCC ID (рис. \ref{fig:trackmanTopAndBottom}).
Проверка кода по базе данных Федеральной комиссии по коммуникациям США показывает, что трекбол был разработан компанией Logitech в 1995 году.

%\begin{figure}[h]
%    \centering
%    \includegraphics[scale=0.5]{1995_logitech_trackman/2.17.JPG}
%    \caption{Изображение Logitech TrackMan вид снизу}
%    \label{fig:trackmanBottom}
%\end{figure}

\begin{thebibliography}{9}
\bibitem{marbleAdv} Melissa J. Perenson. New \& improved. News of announced products and upgrades. // PC Magazine, Vol. 14, No. 22. -- December 19, 1995. -- p. 61 -- 66.
\bibitem{logitech25} 25-year category criteria. Logitech’s 25 Most Important Products \url{https://web.archive.org/web/20201029120308/https://www.logitech.com/lang/pdf/logitech_most_important_products.pdf}
\end{thebibliography}

\end{document}
