\documentclass[11pt, a4paper]{article}
\input{preamble.tex}

\begin{document}

\title{1987 "--- MicroSpeed FastTRAP trackball}
\date{}
\maketitle

Устройство FastTRAP от MicroSpeed, выпущенное в 1987 году, представляет собой трекбол с тремя кнопками, а также дополнительным колесом, благодаря которому он может выдавать координаты не по двум, а по трем координатным осям  — \textit{x}, \textit{y} и \textit{z} — то есть поддерживает на одну координатную плоскость больше, чем мышь или планшет (рис. \ref{fig:FastTRAPPic}). Дополнительная ось была рассчитана в первую очередь на пользователей САПР твердотельного моделирования, поскольку именно в них одновременное изменение сразу в трёх координатных осях может сократить время, необходимое для вращения объекта или проекции в окне просмотра.

\begin{figure}[h]
   \centering
    \includegraphics[scale=0.3]{1987_microspeed_fasttrap/pic_15.jpg}
    \caption{MicroSpeed FastTRAP}
    \label{fig:FastTRAPPic}
\end{figure}

Управление по оси \textit{z} осуществляется вращением колеса; однако в момент выпуска этого устройства концепции колеса прокрутки еще не существовало, поэтому программное обеспечение не позволяет использовать его для скроллинга в графических или текстовых программах.

\begin{figure}[h]
    \centering
    \includegraphics[scale=0.3]{1987_microspeed_fasttrap/top_60.jpg}
    \includegraphics[scale=0.3]{1987_microspeed_fasttrap/bottom_60.jpg}
    \caption{Изображение FastTRAP, вид сверху и снизу}
    \label{fig:FastTRAPTop}
\end{figure}

При подключении к компьютеру FastTRAP использует протокол мыши Microsoft, и может использовать соответствующий стандартный драйвер. Идущий в комплекте специализированный драйвер требуется только для работы с третьей координатой. Также идущая в комплекте программа настройки позволяет запрограммировать функции клавиш, шара и колеса, что позволяет адаптировать устройство под любое программное обеспечение: например выполнять определенные команды DOS \cite{fast}.

\begin{figure}[h]
   \centering
    \includegraphics[scale=0.3]{1987_microspeed_fasttrap/size_15.jpg}
    \caption{Изображение FastTRAP на размерном коврике}
    \label{fig:FastTRAPSize}
\end{figure}

Трекбол является очень крупным (рис. \ref{fig:FastTRAPSize}, \ref{fig:FastTRAPHand}). Помимо третьей координаты, FastTRAP предоставляет пользователям САПР дополнительное преимущество за счет большого диаметра шара: точное размещение курсора в графическом окне оказывается точнее проще, чем при перемещении мыши в нужное положение (иногда не удается избежать случайного перемещения мыши при нажатии кнопки).

\begin{figure}[h]
    \centering
    \includegraphics[scale=0.25]{1987_microspeed_fasttrap/hand_15.jpg}
    \caption{Изображение FastTRAP на размерном коврике с моделью руки человека}
    \label{fig:FastTRAPHand}
\end{figure}

В плане эргономики кнопки трекбола расположены достаточно удобно с точки зрения их досягаемости "--- при условии, что пользователь накрывает шар ладонью. Однако колесо находится на дальнем от от пользователя краю корпуса, и для его вращения может понадобиться перемещать руку (рис. {fig:FastTRAPTop}, \ref{fig:FastTRAPHand}).


\begin{figure}[h]
    \centering
    \includegraphics[scale=0.5]{1987_microspeed_fasttrap/inside_15.jpg}
    \caption{Изображение FastTRAP в разобранном виде}
    \label{fig:FastTRAPInside}
\end{figure}

Внутреннее устройство трекбола можно видеть на рис. \ref{fig:FastTRAPInside}.
Стандартная оптомеханическая конструкция дополнена фрикционной передачей, связывающей ролик с дополнительным оптомеханическим энкодером. Также можно заметить идентичность энкодеров, выполняющих снятие перемещения по всем трём осям: с одной стороны, это оказалось не самой сложной задачей в плане компоновки (учитывая большие размеры корпуса), а с другой, это может быть актуальным, учитывая назначение дополнительного колеса.

\begin{thebibliography}{9}
\bibitem {fast} G. Kunkel. The 3-D FastTRAP points with precision // PC Magazine, November 24, 19987. - p. 56.
\end{thebibliography}
\end{document}
