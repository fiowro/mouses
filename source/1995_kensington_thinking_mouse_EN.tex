\documentclass[11pt, a4paper]{article}
\input{preamble.tex}
\switchlang{en}
\begin{document}

\title{1995 -- Kensington Thinking Mouse}
\date{}
\maketitle
\selectlanguage{english}

Thinking Mouse was released in 1995 by the American company Kensington Computer Products Group as part of a line of several similarly shaped mice \cite{kensingtonfamily}. The Kensington Thinking Mouse, the top model in the line, was available for PCs (fig. \ref{thinkingmousepic}) and for Apple computers with an ADB interface \cite{thinkingmouse}, like the company's other mice and trackballs.

    \begin{figure}[h]
        \centering
    \includegraphics[scale=0.7]{1995_kensington_thinking_mouse/pic_30.jpg}
        \caption{Kensington Thinking Mouse}
        \label{thinkingmousepic}
    \end{figure}

Overall, the Thinking Mouse is designed in the same style as other Kensington mice, with some minor modifications.

The concept of minimalism and the use of geometric abstraction are clearly visible in the design of the mouse. The mouse body is white on top and gray on the bottom, and has an ovoid base (it is symmetrical about the longitudinal axis and slightly tapered toward the front, i.e., the side furthest from the user). Four buttons are located on the front of the body: large left and right buttons are located at the front edge, and a pair of additional buttons, which together form an ellipse, are located just behind them, closer to the user (intentional following the principles of mathematical design is also emphasised by Thinking mouse packaging, where one can see a drawing of the case with auxiliary construction lines showing the circles and arcs used to make its shape \cite{box}). The buttons occupy a third of the mouse's length, almost completely integrated into the shape of the mouse body (except that the junction of the main button and the additional button forms a small ledge) and are separated by a visually distinguishable gap. Apart from the buttons, the mouse top contains no other elements except for the company logo (fig. \ref{thinkingmousetopbottom}). The bottom of the mouse is made of gray plastic (fig. \ref{thinkingmousesize}), in the same color as the logo, the cable, and the plain washer that protects the cable from damage where it exits the body. The bottom of the mouse has the same layout as other members of the Kensington Mouse line from 1995 (it contains a label with technical specifications, five low-friction round ``feet'' that ensure a smooth glide across the desk surface, and a rotating latch ring for removing the ball and cleaning the mouse). However, unlike less advanced models, the bottom of the Thinking Mouse is made of gray ``Kensington's rubberized EasyGrip material'' \cite{thinkingmouse}, which is similar in many ways to modern soft-touch plastic and, according to the manufacturer, minimizes the likelihood of fingers slipping off the sides of the mouse (fig. \ref{thinkingmousesize}).

    \begin{figure}[h]
        \centering
    \includegraphics[scale=0.78]{1995_kensington_thinking_mouse/top_30.jpg}
    \includegraphics[scale=0.78]{1995_kensington_thinking_mouse/bottom_30.jpg}
        \caption{Thinking Mouse, top and bottom views}
        \label{thinkingmousetopbottom}
    \end{figure}

A search of the FCC ID code in the Federal Communications Commission database dates the mouse to 1995.

     \begin{figure}[h]
        \centering
    \includegraphics[scale=0.48]{1995_kensington_thinking_mouse/size_15.jpg}
        \caption{Thinking Mouse on a graduated pad with a grid step of 1~cm}
        \label{thinkingmousesize}
    \end{figure}

Like other Kensington mice, the Thinking Mouse came with Kensington Mouse Works software. This software allowed for fine-tuning mouse behavior (even setting nonlinear mouse acceleration using a curve or table data) and assigning various functions to buttons (by default, the two additional buttons duplicate the main ones, producing the same left and right clicks). The Mouse Works developers have paid close attention to ergonomics. The user manual contains a section detailing the correct hand position on the mouse. Additionally, the assignable functions for the buttons include options such as a single click with locking (to facilitate dragging) and generating a double-click event when pressing the button once \cite{italian}, which could be intended primarily for people with limited mobility. Finally, even in normal use, the mouse retains good ergonomics, despite the fact that adding two more buttons is a rather questionable decision (because of them, the surface area of the main buttons is proportionally reduced, and, while the location of the additional buttons minimizes the probability of accidental clicks, it also makes them less comfortable to press from a normal hand position, as can be clearly seen in fig. \ref{thinkingmousehand}).

The internal structure of the mouse is shown in fig. \ref{thinkingmouseinside}.

    \begin{figure}[h!]
        \centering
    \includegraphics[scale=0.5]{1995_kensington_thinking_mouse/hand_30.jpg}
        \caption{Thinking Mouse with a human hand model}
        \label{thinkingmousehand}
    \end{figure}

    \begin{figure}[h!]
    \centering
    \includegraphics[width=\textwidth]{1995_kensington_thinking_mouse/inside_30.jpg}
        \caption{Thinking Mouse disassembled}
        \label{thinkingmouseinside}
    \end{figure}


The unusual L-shaped arrangement of the microswitches on the printed circuit board is due to the need to accommodate four fairly large components without changing the shape of the case. Otherwise, the mouse is no different from the entry-level models in the line, featuring the classic optomechanical system typical of mice from the first half of the 1990s, and mounting the electronic components on the back of the board.

As with other Kensington mice of 1995, the Thinking Mouse production was contracted to Mitsumi Electric.

\begin{thebibliography}{9}
    \bibitem{kensingtonfamily} The first family in mice // PC Magazine, February 10, 1998 -- P. 270 \url{https://books.google.by/books?id=fFrjSBw0w14C&lpg=PA270&dq=kensington%20mouse%20in%20a%20box&hl=ru&pg=PA270#v=onepage&q&f=false}
    \bibitem {thinkingmouse} Kensington: Thinking Mouse -- kensington.com. January 06, 1997 \url{https://web.archive.org/web/19970106170908/http://www.kensington.com/prod/mice/mice3c.html}
    \bibitem {box} Kensington Thinking Mouse package \url{https://github.com/fiowro/mouses/blob/main/source/OCR/thinking_mouse_box.pdf}
    \bibitem{italian} Truscelli M. Kensington Thinking Mouse // MCmicrocomputer, No. 152, giugno 1995. -- P. 214--215 \url{http://www.digitanto.it/mc-online/PDF/Articoli/152_214_215_0.pdf}
\end{thebibliography}

\end{document}
