\documentclass[11pt, a4paper]{article}
\input{preamble.tex}
\switchlang{ru}
\begin{document}

\title{1983 "--- DEC VS10X-EA Mouse}
\date{}
\maketitle
\selectlanguage{russian}
Мышь DEC VS10X-EA (рис. \ref{fig:DecVS10XPic}) была выпущена в 1983 году на базе разработки компании Hawley Mouse House \cite{hawley,mouses}, созданной Джеком Хоули, соразработчиком мыши для компьютеров Xerox Alto и одним из авторов патента Xerox 1973 года на мышь с двумя наклонными колесами \cite{pat}. Фактически, DEC VS10X-EA является модификацией мыши Hawley Mark II X063X Mouse того же года выпуска. Сравнение двух мышей не выявляет конструктивных различий; все расхождения касаются формы корпуса и разъёма. Данной мышью комплектовались графические терминалы DEC VAXstation 100 \cite{reddit}, использовавшиеся при создании графической оконной системы Unix-подобных ОС, X Windows System.

\begin{figure}[h]
   \centering
    \includegraphics[scale=0.6]{1983_dec_vs10x_ea_mouse/pic_30.jpg}
    \caption{DEC VS10X-EA Mouse, вид спереди}
    \label{fig:DecVS10XPic}
\end{figure}

Если корпус мыши-прототипа, Hawley Mark II, представляет собой паралеллепипед, с тремя прямоугольными кнопками, то корпус DEC VS10X-EA имеет более сглаженную форму с большим наклоном стенок корпуса и плавными переходами между гранями. Достигается это за счет увеличенного корпуса, в который <<вписан>> легко узнаваемый прямоугольник мыши Hawley (рис. \ref{fig:DecVS10XTopAndBottom}). Очевидно, такой дизайн не мог не сказаться положительно на эргономике манипулятора.

\begin{figure}[h]
    \centering
    \includegraphics[scale=0.5]{1983_dec_vs10x_ea_mouse/top_60.jpg}
    \includegraphics[scale=0.5]{1983_dec_vs10x_ea_mouse/bottom_30.jpg}
    \caption{DEC VS10X-EA Mouse, вид сверху и снизу}
    \label{fig:DecVS10XTopAndBottom}
\end{figure}

Нижняя сторона целиком выполнена из металла (рис. \ref{fig:DecVS10XTopAndBottom}). Вращение регистрируется гладким стальным шаром в центре, а еще два шарика меньшего размера играют роль ножек для минимизации трения. Мышь не комплектовалась ковриком: руководство пользователя VAXstation 100 предлагает использовать в этом качестве обычный лист бумаги \cite{manual}.
 Съемное кольцо, позволяющее извлечь шар для удаления собравшегося мусора, в данной модели еще не предусмотрено, поэтому для чистки необходима полная разборка.

\begin{figure}[h]
    \centering
    \includegraphics[scale=0.7]{1983_dec_vs10x_ea_mouse/size_30.jpg}
    \caption{DEC VS10X-EA на размерном коврике с шагом сетки 1~см}
    \label{fig:DecVS10XSize}
\end{figure}

Мышь имеет небольшие размеры, характерные для мышей 1980-х годов (рис. \ref{fig:DecVS10XSize}); рука опирается на корпус лишь в незначительной степени (рис. \ref{fig:DecVS10XHand}).

\begin{figure}[h]
    \centering
    \includegraphics[scale=0.7]{1983_dec_vs10x_ea_mouse/hand_30.jpg}
    \caption{DEC VS10X-EA с моделью руки человека}
    \label{fig:DecVS10XHand}
\end{figure}

Внутреннее устройство мыши показано на рис. \ref{fig:DecVS10XInside}. Можно отметить съемную глухую защиту шара, требующую дополнительных операций разборки для удаления мусора. В мыши использованы контактные энкодеры (с четырьмя контактами для большей надежности), включающие в себя металлический контактный барабан, вместо более распространенного в последующих моделях диска.

 \begin{figure}[h]
    \centering
    \includegraphics[scale=1]{1983_dec_vs10x_ea_mouse/inside_30.jpg}
    \caption{DEC VS10X-EA в разобранном виде}
    \label{fig:DecVS10XInside}
\end{figure}

\begin{thebibliography}{9}
\bibitem{hawley} Hawley Mouse House \url{https://web.archive.org/web/20211020150835/https://oldmouse.com/mouse/hawley/}

\bibitem{mouses} Hawley Mark II X063X Mouses \url{https://web.archive.org/web/20211020000256/https://www.oldmouse.com/mouse/hawley/X063X.shtml}

\bibitem{pat} Transducer for a display-oriented pointing device \url{https://patents.google.com/patent/US3892963A/en}

\bibitem{reddit} Another DEC mouse \url{https://www.reddit.com/r/vintagecomputing/comments/mm4des/another_dec_mouse/}

\bibitem{manual} VAXstation 100 User Guide \url{http://www.bitsavers.org/pdf/dec/vax/vaxstation100/AA-N660A-TE_VAXstation_100_Users_Guide_Jun84.pdf}
\end{thebibliography}
\end{document}
