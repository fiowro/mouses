\documentclass[11pt, a4paper]{article}
\input{preamble.tex}
\switchlang{ru}
\begin{document}

\title{1985 "--- Logitech С7 mouse}
\date{}
\maketitle
\selectlanguage{russian}
Мышь Logitech C7 была выпущена в 1985 году. Она стала первой мышью, выпущенной под брэндом Logitech \cite{timeline} и одновременно пробным шаром Logitech на ниве выпуска мышей для розничной продажи (до этого компания производила мышей для поставок ОЕМ). Отпускная цена C7 составляла всего \$100, что было заметно дешевле других оптомеханических мышей того времени. По-видимому именно это, наряду с удачной конструкцией, сделало мышь очень популярной среди конечных пользователей, несмотря на её заметную <<квадратность>> (рис. \ref{fig:LogitechC7Pic}).

\begin{figure}[h]
   \centering
    \includegraphics[scale=0.45]{1985_logitech_c7_mouse/pic_60.jpg}
    \caption{Logitech C7 mouse, вид спереди}
    \label{fig:LogitechC7Pic}
\end{figure}

В форме корпуса мыши прослеживается ощутимое сходство с более ранней моделью, LOGIMOUSE P5, созданной при участии известного швейцарского дизайнера Антуана Каэна. В С7 отказались от стогой призматической формы корпуса, предоставив пользователю горизональную площадку для опоры ладонью, а также слегка скруглённые ребра и углы. Узкие диагональные кнопки P5 уступили место значительно более удобным прмоугольным клавишам с большой площадью. Нижняя сторона сохранила еще больше сходства с P5: на ней также присутствуют четыре белые опоры с низким коэффициентом трения (на этот раз наклеенные на стенку корпуса, что вскоре станет повсеместно распространенным явлением) и съемное поворотное кольцо на защелках, позволяющее извлечь шар для чистки мыши (рис. \ref{fig:LogitechC7TopAndBottom}).

\begin{figure}[h]
    \centering
    \includegraphics[scale=0.4]{1985_logitech_c7_mouse/top_30.jpg}
    \includegraphics[scale=0.4]{1985_logitech_c7_mouse/bottom_30.jpg}
    \caption{Logitech C7 mouse, вид сверху и снизу}
    \label{fig:LogitechC7TopAndBottom}
\end{figure}

\begin{figure}[h]
    \centering
    \includegraphics[scale=0.35]{1985_logitech_c7_mouse/size_30.jpg}
    \caption{Logitech C7 mouse на размерном коврике с шагом сетки 1~см}
    \label{fig:LogitechC7Size}
\end{figure}

Мышь имеет размеры, типичные для мышей 1980-х годов (рис. \ref{fig:LogitechC7Size}), не слишком отличающиеся от предыдущей модели, P5. Что касается эргономики, то очевидно, отказ от оригинального дизайна Антуана Каэна в пользу более прагматичного решения пошел ей на пользу. Верхняя часть корпуса предусматривает опору для ладони, а кнопки достаточно удобно нажимать палцами (рис. \ref{fig:LogitechC7Hand}). В начале 90-х угловатость C7 будет смотреться 
проигрышно в сравнении с мышами более обтекаемых форм, однако для середины 80-х годов мышь имеет неплохой уровень эргономики.

\begin{figure}[h]
    \centering
    \includegraphics[scale=0.35]{1985_logitech_c7_mouse/hand_30.jpg}
    \caption{Logitech C7 mouse с моделью руки человека}
    \label{fig:LogitechC7Hand}
\end{figure}

Внутреннее устройство мыши показано на рис. \ref{fig:LogitechC7Inside}. Как и LOGIMOUSE P5, данная мышь является оптомеханической. Компоновка C7 также во многом повторяет P5, включая оптопары с оптическим прерывателем и весь механический узел. Среди отличий "---  отнести усложнившееся схемное решение (из-за использования последовательного интерфейса), ограничитель для защиты кабеля на выходе из корпуса, а также металлические пластины, которыми подпружиненны кнопки. Также следует отметить, что С7, вероятно, одна из первых мышей с интерфейсом RS-232, в которой были применены низкопотребляющие светодиоды и потому не требовалось внешнее питание (в случае P5 такой вопрос не был актуальным, поскольку квадратурный интерфейс в любом случае позволял получить достаточно мощности).

\begin{figure}[h]
    \centering
    \includegraphics[scale=0.7]{1985_logitech_c7_mouse/inside_60.jpg}
    \caption{Logitech C7 mouse в разобранном виде}
    \label{fig:LogitechC7Inside}
\end{figure}

\begin{thebibliography}{9}
\bibitem {history} Logitech History. 2007 \url{https://web.archive.org/web/20081221120203/http://www.logitech.com/lang/pdf/logitech_history_200703.pdf}
\bibitem {timeline} Mouse timeline scroll by Logitech. Historic Firsts: The Mouse. Doug Engelbart Institute. \url{https://www.dougengelbart.org/content/view/162/#img1}
\bibitem {manual1} LOGIMOUSE C7 Technical Reference Manual. Firmware Revision 3.0. 1986. \url{https://bitsavers.org/pdf/logitech/mouse/Logitech_Logimouse_C7_Firmware_Rev_3.0_Jan86.pdf}
\bibitem {manual2} LOGITECH MOUSE User's Manual. Serial Mouse, Bus Mouse, Series 2 Mouse. 1987. \url{https://bitsavers.org/pdf/logitech/mouse/Logitech_Mouse_Users_Manual_Feb87.pdf}
\end{thebibliography}
\end{document}
