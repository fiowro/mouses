\documentclass[11pt, a4paper]{article}
\input{preamble.tex}
\switchlang{ru}
\begin{document}

\title{1984 "--- Mindset Joystick}
\date{}
\maketitle
\selectlanguage{russian}

Джойстик Mindset, показаннный на рисунке \ref{fig:MindsetJoystickPic}, был разработан как дополнительный аксессуар для работы с персональным компьютером Mindset. Компьютер Mindset был выпущен в 1984 году компанией Mindset Corporation, и находился в продаже всего год. В техническом плане он был частично совместим с IBM PC, оснащался процессором Intel 80186 и нестандартной графической подсистемой, обладавшей повышенными возможностями, включая аппаратное ускорение некоторых типовых графических операций \cite{wiki, byteMagazine}. И джойстик и мышь Mindset подключались к одному из двух боковых разъёмов на клавиатуре \cite{adv}. Реальным производителем джойстика, как и мыши, выступила японская компания ALPS.

\begin{figure}[h]
   \centering
    \includegraphics[scale=0.76]{1984_mindset_joystick/pic_30.jpg}
    \caption{Mindset Joystick}
    \label{fig:MindsetJoystickPic}
\end{figure}

Джойстик выполнен в молочно-белом корпусе с черными и красными элементами, аналогично другим частям компьютера Mindset. На  верхней стороне корпуса  присутствует название компьютера (рис. \ref{fig:MindsetJoystickPic}). Кнопки имеют круглую форму, они красного цвета и почти полностью утоплены в корпусе. Металлический стик джойстика имеет пластиковый наконечник, совпадающий по цвету с корпусом.

\begin{figure}[h]
    \centering
    \includegraphics[scale=0.81]{1984_mindset_joystick/top_30.jpg}
    \includegraphics[scale=0.81]{1984_mindset_joystick/bottom_30.jpg}
    \caption{Mindset Joystick, вид сверху и снизу}
    \label{fig:MindsetJoystickTopAndBottom}
\end{figure}

Джойстик представляет собой весьма миниатюрное устройство (рис. \ref{fig:MindsetJoystickSize}). Судя по размеру пластикового наконечника, стик предназначен для удерживания двумя пальцами (рис. \ref{fig:MindsetJoystickHand}).  Также можно заметить, что миниатюризация и дизайнерский минимализм привели к полному отсутствию элементов регулировки: в конструкции устройства не нашлось места для триммеров,  которые позволили исключить дрейф выставлением нулевого напряжения на выходах X и Y при вертикальном положении стика, а также в ней не предусмотрены регулировочные винты, которые позволили бы включать и отключать автоматический возврат стика в вертикальное положение.

\begin{figure}[h]
    \centering
    \includegraphics[scale=0.42]{1984_mindset_joystick/size_30.jpg}
    \caption{Mindset Joystick на размерном коврике с шагом сетки 1~см}
    \label{fig:MindsetJoystickSize}
\end{figure}

Кнопки дублируют друг друга и расположены по бокам корпуса, что является не самым удачным эргономическим решением при управлении одной рукой, как показано на рисунке \ref{fig:MindsetJoystickHand}, поскольку миниатюрный корпус будет сдвигаться при отклонении стика и нажатии кнопок. Очевидно, предполагалось, что пользователь будет держать джойстик в ладони одной руки, а управление стиком будет осуществлять пальцами другой (рис. \ref{fig:MindsetJoystickHand2}).

\begin{figure}[h!]
    \centering
    \includegraphics[scale=0.42]{1984_mindset_joystick/hand_30.jpg}
    \caption{Mindset Joystick с моделью руки человека}
    \label{fig:MindsetJoystickHand}
\end{figure}

\begin{figure}[h!]
    \centering
    \includegraphics[scale=0.51]{1984_mindset_joystick/hand2_30.jpg}
    \caption{Mindset Joystick, управление двумя руками}
    \label{fig:MindsetJoystickHand2}
\end{figure}

Внутреннее устройство джойстика показано на рис. \ref{fig:MindsetJoystickInside}. Как можно видеть, он представляет собой типичную конструкцию аналогового джойстика на основе двух перекрещивающихся коромысел, с поправкой миниатюризацию.

Как уже упоминалось, реальным производителем джойстика выступила японская компания ALPS "--- контрактный производитель мышей и джойстиков для ряда известных компаний, включая выпущенные в 1983 году первую японскую мышь, MZ-1X10 mouse, и первую мышь компании Microsoft, известную из-за зеленого цвета кнопок как <<зеленоглазая мышь>>.

 \begin{figure}[h!]
    \centering
    \includegraphics[width=\textwidth]{1984_mindset_joystick/inside_30.jpg}
    \caption{Mindset Joystick в разобранном виде}
    \label{fig:MindsetJoystickInside}
\end{figure}

\begin{thebibliography}{9}
\bibitem {wiki} Mindset (computer) - Wikipedia \url{https://en.wikipedia.org/wiki/Mindset_(computer)}
\bibitem {byteMagazine} Wadlow T. The Mindset Personal Computer // Byte Magazine, Vol. 10, No. 6. June, 1985. - P. 324-232 \url{https://archive.org/details/byte-magazine-1985-06/page/n331/mode/2up}
\bibitem{adv} Mindset Personal Computer System \url{https://archive.org/details/bitsavers_mindsetBrore_3744143/page/n1/mode/2up}
\end{thebibliography}
\end{document}
