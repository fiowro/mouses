\documentclass[11pt, a4paper]{article}
\input{preamble.tex}
\switchlang{en}
\begin{document}

\title{1997 -- Kensington Mouse, ADB 2-nd generation version}
\date{}
\maketitle
\selectlanguage{english}

The single-button Kensington Mouse, marketed and advertised as ``Mouse in a Box'' due to its cube-shaped packaging, was released by the American company Kensington Computer Products Group as part of a line of several similarly shaped mice \cite{kensingtonfamily, mouseinabox}. This particular model is a 2nd generation Kensington Mouse designed for Apple computers with an ADB interface (fig. \ref{mouseinaboxtopbottom}).

    \begin{figure}[h]
        \centering
    \includegraphics[scale=0.67]{1997_kensington_mouse_in_a_box_adb_2gen/pic_30.jpg}
        \caption{Kensington Mouse 2nd generation}
        \label{kensingtonmousepic}
    \end{figure}

The concept of minimalism and the active use of geometric abstraction are clearly visible in the design of the mouse. The mouse body is similar in shape to the first-generation Kensington Mouse -- it's made of white plastic and has an ovoid base (it is symmetrical about the longitudinal axis and slightly tapered toward the front, i.e., the side furthest from the user). There is a single button on the front part of the body (in case of the PC-compatible version this button is divided into two). The button is integrated into the shape of the body, takes up a third of the mouse's length, and is separated from the rest of the body by a visually distinguishable arch-shaped gap, behind which is the company logo (fig. \ref{mouseinaboxtopbottom}). The logo is dark gray, matching the color of the cable and the plain washer that protects the cable from damage where it exits the body. The bottom of the mouse shows a label with the technical specifications of the mouse, two arched low-friction pads (in contrast to the five round ``feet'' of the 1st generation Kensington mice), and a rotating latch ring for removing the ball and cleaning the mouse.

This unit does not have an FCC ID, but similar to other 2nd generation Kensington Mouse models, it can be dated back to 1997.

   \begin{figure}[h]
        \centering
    \includegraphics[scale=0.79]{1997_kensington_mouse_in_a_box_adb_2gen/top_30.jpg}
    \includegraphics[scale=0.79]{1997_kensington_mouse_in_a_box_adb_2gen/bottom_30.jpg}
        \caption{Kensington Mouse 2nd generation, top and bottom views} \label{mouseinaboxtopbottom}
    \end{figure}

The mouse is of medium size, the ball is located exactly in the center of the body and has a fairly small diameter (obviously, its size is limited by the height of the body, which is also quite small, as can be seen from fig. \ref{mouseinaboxsize}). As noted in the company's materials, the mouse suits everyone, ``left handed or right handed, short fingers or long''. Indeed, as with the two-button Kensington Mouse, the palm position on the mouse body feels very natural (fig. \ref{mouseinaboxhand}).

    \begin{figure}[h]
        \centering
    \includegraphics[scale=0.5]{1997_kensington_mouse_in_a_box_adb_2gen/size_15.jpg}
        \caption{Kensington Mouse 2nd generation on a graduated pad with a grid step of 1~cm}
        \label{mouseinaboxsize}
    \end{figure}

Inside the Kensington Mouse there is a classic optomechanical system typical of mice from the first half of the 1990s (fig. \ref{mouseinaboxinside}).

The circuit board differs slightly from its predecessor in the first-generation Kensington Mouse. However, as with the first generation, it features underside mounting of most components. A distinctive feature of the internal design of Kensington 2nd generation mice is that the roller mounts and the ball's ``dome'' are protruding elements of a cast part, which itself is actually the bottom of the mouse body. In the 1995 models, this purpose was served by an additional, complex-shaped plastic part, attached directly to the circuit board, which was undoubtedly a less cost-effective and less technologically practical solution.

As with Kensington mice of 1995, the Thinking Mouse production was contracted to Mitsumi Electric.

    \begin{figure}[h!]
        \centering
    \includegraphics[scale=0.51]{1997_kensington_mouse_in_a_box_adb_2gen/hand_30.jpg}
        \caption{Kensington Mouse 2nd generation with a human hand model}
        \label{mouseinaboxhand}
    \end{figure}

    \begin{figure}[h!]
        \centering
    \includegraphics[width=\textwidth]{1997_kensington_mouse_in_a_box_adb_2gen/inside_30.jpg}
        \caption{Kensington Mouse 2nd generation disassembled}
        \label{mouseinaboxinside}
    \end{figure}

\begin{thebibliography}{9}
    \bibitem{kensingtonfamily} The first family in mice // PC Magazine, February 10, 1998 -- P. 270 \url{https://books.google.by/books?id=fFrjSBw0w14C&lpg=PA270&dq=kensington%20mouse%20in%20a%20box&hl=ru&pg=PA270#v=onepage&q&f=false}
    \bibitem {mouseinabox} Kensington: Mouse in a Box -- kensington.com. January 06, 1997 \url{https://web.archive.org/web/19970106170908/http://www.kensington.com/prod/mice/mice3d.html} 
\end{thebibliography}

\end{document}
