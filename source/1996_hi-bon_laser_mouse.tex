\documentclass[11pt, a4paper]{article}
\input{preamble.tex}

\begin{document}

\title{1996 "--- Hi-Bon Optical laser mouse LMOX-2}
\date{}
\maketitle

    Hi-Bon Optical laser mouse LMOX-2 (рис. \ref{fig:OpticalLaserMousePic}) выпускалась в Южной Корее и была, наряду с мышью Q500, одной из двух необычных оптических мышей, разработанных в 1996 году iO TEK и использующих в своей конструкции световоды. Разрешающая способность данной мыши составляет 450 точек на дюйм, а надпись на коробке помимо этого факта упоминает в качестве революционной особенности отсутствие шара и, соответственно, отсутствие необходимости в чистке.

\begin{figure}[h]
    \centering
    \includegraphics[scale=0.4]{1996_hi-bon_laser_mouse/pic_30.jpg}
    \caption{Hi-Bon Optical laser mouse}
    \label{fig:OpticalLaserMousePic}
\end{figure}

Как в случае абсолютного большинства ранних оптических мышей, данному манипулятору требуется коврик с отражающей сеткой (рис. \ref{fig:OpticalLaserMousePad}). В отличие от металлических ковриков Mouse Systems, здесь применяется белая поверхность, отражающая инфракрасное излучение, с нанесенными на нее черными точками, имеющими меньший коэффициент отражения в инфракрасном диапазоне. Коврик имеет небольшой размер, который однако компенсируется высокой разрешающей способностью мыши.

\begin{figure}[h]
    \centering
    \includegraphics[scale=0.4]{1996_hi-bon_laser_mouse/pic2_30.jpg}
    \caption{Hi-Bon Optical laser mouse на комплектном коврике}
    \label{fig:OpticalLaserMousePad}
\end{figure}

Мышь имеет минималистичный дизайн и две кнопки, дна из которых выделена цветом (во второй половине 90-х это скорее дизайнерское решение, чем попытка наглядно показать пользователю главную кнопку мыши). С нижней стороны видны два светодиода и выходы 16 световодов (рис. \ref{fig:OpticalLaserMouseTopBottom}).

\begin{figure}[h]
    \centering
    \includegraphics[scale=0.4]{1996_hi-bon_laser_mouse/top_60.jpg}
    \includegraphics[scale=0.4]{1996_hi-bon_laser_mouse/bottom_60.jpg}
    \caption{Hi-Bon Optical laser mouse вид сверху и снизу}
    \label{fig:OpticalLaserMouseTopBottom}
\end{figure}

По размеру и форме мышь является типичным двухкнопочным манипулятором (рис. \ref{fig:OpticalLaserMouseSize}).

\begin{figure}[h]
    \centering
    \includegraphics[scale=0.3]{1996_hi-bon_laser_mouse/size.jpg}
    \caption{Hi-Bon Optical laser mouse на размерном коврике с шагом сетки 1~см}
    \label{fig:OpticalLaserMouseSize}
\end{figure}

Третья кнопка, достаточно удобно расположенная сбоку корпуса в зоне досягаемости большого пальца (рис. \ref{fig:OpticalLaserMouseHand}) используется для переключения режимов. Дополнительно эта боковая кнопка эмулирует двойное нажатие, однако учитывая ее малую площадь (в отличие от основных двух), данная функция не слишком удобна для частого использования.

\begin{figure}[h]
    \centering
    \includegraphics[scale=0.3]{1996_hi-bon_laser_mouse/hand_60.jpg}
    \caption{Изображение Hi-Bon Optical laser mouse с моделью руки человека}
    \label{fig:OpticalLaserMouseHand}
\end{figure}

Laser mouse содержит управляющий микроконтроллер, реализующий 3 режима работы.

Первый режим "--- стандартный. Второй (включается одновременным нажатием дополнительной кнопки сбоку корпуса и левой кнопки мыши) переключает мышь в режим джойстика. В этом режиме перемещение мыши от центра (точки, в которой мышь находилась в момент переключения режима) интерпретируется как отклонение рукоятки джойстика. Необходимо заметить, что применение мыши в качестве игрового джойстика является достаточно неудобным \cite{LittleMagick}, учитывая что центр, к которому необходимо вернуться чтобы остановить движение персонажа, никак не обозначен.

Третий режим (включается одновременным нажатием боковой и правой кнопок) активирует т.~н. «прецизионный» режим работы, предназначенный для художников и дизайнеров (фактически, активирует заявленное разрешение 450 DPI).

Внутреннее устройство (рис. \ref{fig:OpticalLaserMouseInside}) показывает два пучка световодов, расходящиеся по четырем фотоприемникам. Применение удвоенного числа фотоприемников позволило разработчикам отказаться от расчерчивания коврика продольными и поперечными полосами (в отличие от более дешевой модели Q500 и оптических мышей Mouse Systems), заменив их сеткой темных точек. Благодаря этому поворот коврика на $90^\circ$ не влияет на работоспособность мыши; однако углы некратные 90 затрудняют считывание движения, а угол $45^\circ$ делает управление курсором практически невозможным \cite{comparison}.

\begin{figure}[h]
    \centering
    \includegraphics[scale=0.8]{1996_hi-bon_laser_mouse/inside_60.jpg}
    \caption{Hi-Bon Optical laser mouse в разобранном виде}
    \label{fig:OpticalLaserMouseInside}
\end{figure}

Надпись на печатной плате показывает , что данная мышь была, как и мышь Q500, разработана на контрактной основе компанией iO TEK в 1996 году.

\begin{thebibliography}{9}
\bibitem{LittleMagick} This Serial ''Optical Laser Mouse'' from 1996 \url{https://www.youtube.com/watch?v=8CeKiSn5lGU}
\bibitem{comparison} LMOX2, The Other Weirdest Mouse \url{https://www.youtube.com/watch?v=2UXmDuiqMW0}

\end{thebibliography}
\end{document}
