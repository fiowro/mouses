\documentclass[11pt, a4paper]{article}

\input{preamble.tex}
%\twocolumn
\begin{document}

\title{Механические мыши}
\author{Общая информация}
\date{}
\maketitle

Для отслеживания движения механические мыши используют колеса или шарик, преобразуя их линейного движения по поверхности во вращательное движение коммутаторов или датчиков вращения ролика.

\begin{figure}[h]
    \centering
    \includegraphics[width=0.3\linewidth]{theory_mech/2.1.png}
    \caption{Механическая мышь с шариком и роликами}
    \label{fig:theoryBallOpt}
\end{figure}

\begin{figure}[h]
    \centering
    \includegraphics[width=0.3\linewidth]{theory_mech/2.2.png}
    \caption{Шар и ролик}
    \label{fig:theoryBallRoll}
\end{figure}

Мыши, которые используют шар для определения движения, могут быть представлены системой, показанной на рисунке \ref{fig:theoryBallRoll}. Скорость окружности шара $V_r$, равна скорости мыши, $V$. Так как ролик не прикреплен непосредственно к оси шара, а опирается на его окружность, при условии отсутствия проскальзывания скорость окружности ролика равна скорости окружности шара. Угловая скорость и вращение ролика теперь связаны с движением мыши с помощью приведенных выше уравнений, но радиус $R$ теперь намного меньше, и вал вращается намного быстрее.

$$\omega = V/R_1$$

\noindent где $V$ -- скорость мыши, а $R_1$ -- радиус ролика. Поскольку ролик меньше радиусом, он вращается быстрее при заданной скорости мыши.

Движение передается на датчики следующим образом. Ролики, которые прямо или косвенно вращаются колесом или шаром, подключены непосредственно к датчикам движения.

%\section{Оптомеханические мыши} \label{title:OptoMechanical}

Оптомеханические мыши для генерации квадратурных сигналов A и B используют устройство, называемое оптическим прерывателем. Как показано на рисунке \ref{fig:theoryQuadEncoder}, оптомеханическая система состоит из источника света (обычно светодиода), фотоприемника и оптического прерывателя, который соединен с вращающимся роликом мыши.

Прерыватель имеет серию чередующихся черных и белых полос, которые позволяют свету от светодиода попадать на детектор. Поскольку прерыватель вращается поперек линии светового луча, сплошные сегменты, расположенные между щелями, будут прерывать луч, и на выходе детектора появится серия импульсов напряжения. Второй квадратурный выход получается при использовании второго светодиода и детектора, которые смещены относительно первого светодиода и детектора на одну четверть угла радиальных прорезей, или при использовании прорезей, которые смещены на одну четверть их периода, аналогично смещенным проводящим сегментам коммутатора. Маска с двумя сквозными отверстиями может использоваться с коммутатором, чтобы световые лучи находились в квадратуре относительно вращения прерывателя. Маска может быть просверлена или выполнена методом литья так, чтобы отверстия различались по фазе точно на 90 градусов.

\begin{figure}[h]
    \centering
    \includegraphics[width=0.5\linewidth]{theory_mech/2.76.PNG}
    \caption{Оптический энкодер с квадратурными выходами}
    \label{fig:theoryQuadEncoder}
\end{figure}

\begin{figure}[h]
    \centering
    \includegraphics[width=0.5\linewidth]{theory_mech/2.77.PNG}
    \caption{Квадратурные сигналы}
    \label{fig:theoryQuadDiag}
\end{figure}

Выход оптического энкодера представляет собой два квадратурных сигнала, как показано на рисунке \ref{fig:theoryQuadDiag}. Направление можно определить, изучив соотношение фаз двух сигналов. Если сигнал A находится в состоянии высокого уровня, когда на сигнале B возникает восходящий фронт, то движение происходит в прямом направлении. Если микропроцессор достаточно быстр, сигналы могут быть подключены непосредственно к входному порту, а все декодирование и подсчет выполняются в программном обеспечении.

\end{document}
