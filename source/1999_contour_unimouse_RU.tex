\documentclass[11pt, a4paper]{article}
\input{preamble.tex}
\switchlang{ru}
\begin{document}

\title{1999 "--- Contour UniMouse}
\date{}
\maketitle
\selectlanguage{russian}
Мышь UniMouse была выпущена компанией Contour Design, специализирующейся исключительно на эргономических манипуляторах, для компьютеров Apple iMac, как замена для комплектной Apple USB mouse, из-за круглой формы известной как <<шайба>> (англ. <<puck>>). Многие критиковали Apple USB mouse за плохую эргономику, и через год после ее выхода на рынок последовал выпуск альтернативы от одного из лидеров в производстве эргономичных средств управления курсором \cite{pressrelease}.

\begin{figure}[h]
    \centering
    \includegraphics[scale=0.5]{1999_contour_unimouse/pic_30.jpg}
    \caption{Contour UniMouse}
    \label{fig:ContourUniMousePic}
\end{figure}

Мышь имеет полупрозрачный корпус, через который просвечивает печатная плата \ref{fig:ContourUniMouseTopAndBottom}, украшенный тонированными кнопками и принтом с названием мыши. Две крупные кнопки расположены в передней части корпуса. Между ними присутствует выпуклая третья кнопка, визуально стилизованная под колесо прокрутки. Нижняя часть мыши из прозрачного пластика выглядит вполне традиционно, включая ножки из низкофрикционного материала и съемное кольцо-защелку, позволяющее извлечь шар для чистки. По бокам корпуса размещены резиновые накладки для более комфортного захвата.

\begin{figure}[h]
    \centering
    \includegraphics[scale=0.45]{1999_contour_unimouse/top_30.jpg}
    \includegraphics[scale=0.45]{1999_contour_unimouse/bottom_30.jpg}
    \caption{Contour UniMouse, вид сверху и снизу}
    \label{fig:ContourUniMouseTopAndBottom}
\end{figure}

Мышь имела несколько вариантов цветового оформления кнопок, резиновых накладок и надписи для соответствия вариантам оформления корпуса iMac \cite{web}.

Размеры мыши (в отличие от Apple USB mouse) средние, типичные для мышей конца девяностых (рис. \ref{fig:ContourUniMouseSize}).

\begin{figure}[h]
    \centering
    \includegraphics[scale=0.5]{1999_contour_unimouse/size_30.jpg}
    \caption{Contour UniMouse на размерном коврике с шагом сетки 1~см}
    \label{fig:ContourUniMouseSize}
\end{figure}

Благодаря симметричности мышь одинаково подходит как для левшей, так и для правшей. Учитывая специализацию Contour Design, не приходится удивляться тому, что рука комфортно лежит на мыши (рис. \ref{fig:ContourUniMouseHand}).

\begin{figure}[h]
    \centering
    \includegraphics[scale=0.55]{1999_contour_unimouse/hand_30.jpg}
    \caption{Contour UniMouse с моделью руки человека}
    \label{fig:ContourUniMouseHand}
\end{figure}

В качестве незначительного недостатка \cite{mactoday} упоминает малую площадь опоры под запястье. Отсутствие колеса прокрутки компенсировалось программной реализацией: скроллинг в четырех направлениях осуществлялся перемещением мыши при нажатии на среднюю кнопку. Очевидно, что это решение трудно назвать идеальным: оно более трудоемко для пользователя, чем традиционный скроллинг, сама кнопка менее удобна, а кроме того требуется скачивать специально сконфигурированный драйвер и/или вспомогательное программное обеспечение с сайта производителя мыши. Однако в целом, для платформы Apple данная мышь была весьма качественным решением, компенсировавшим проблемную эргономику Apple USB mouse, для которой сторонними производителями выпускались даже специальные адаптеры-накладки, придававшие ей более продолговатую форму.

\begin{figure}[h]
    \centering
    \includegraphics[scale=0.7]{1999_contour_unimouse/inside_30.jpg}
    \caption{Contour UniMouse, в разобранном виде}
    \label{fig:ContourUniMouseInside}
\end{figure}

Внутреннее устройство данной мыши показано на рис. \ref{fig:ContourUniMouseInside}, что позволяет классифицировать его как традиционную оптомеханическую конструкцию. Форма диска оптического прерывателя с зубцами вместо прорезей типична скорее для мышей более раннего периода, однако в данной модели частота зубцов заметно увеличена для большей разрешающей способности и соответствует частоте прорезей в прерывателях других достаточно качественных мышей конца девяностых. Вместе с тем в конструкции максимально используется пластик, отсутствуют дорогие механические элементы, что позволило вписать мышь в бюджетный ценовой диапазон (в пределах \$40).

\begin{thebibliography}{9}
    \bibitem{pressrelease} Contour UniMouse Press Releases "--- \url{http://www.contourdesign.com/unimouse_press.htm} 
    \bibitem{web} Contour UniMouse Home Page "--- \url{http://contourdesign.com/unimouse.htm}
    \bibitem{mactoday} Hemmel J. Contour UniMouse. Avoiding Apple's mouse trap // mac today, May/June 1999 \url{https://web.archive.org/web/20000619173033/http://mactoday.com/mayjun99/unimouse.html}
\end{thebibliography}

\end{document}
