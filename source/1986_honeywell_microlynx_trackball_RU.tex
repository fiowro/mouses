\documentclass[11pt, a4paper]{article}
\input{preamble.tex}
\switchlang{ru}
\begin{document}

\title{1986 "--- Honeywell microLYNX trackball}
\date{}
\maketitle
\selectlanguage{russian}
Трекбол microLYNX (или <<$\mu LYNX$>> в случаях, когда написание допускает применение символов греческого алфавита), показанный на рис. \ref{fig:microLYNXPic}, выпускался в калифорнии компанией Honeywell "--- дочерним предприятием Disc Instruments. Появившись в 1986 году либо годом ранее (часть рекламных материалов датирована 1985 годом), трекбол оказался долгожителем и в дальнейшем выдержал множество переизданий под модельным номером <<LX200>>, отличавшихся интерфейсом подключения, блоком электроники и часто подвергавшихся ребрендингу \cite{lx200}.

%В плане технической реализации трекбол Honeywell оснащен механическими
%датчиками положения, то есть не использует наиболее часто встречающуюся в подобных устройствах оптомеханическую схему.

\begin{figure}[h]
    \centering
    \includegraphics[scale=0.45]{1986_honeywell_microlynx_trackball/pic_30.jpg}
    \caption{Внешний вид трекбола microLYNX}
    \label{fig:microLYNXPic}
\end{figure}

Как можно видеть, устройство выполнено в минималистичном дизайне, оснащено большим черным шаром, утопленном в бежевом корпусе с вытянутой подставкой под запястье \ref{fig:microLYNXTopBottom}. Шар плотно прилегает к краям отверстия в корпусе, что неплохо защищает от попадания мусора внутрь трекбола, но делает извлечение шара для чистки невозможным без разборки корпуса. Перед шаром расположены три кнопки, копирующие внешним видом клавиши классической полноразмерной клавиатуры.

\begin{figure}[h]
    \centering
    \includegraphics[scale=0.4]{1986_honeywell_microlynx_trackball/top_30.jpg}
    \includegraphics[scale=0.4]{1986_honeywell_microlynx_trackball/bottom_30.jpg}
    \caption{microLYNX, вид сверху и снизу}
    \label{fig:microLYNXTopBottom}
\end{figure}

По всей видимости, ранних вариантов <<LX200>> было два: модель comLYNX обладала последовательным интерфейсом \cite{comlynx}, вариант quadLYNX представлял собой наиболее простое в плане логики квадратурное устройство, подключаемое к шинному интерфейсу, а наиболее необычный из трех вариантов, microLYNX, использовал в качестве интерфейса подключения порт клавиатуры, фактически включаясь в разрыв клавиатурного кабеля. Позднее появились и варианты с интерфейсом PS/2.

По умолчанию microLYNX работает в <<текстовом режиме>>, при котором поворот шара производит тот же эффект, что и нажатие клавиш курсора на клавиатуре: фактически при движении шара трекбол генерирует и посылает в компьютер скан-коды курсорных клавиш. Кнопки трекбола программируются: пользователь может настроить кнопку на выдачу от 1 до 30 кодов клавиш в виде макросов. Также можно настроить скорость перемещения курсора (частоту генерирования кодов курсорных клавиш). 

Для графических программ под DOS драйвер microLYNX может эмулировать мышь Microsoft. В этом случае пользователь может использовать трекбол для перемещения курсора и перетаскивания при выделении текста или перемещении графического элемента (для перетаскивания могут использоваться две внешние кнопки, расположенные перед шаром).

\begin{figure}[h]
    \centering
    \includegraphics[scale=0.4]{1986_honeywell_microlynx_trackball/size.jpg}
    \caption{Трекбол microLYNX на размерном коврике с шагом сетки 1 см}
    \label{fig:microLYNXSize}
\end{figure}

Однако, учитывая размеры устройства (рис. \ref{fig:microLYNXSize}), удерживать нажатой кнопку, пока происходит
перемещение шара "--- это в лучшем случае сложный маневр. Honeywell решила эту проблему на трекболе comLYNX, в котором можно использовать среднюю кнопку в качестве <<защелки>> перетаскивания \cite{comlynx}. Сначала выполняется нажатие средней кнопки, затем левой либо правой, и это программно фиксирует выбранную кнопку в нажатом положении. Повторное нажатие любой из трех кнопок отключает данный режим. Однако в microLYNX эта функция не поддерживается.

Трекбол поставляется с двумя наборами программ: драйвером для DOS, эмулирующим мышь Microsoft, и отдельной резидентной программой для DOS, которая показывает всплывающее меню, а также позволяет запрограммировать клавиатурные макросы для кнопок трекбола и перемещения курсора.

Кроме того, настройка может выполняться без использования драйвера, в <<альтернативном режиме>>, включающемся при нажатии на клавиатуре компьютера сочетания <<Ctrl+Alt+Shift>> (из-за особенности подключения microLYNX, через него проходит все, что клавиатура отправляет в компьютер). При переходе в этот режим трекбол выводит на стандартный вывод текстовые подсказки и предлагает пользователю набирать на клавиатуре специальные команды, изменяющие его настройки.

\begin{figure}[h]
    \centering
    \includegraphics[scale=0.4]{1986_honeywell_microlynx_trackball/hand_60.jpg}
    \caption{Трекбол microLYNX в комплекте с моделью руки человека}
    \label{fig:microLYNXHand}
\end{figure}

В отношении эргономики размеры трекбола, подставка под запястье, а также скругленные грани и углы создают достаточно комфортные условия для работы (рис. \ref{fig:microLYNXHand}). Однако кнопки расположены далеко от шара и существенно ниже по высоте, что лишает пользователя как возможности нажимать их одной рукой без перемещения кисти, так и двумя руками, поскольку при вращении шара кнопки закрыты ладонью. Поэтому microLYNX трудно использовать в условиях графического интерфейса, требующего интенсивной работы с постоянным чередованием перемещений курсора и нажатием кнопок. Однако трекбол может оказаться достаточно эффективен в различных приложениях по управлению оборудованием.

\begin{figure}[h]
    \centering
    \includegraphics[scale=0.52]{1986_honeywell_microlynx_trackball/inside_60.jpg}
    \caption{microLYNX в разобранном виде}
    \label{fig:microLYNXInside}
\end{figure}

Внутреннее устройство данного трекбола показано на рис. \ref{fig:microLYNXInside}, что позволяет классифицировать его как оптомеханическое устройство. Ролики реализованы с использованием подшипников и валов из нержавеющей стали, что обеспечивает максимальную надежность и долговечность конструкции. Высокая надежность трекбола и эргономика, хорошо подходящая для ряда технических задач, сделали данную модель <<долгожителем>>: устройство не меньше десяти лет выпускалось для индустриальных применений под различными марками, сохранив неизменными внешний вид и механическую часть конструкции.

\begin{thebibliography}{9}
\bibitem {comlynx} Trackballs: Stationary mice // PC Magazine. August 1987, page 199-202 \url{https://trackballs.eu/media/Fulcrum/PC%20Mag%20Aug-1987%20p199-202.pdf}
\bibitem {lx200} Disc Instruments LX200 \url{https://web.archive.org/web/20220501213039/https://forum.trackballs.eu/viewtopic.php?f=17&t=16}
\end{thebibliography}
\end{document}
