\documentclass[11pt, a4paper]{article}
\input{preamble.tex}
\switchlang{ru}
\begin{document}

\title{1986 "--- American Mouse}
\date{}
\maketitle
\selectlanguage{russian}
В 1986 году компания American Computer and Peripheral выпустила настольный компьютер American 286 на основе процессора i80286 \cite{adv}, комплектовавшийся манипулятором того же производителя под названием American Mouse (рис. \ref{fig:AmericanPic}). Цена устройства в отдельной продаже составляла \$125 \cite{review}.

\begin{figure}[h]
    \centering
    \includegraphics[scale=0.6]{1986_american_mouse/pic_30.jpg}
    \caption{American Mouse}
    \label{fig:AmericanPic}
\end{figure}

В углублении на слегка скошенной передней части крышки корпуса мыши расположены три феноменально узкие кнопки. Массивные переключатели, расположенные под кнопками, вынудили разработчиков применить асимметричное расположение кабеля мыши, проходящего между левой и средней кнопками (рис. \ref{AmericanTopAndBottom}). Нижняя сторона корпуса содержит съёмное кольцо для извлечения шара и очистки мыши. На корпусе отсутствуют какие-либо надписи или эмблемы, но на разъеме присутствует выполненная методом литья надпись <<Z-nix>>, указывающая на фактического производителя мыши.

\begin{figure}[h]
    \centering
    \includegraphics[scale=0.7]{1986_american_mouse/top_60.jpg}
    \includegraphics[scale=0.7]{1986_american_mouse/bottom_60.jpg}
    \caption{American Mouse, вид сверху и снизу}
    \label{AmericanTopAndBottom}
\end{figure}

В плане размера манипулятор представляет собой типичное для 80-х годов оптомеханическое устройство управления курсором (рис. \ref{fig:AmericanSize}).

\begin{figure}[h]
    \centering
    \includegraphics[scale=0.5]{1986_american_mouse/size_30.jpg}
    \caption{Изображение American Mouse на размерном коврике с шагом сетки 1~см}
    \label{fig:AmericanSize}
\end{figure}

Во внешнем виде American Mouse прослеживается индустриальный дизайн. При этом угловатый корпус оснастили скошенными гранями, чтобы обеспечить более комфортное расположение ладони; однако необходимость нажимать на чрезвычайно узкие кнопки, врезающиеся в подушечки пальцев, сводит на нет эргономические усилия разработчиков и позволяют причислить данный манипулятор к ряду устройств с наиболее сомнительным дизайном (рис. \ref{fig:AmericanHand}).

\begin{figure}[h]
    \centering
    \includegraphics[scale=0.5]{1986_american_mouse/hand_30.jpg}
    \caption{American Mouse с моделью руки человека}
    \label{fig:AmericanHand}
\end{figure}

Americanl Mouse имеет разъём D-SUB и интерфейс подключения RS-232. Устройство работает по протоколу Mouse Systems, однако информация об этом отсутствует в сопроводительной брошюре, а драйвер данного протокола во время присутствия мыши на рынке можно было найти в основном в комплекте с оптическими мышами Mouse Systems, что ограничивало возможность использования American Mouse со сторонним программным обеспечением. В комплекте с мышью поставлялся единственный программный пакет "--- графический редактор Compu-Brush, представлявший собой ребрэндинг программы PC Paint.

\begin{figure}[h]
    \centering
    \includegraphics[scale=0.9]{1986_american_mouse/inside_60.jpg}
    \caption{American Mouse в разобранном виде}
    \label{fig:AmericanInside}
\end{figure}

В разобранном виде манипулятор показан на рис. \ref{fig:AmericanInside}, где можно увидеть достаточно стандартную конструкцию опто-механической мыши. Особенностью является плотное заполнение внутреннего пространства корпуса различными пластмассовыми деталями, которые отсутствуют в конструкциях практически пустотелых оптомеханических мышей более позднего периода.

\begin{thebibliography}{9}
\bibitem {adv} I love American (advertising). // PC MAGAZINE, V. 5, No. 18. October 1986, p. 157. \url{https://archive.org/details/PC-Mag-1986-10-28/page/n159}
\bibitem {review} Barr Ch. Mice for mainstream applications // PC MAGAZINE, V. 6, No. 14., August 1987, pp. 119 – 146 \url{https://archive.org/details/PC-Mag-1987-08-01/page/n121}
\end{thebibliography}
\end{document}
