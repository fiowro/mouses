\documentclass[11pt, a4paper]{article}
\input{preamble.tex}
\switchlang{ru}
\begin{document}

\title{199x "--- NEOS Trad Mouse}
\date{}
\maketitle
\selectlanguage{russian}
ышь Neos TRAD mouse 200 (рис.\ref{fig:C300Pic}) также имеющая обозначение NEOS NT-200  -- это одно из поздних устройств, представленное на рынок компанией Nippon (или Nihon) Electronics (NEOS) \cite{neos}.

\begin{figure}[h]
    \centering
    \includegraphics[scale=1.6]{mouse/fpic_30.jpg}
    \caption{Neos NT-200 mouse}
    \label{fig:C300Pic}
\end{figure}

Компания NEOS в основном известна мышью MS-10 и MS-30 для платформы MSX, выпущенной в 1985 году и адаптированной также для платформы Commodore 64 \cite{msx}. Компьютеры MSX выпускались с 1983 по 1995 год. Как можно заключить из исследования мыши Neos NT-200, после прекращения выпуска MSX-совместимых манипуляторов компания еще какое-то время выпускала мышей для других компьютерных архитектур. В частности, данный экземпляр представляет собой шинную мышь, предназначенную для работы с компьютерами семейства PC-98.

Персональные компьютеры архитектуры NEC PC-9800, или просто PC-98 выпускались с 1982 по 2003 годы. Архитектура PC-98 построена на базе x86-процессоров, но без совместимости с архитектурой IBM PC. В более поздних модификациях серии, таких как PC-9801F3, был добавлен встроенный порт для подключения bus-mouse, что позволяло подключать мышь без использования отдельной интерфейсной платы  \cite{info_7}.

Корпус мыши выполнен из глянцево-белого пластика, имеет обтекаемую форму, с колоколообразным поперечным  сечением в средней части и куполообразным -- в задней (ближней к пользователю) части мыши (рис. \ref{fig:C300TopAndBottom}). На боковой стороне мыши присутствует ее название, нанесенное красной краской. В передней части корпуса находятся две крупные кнопки, вписанные в его форму и занимающие до трети длины мыши. Нижняя сторона демонстрирует три накладки из низкофрикционного материала, ярлык с техническими данными, а также поворотное кольцо-защелку, позволяющее извлечь шар для чистки мыши.

\begin{figure}[h]
    \centering
    \includegraphics[scale=0.85]{mouse/top_30.jpg}
    \includegraphics[scale=0.85]{mouse/bottom_30.jpg}
    \caption{Neos NT-200 mouse, вид сверху и снизу}
    \label{fig:C300TopAndBottom}
\end{figure}

Мышь симметрична, благодаря чему одинаково удобна для использования как правой, так и левой рукой. Форма корпуса достаточно удобна для захвата ладонью: кнопки предоставляют достаточную площадь для размещения пальцев (рис. \ref{fig:C300Hand}), задняя часть корпуса -- опору для ладони. 

\begin{figure}[H]
    \centering
    \includegraphics[scale=0.35]{mouse/hand_30.jpg}
    \caption{Neos NT-200 mouse с моделью руки человека}
    \label{fig:C300Hand}
\end{figure}

\begin{figure}[h]
    \centering
    \includegraphics[scale=0.7]{mouse/inside_30.jpg}
%    \includegraphics[scale=0.15]{mouse/inside_30.jpg}
    \caption{Neos NT-200 mouse в разобранном виде}
    \label{fig:C300Inside}
\end{figure}

Фотография мыши в разобранном виде (рис. \ref{fig:C300Inside}) демонстрирует оптомеханическую конструкцию, типичную для первой половины 90-х годов, а также надпись Made in Taiwan, свидетльствующую о том, что в поздний период NEOS сменила производителя, выпускавшего по контракту компоненты мышей (модели для архитектуры MSX производились для NEOS японской компанией Mitsumi Electric).

%NEOS 
%TRAD
%NT-V800 AUTO

%A completely new mouse born from the pursuit of ease of use. Backed by NEOS's tradition, this new concept mouse was made possible by NEOS. Its superior operability, smooth movement, and comfortable fit are immediately apparent. NEOS has incorporated the extensive know-how it has accumulated as a leading mouse company, and this is one of the conclusions it has reached as a new generation mouse.

%cleaning

%Occasionally, remove the back and lid of the mouse, take out the pole, and use a soft cloth to wipe off any dirt on the pole and the roller that comes into contact with the ball inside the mouse. Do not disassemble any parts other than the back and lid.

%Compatible models

%DOS/V machines from various companies (Compaq, Fujitsu FM/V, etc.), Toshiba J3100 series, Dyna Book series, IBM Japan ThinkPad

%Please contact the support center for information on compatible models for new products, etc.

%specification

%Interface PS/2 mouse 1/F, RS232C serial 1/F

%switch

%micro SW

%decomposition energy

%Fixed mode: 100, 200, 400, 800 AUTO mode: 100-800

%cable length

%1.5m

%Standard price ¥5,500

%Standard price does not include consumption tax

%T4956392731212

%Nippon Electronics Corporation, pursuing Human Electronics Communication

%3-4-8 Jingumae, Shibuya-ku, Tokyo 150

%Support Center TEL: (03)5474-4180

%Made in Tarwan


\begin{thebibliography}{99}
	\bibitem{neos} All Nippon Electronics (NEOS) hardware that they developed -- Generation MSX \url{https://www.generation-msx.nl/company/nippon-electronics-neos/102/hardware/"}
	\bibitem{msx} Category:Mice -- MSX Wiki \url{https://www.msx.org/wiki/Category:Mice}.
	\bibitem {info_7} PC-98 -- Wikipedia \url{https://en.wikipedia.org/wiki/PC-98}
\end{thebibliography}

\end{document}
