\documentclass[11pt, a4paper]{article}
\input{preamble.tex}
\switchlang{ru}
\begin{document}

\title{1984 "--- Siemens PC-D mouse}
\date{}
\maketitle
\selectlanguage{russian}
В 1984 году компания Siemens анонсировала персональный компьютер PC-D, работающий под управлением MS DOS, но аппаратно несовместимый с IBM PC. Этот компьютер был основан на процессоре Intel 80186, имел проприетарный монохромный графический адаптер с 12-дюймовым дисплеем, нестандартную клавиатуру иоперативную память объемом от 128 Кб до 1 Мб. Программное обеспечение включало MS DOS 2.11 и Windows 1.0, а также ряд офисных приложений и несколько простых игр \cite{wiki}. Годом позже список его дополнительных периферийных устройств пополнила двухкнопочная мышь (рис. \ref{fig:SiemensPCDPic}) \cite{blog}.

\begin{figure}[h]
    \centering
    \includegraphics[scale=0.6]{1985_siemens_pcd_mouse/pic_30.jpg}
    \caption{Мышь Siemens PC-D Mouse}
    \label{fig:SiemensPCDPic}
\end{figure}

Как можно видеть (рис. \ref{fig:SiemensPCDTopBottom}), мышь выполнена в контрастной черно-бежевой цветовой схеме (однако встречаются также фотографии данной мыши в однотонном бежевом корпусе). Сверху присутствуют две контрастные кнопки и название компании; в целом корпус имеет сложную рубленую форму и не содержит дополнительных элементов. Нижняя часть содержит прорезиненный шар и фиксирующее кольцо, которое можно сдвинтуть в сторону, чтобы извлечь шар для чистки мыши. Вокруг фиксирующего кольца наблюдается контрастная кольцеобразная накладка из низкофрикционного материала, являющася отличительной чертой многих мышей 80-х годов, разработанных в Японии.

\begin{figure}[h]
    \centering
    \includegraphics[scale=0.45]{1985_siemens_pcd_mouse/top_30.jpg}
    \includegraphics[scale=0.45]{1985_siemens_pcd_mouse/bottom_30.jpg}
    \caption{Siemens PC-D Mouse, вид сверху и снизу}
    \label{fig:SiemensPCDTopBottom}
\end{figure}

В плане размера манипулятор представляет собой типичное для 80-х годов оптомеханическое устройство управления курсором (рис. \ref{fig:SiemensPCDSize}).

\begin{figure}[h]
    \centering
    \includegraphics[scale=0.42]{1985_siemens_pcd_mouse/size_30.jpg}
    \caption{Изображение Siemens PC-D Mouse на размерном коврике с шагом сетки 1~см}
    \label{fig:SiemensPCDSize}
\end{figure}

В плане эргономики и во внешнем виде PC-D Mouse прослеживается ярко выраженный индустриальный дизайн. При этом большое количество углов и плоских граней отчасти компенсируется закругленными стыками граней. Кнопки расположены на наклонной передней грани и заходят на верхнюю стенку тела (рис. \ref{fig:SiemensPCDHand}). По рекомендации производителя их следует нажимать средним и безымянным пальцем, придерживая мышь с боков указательным пальцем и мизинцем \cite{manual}.

\begin{figure}[h]
    \centering
    \includegraphics[scale=0.42]{1985_siemens_pcd_mouse/hand_30.jpg}
    \caption{Изображение Siemens PC-D Mouse с моделью руки человека}
    \label{fig:SiemensPCDHand}
\end{figure}

Внутреннее устройство Siemens PC-D Mouse показано на рис. \ref{fig:SiemensPCDInside}, что позволяет классифицировать мышь как устройство с опто-механическим энкодером, изготовленное в Японии.

\begin{figure}[h]
    \centering
    \includegraphics[scale=0.8]{1985_siemens_pcd_mouse/inside_30.jpg} 
    \caption{Siemens PC-D Mouse в разобранном состоянии}
    \label{fig:SiemensPCDInside}
\end{figure}

Очевидно, реальным изготовителем мыши была японская компания Alps, часто выступавшая подрядчиком в разработке мышей по заказу других фирм (услугами Alps в частности пользовались такие компании, как IBM, Microsoft, NeXT, Intergraph). Alps обычно изготавливала для каждого заказчика дизайн мыши с уникальной формой корпуса и расположением кнопок, но использовала одно и то же конструктивное исполнение в нескольких устройствах. В частности, данная мышь совпадает (включая нижнюю часть корпуса, узел оптомеханического преобразования на основе закрытых энкодеров, печатную плату с радиоэлементами) с оригинальной мышью компьютеров NeXT, выпускавшихся с 1988 года, и с мышью компьютеров Intergraph InterPro 2020, появившихся на рынке в 1992 году.

\begin{thebibliography}{9}
\bibitem {wiki} Siemens PC-D -- Wikipedia \url{https://en.wikipedia.org/wiki/Siemens_PC-D}
\bibitem {blog} Der PC von Siemens. Heinz Nixdorf MuseumsForum \url{https://blog.hnf.de/der-pc-von-siemens/}
\bibitem {manual} MAUS. Bedienelement für Siemens PC-D. Anwendungsbeschreibung. Ausgabe Dezember 1985. \url{https://github.com/fiowro/mouses/blob/main/source/OCR/PC-D_mouse.pdf}
\end{thebibliography}
\end{document}
