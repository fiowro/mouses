\documentclass[11pt, a4paper]{article}
\input{preamble.tex}
\switchlang{en}
\begin{document}

\title{1983 "--- Microsoft Green Eyed Mouse}
\date{}
\maketitle
\selectlanguage{english}

Microsoft's first mouse was released in 1983 and was also the first product from its hardware division, which the company opened a year earlier. Because of the two green buttons, this model is known as the ``green-eyed mouse''.

Since Microsoft at that time did not have enough experience in the hardware development and manufacture, the real manufacturer of the mouse was the Japanese company Alps -- the manufacturer of the first Japanese mouse, MZ-1X10, introduced in the same 1983.

\begin{figure}[h]
   \centering
    \includegraphics[scale=0.6]{1983_microsoft_green_eyed_mouse/pic_30.jpg}
    \caption{Microsoft Green Eyed Mouse}
    \label{fig:MicrosoftGreenEyedPic}
\end{figure}

Technically, Microsoft's first mouse and the MZ-1X10 have a lot in common. However, the body of the MZ-1X10 mouse is a cuboid with slightly rounded edges and a pair of rectangular buttons on the upper side of the body, while the body of the Microsoft mouse has a more complex shape, and the buttons are shifted to the sloping front side. (fig.  \ref{fig:MicrosoftGreenEyedPic}).

\begin{figure}[h]
    \centering
    \includegraphics[scale=0.6]{1983_microsoft_green_eyed_mouse/top_60.jpg}
    \includegraphics[scale=0.6]{1983_microsoft_green_eyed_mouse/bottom_60.jpg}
    \caption{Microsoft Green Eyed Mouse, top and bottom views}
    \label{fig:MicrosoftGreenEyedTopAndBottom}
\end{figure}

Clearly, Microsoft placed a lot of importance on the design of their mouse. The light cream-colored case is made in a minimalist style, the only elements are two contrasting green buttons and a barely noticeable company name at the edge of the body closest to the user (fig. \ref{fig:MicrosoftGreenEyedTopAndBottom}). Motion is detected by a heavy steel ball located towards the back of the mouse, and three small, smooth-polished balls act as feet to minimize friction. Also, a removable ring is provided in the bottom of the case, which allows you to remove the ball and get rid of the collected debris; however, the snap ring option has not yet been invented, so it must be unscrewed with a screwdriver.

\begin{figure}[h]
    \centering
    \includegraphics[scale=0.5]{1983_microsoft_green_eyed_mouse/size_30.jpg}
    \caption{Microsoft Green Eyed Mouse on a graduated pad with a grid step of 1~cm}
    \label{fig:MicrosoftGreenEyedSize}
\end{figure}

Despite the small size of the mouse (fig. \ref{fig:MicrosoftGreenEyedSize}), it is quite heavy. Clearly, Microsoft's focus on body shape was more or less beneficial to ergonomics. Compared to its closest relative, the MZ-1X10 mouse, it allows the palm and fingers to be placed in a more natural position. (fig. \ref{fig:MicrosoftGreenEyedHand}).

\begin{figure}[h]
    \centering
    \includegraphics[scale=0.5]{1983_microsoft_green_eyed_mouse/hand_30.jpg}
    \caption{Microsoft Green Eyed Mouse with a human hand model}
    \label{fig:MicrosoftGreenEyedHand}
\end{figure}

Early versions of the Microsoft mouse had a bus interface and were equipped with a special adapter board. Later versions appeared with a serial interface and a 25 or 9-pin connector \cite{mouses}. Mouse resolution was only 100 DPI \cite{review}.

The internal structure of the mouse can be seen in fig. \ref{fig:MicrosoftGreenEyedInside}. The mouse uses closed contact encoders. When compared with the MZ-1X10 mouse, the technical design is almost identical: the differences come down to the configuration of the printed circuit board and are obviously caused by the front placement of the buttons.

 \begin{figure}[h]
    \centering
    \includegraphics[scale=1]{1983_microsoft_green_eyed_mouse/inside_30.jpg}
    \caption{Microsoft Green Eyed Mouse disassembled}
    \label{fig:MicrosoftGreenEyedInside}
\end{figure}

\begin{thebibliography}{9}
\bibitem{mouses} Microsoft Green Eyed Mouse \url{https://web.archive.org/web/20211205011010/https://www.oldmouse.com/mouse/microsoft/greeneyed.shtml}
\bibitem{review} Hart G. Building a better mouse interface // PC Magazine, February 25, 1986. -- pp. 167-170. \url{https://archive.org/details/PC-Mag-1986-02-25/page/173/mode/2up}
\end{thebibliography}
\end{document}
