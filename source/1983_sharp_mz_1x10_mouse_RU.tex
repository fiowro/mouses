\documentclass[11pt, a4paper]{article}
\input{preamble.tex}
\switchlang{ru}
\begin{document}

\title{1983 "--- Sharp MZ-1X10 Mouse}
\date{}
\maketitle
\selectlanguage{russian}

Мышь Sharp MZ-1X10, получившая известность как первая мышь, выпущенная в Японии, появилась на рынке в 1983 году, практически одновременно с первой мышью Microsoft, известной из-за двух зеленых кнопок как <<зеленоглазая мышь>>. Поскольку у Microsoft на тот момент было недостаточно опыта в разработке и изготовлении аппаратного обеспечения, реальным производителем обеих мышей выступила японская компания Alps. Мышь MZ-1x10 предназначалась для использования с компьютерами Sharp MZ-5500 "--- построенными на базе процессора Intel 8086, работавшими под управлением MS-DOS и рассчитанными на бизнес-пользоватлей \cite{review, wiki}.

\begin{figure}[h]
   \centering
    \includegraphics[scale=0.7]{1983_sharp_mz_1x10_mouse/pic_30.jpg}
    \caption{Sharp MZ-1X10 Mouse}
    \label{fig:SharpMZ1x10Pic}
\end{figure}

Корпус мыши MZ-1X10 представляет собой параллелепипед со слегка скругленными гранями и парой прямоугольных кнопок на верхней стороне корпуса (рис.  \ref{fig:SharpMZ1x10Pic}). 
\begin{figure}[h]
    \centering
    \includegraphics[scale=0.55]{1983_sharp_mz_1x10_mouse/top_15.jpg}
    \includegraphics[scale=0.55]{1983_sharp_mz_1x10_mouse/bottom_15.jpg}
    \caption{Sharp MZ-1X10 Mouse, вид сверху и снизу}
    \label{fig:SharpMZ1x10TopAndBottom}
\end{figure}

Нижняя сторона демонстрирует стальной шар, три металлических шарика, облегчающие скольжение мыши, и съёмное кольцо, позволяющее извлечь шар для чистки. Вариант кольца на защелках еще не был придуман, поэтому его требуется отвинчивать с помощью отвертки. Пластиковый ограничитель, защищающий провод от повреждения в месте его выхода из корпуса мыши, конструкцией не предусмотрен (рис. \ref{fig:SharpMZ1x10TopAndBottom}).

\begin{figure}[h]
    \centering
    \includegraphics[scale=0.5]{1983_sharp_mz_1x10_mouse/size_30.jpg}
    \caption{Sharp MZ-1X10 Mouse на размерном коврике с шагом сетки 1~см}
    \label{fig:SharpMZ1x10Size}
\end{figure}

Несмотря на малые размеры мыши (рис. \ref{fig:SharpMZ1x10Size}), она довольно тяжелая. Дизайн мыши предельно аскетичен, форма ассоциируется с блоком питания домашних плееров и некоторых других бытовых устройств соответствующего временного периода (либо, благодаря специфицеским кнопкам, с автоматическими предохранителями сетей электроснабжения). Очевидно, скошенная задняя грань должна обеспечить более комфортное расположение ладони, однако с учетом размеров мыши существенных улучшений в эргономику это не привносит (рис. \ref{fig:SharpMZ1x10Hand}). Кнопки имеют не собенно большой размер, что также не добавляет пользователю комфорта.

\begin{figure}[h]
    \centering
    \includegraphics[scale=0.5]{1983_sharp_mz_1x10_mouse/hand_30.jpg}
    \caption{Sharp MZ-1X10 Mouse с моделью руки человека}
    \label{fig:SharpMZ1x10Hand}
\end{figure}

Мышь подключалась к компьютеру по разновидности последовательного интерфейса. Левая кнопка использовалась для возврата к началу координат (левый верхний угол экрана), а правая работала как основная кнопка мыши, то есть генерировала клик в координатах, соответствовавших положению курсора \cite{manual}.

 \begin{figure}[h]
    \centering
    \includegraphics[scale=0.8]{1983_sharp_mz_1x10_mouse/inside_30.jpg}
    \caption{Sharp MZ-1X10 Mouse в разобранном виде}
    \label{fig:SharpMZ1x10Inside}
\end{figure}

Внутреннее устройство мыши показано на рис. \ref{fig:SharpMZ1x10Inside}. В мыши использованы закрытые контактные энкодеры. При сравнении с зеленоглазой мышью Microsoft обнаруживается почти полная идентичность конструкции: отличия наблюдаются в конфигурации печатной платы и связаны с расположением кнопок на верхней стороне корпуса.

\begin{thebibliography}{9}
\bibitem{review}  Ohishi Nobuaki. MZ-1X10 (mouse) [in Japanese] \url{http://retropc.net/ohishi/museum/mz1x10.htm}
\bibitem{wiki} Sharp MZ -- Wikipedia \url{https://en.wikipedia.org/wiki/Sharp_MZ#MZ-3500/5500/6500_group}
\bibitem{manual} Sharp MZ-1X10 Instruction Manual \url{https://www.manualslib.com/download/900861/Sharp-Mz-1x10.html}
\end{thebibliography}
\end{document}
