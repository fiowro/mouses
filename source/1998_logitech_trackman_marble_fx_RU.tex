\documentclass[11pt, a4paper]{article}
\input{preamble.tex}
\switchlang{ru}
\begin{document}

\title{1998 "--- Трекбол Logitech TrackMan Marble FX}
\date{}
\maketitle
\selectlanguage{russian}
Безусловно, основной отличительной особенностью трекбола Trackman Marble FX (рис. \ref{fig:trackman}), выпущенного компанией Logitech в 1998 году, является его необычная форма, предоставляющая опору для запястья и открывающая доступ к шару сразу с обеих сторон корпуса. По задумке производителя, это позволяет перемещать шар либо одним пальцем, либо двумя одновременно (большим и указательным) для максимальной точности малых перемещений курсора \cite{marbleBoot}.

\begin{figure}[h]
    \centering
    \includegraphics[scale=0.38]{1998_logitech_trackman_marble_fx/pic_30.jpg}
    \caption{Изображение Logitech TrackMan Marble FX}
    \label{fig:trackman}
\end{figure}

Сложная форма, наводящая на мысли о скалах, подверженных длительной работе ветра или волн, соответствует скорее стилю дизайнерских манипуляторов Луиджи Колани, чем каким-либо прежним разработкам Logitech. Трекбол получился ярким, запоминающимся, и совершенно заслуженно стал призером престижной премии <<IF Design Award>> проектной организации «iF International Forum Design GmbH» \cite{award}.

\begin{figure}[h]
    \centering
    \includegraphics[scale=0.32]{1998_logitech_trackman_marble_fx/top_60.jpg}
    \includegraphics[scale=0.32]{1998_logitech_trackman_marble_fx/bottom_60.jpg}
    \caption{Изображение Logitech TrackMan Marble FX, вид сверху и снизу}
    \label{fig:trackmanTopAndBottom}
\end{figure}

Клавиш у TrackMan Marble FX четыре: три на левой стороне корпуса и одна на правой (рис. \ref{fig:trackmanTopAndBottom}).
Кнопки белого цвета отвечают за стандартные функции кнопок мыши, а при нажатии на красную кнопку, расположенную недалеко от шара, поставлявшийся в комплекте с трекболом драйвер переключался между режимом перемещения курсора и режимом прокрутки/масштабирования.

Регулярный узор из тёмных точек на поверхности шара вызван применением оптического датчика для считывания перемещений. Подключение к компьютеру осуществляется по интерфейсу PS/2.

\begin{figure}[h]
    \centering
    \includegraphics[scale=0.4]{1998_logitech_trackman_marble_fx/size_30.jpg}
    \caption{Изображение Logitech TrackMan Marble FX на размерном коврике с шагом сетки 1 см}
    \label{fig:trackmanSize}
\end{figure}

TrackMan Marble FX имеет крупные размеры и шар б\'{о}льшего диаметра, чем у предыдущих трекболов Logitech (рис. \ref{fig:trackmanSize}). Трекбол асимметричен и предназначен для использования исключительно правой рукой, а наклон и изгибы корпуса нацелены на то, чтобы запястье руки пользователя находилось в максимально естественном положении (рис. \ref{fig:trackmanHand}).

\begin{figure}[h]
    \centering
    \includegraphics[scale=0.6]{1998_logitech_trackman_marble_fx/hand_30.jpg}
    \caption{Изображение Logitech TrackMan Marble FX с моделью руки человека}
    \label{fig:trackmanHand}
\end{figure}

В статьях, опубликованных в ZDNET \cite{zdnet} и в журнале Computer Gaming World \cite{gaming} отмечается, что TrackMan Marble FX достаточно удобен в использовании, но отверстие в корпусе, задуманное чтобы шар можно было перемещать пальцами с двух сторон, на практике не очень подходит для этой цели из-за своего малого размера (скорее, оно несет эстетическую функцию,а также облегчает извлечение шара из корпуса для чистки). Кроме того, часть корпуса, являющаяся подставкой под запястье, вызывала смешанную реакцию у некоторых пользователей, которые предпочли бы большую гибкость за счет использования в этой роли съемного аксессуара.

Разбор трекбола (рис. \ref{fig:trackmanInside}) показывает конструкцию, являющуюся аналогом оптической мыши, считывавшей изменения яркости с помощью специального коврика с нанесенной на нём сеткой (роль коврика играет рисунок на шаре).

\begin{figure}[h]
    \centering
    \includegraphics[scale=0.6]{1998_logitech_trackman_marble_fx/inside_30.jpg}
    \caption{Изображение Logitech TrackMan Marble FX изнутри}
    \label{fig:trackmanInside}
\end{figure}

\begin{thebibliography}{9}
\bibitem{marbleBoot} Pure lust. High-tech toys and tools with the right stuff. // Boot Magazine: Issue 20 - April 1998. -- p. 18. \url{https://archive.org/details/boot-magazine-issue20-pc-notebook-autopsy-apr-1998/page/n19/mode/2up}
\bibitem{gaming} Case L. Revew - Logitech Trackman Marble/FX. Absolutely Marble-ous. // Computer Gaming World: Issue 168 - July 1998. -- p. 127.  \url{https://archive.org/details/Computer_Gaming_World_Issue_168/page/n129/mode/2up}
\bibitem {zdnet} Watson J.A. Trackballs that I have known and loved: A history in hardware. - ZDNET, March 16, 2017. \url{https://www.zdnet.com/article/trackballs-that-i-have-known-and-loved-a-history-in-hardware/}
\bibitem {award} IF Design - TrackMan Marble FX \url{https://ifdesign.com/en/winner-ranking/project/trackman-marble-fx/16939}
\end{thebibliography}

\end{document}
