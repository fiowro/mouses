\documentclass[11pt, a4paper]{article}
\input{preamble.tex}
\switchlang{en}
\begin{document}

\title{1985 -- Torrington Manager Mouse}
\date{}
\maketitle
\selectlanguage{english}

The Manager Mouse was released in 1985 by the Torrington Company, which began its business as Excelsior Needle Company in the mid-19th century -- an innovative manufacturer of sewing needles. In the more than a century preceding the release of the Manager Mouse, the company had undergone several phases of product expansion into neighboring markets, such as bicycle spokes and needle bearings. Shortly before 1985 the company became an autonomous subsidiary of Ingersoll Rand Inc., abandoning needle production entirely. At that time, Torrington was known primarily as a major international bearing manufacturer \cite{torrington} -- and, obviously, continued to actively experiment with new markets, which resulted in the creation of this computer mouse.

\begin{figure}[h]
   \centering
    \includegraphics[width=\textwidth]{1985_torrington_manager_mouse/pic_30.jpg}
    \caption{Manager Mouse}
    \label{fig:ManagerMousePic}
\end{figure}

The mouse was released in two versions: a wired version (model 1001C) and a wireless infrared version (model 1001C-IR). Besides being the first wireless mouse to be widely sold, Manager Mouse also attracted attention for its unusual wheel-based motion detection. Several reviews were devoted to it in IT journals, and in subsequent years, Manager Mouse (especially the wireless one) could be found under other brands, such as Maynard Electronics \cite{admaynard} and Numonics Corporation \cite{adnumonics} (in 1989, Numonics went a step further and released its own version of the Manager Mouse in a more modern case).

The year when the mouse had appeared, as well as its original developer, can be figured out using the FCC ID code, which, according to the Federal Communications Commission database, shows that the mouse was released in 1985 by Torrington, in Connecticut.

The mouse is made in a beige case, expressing strict industrial design and minimalism. The case has an almost rectangular shape, if you do not count the forward-slanted part of the upper side, on which there are three elongated rectangular buttons of a darker color.

\begin{figure}[h]
    \centering
    \includegraphics{1985_torrington_manager_mouse/top_30.jpg}
    \includegraphics{1985_torrington_manager_mouse/bottom_30.jpg}
    \caption{Manager Mouse, top and bottom views}
    \label{fig:ManagerMouseTopAndBottom}
\end{figure}

The mouse cable is not provided with protection from mechanical damage at the point where it exits the case. On the top side, in the user's wrist area, there is a label saying ``Manager Mouse''. On the bottom side (fig. \ref{fig:ManagerMouseTopAndBottom}), one can find a sticker with some technical information, as well as two small cone-shaped wheels used to register movements instead of the ball which would be typical for mechanical mice. This unusual design comes back to Douglas Engelbart's first mice: the wheels are positioned orthogonally to each other, so that one always rotates while the other always slides during the longitudinal and lateral movements. However, compared to Engelbart's mouse, the Manager Mouse's wheels are significantly smaller, and their axes are positioned at an angle to the horizontal plane, which should improve the registration of diagonal movements.

\begin{figure}[h]
    \centering
    \includegraphics[width=\textwidth]{1985_torrington_manager_mouse/size_30.jpg}
    \caption{Manager Mouse on a graduated pad with a grid step of 1~cm}
    \label{fig:ManagerMouseSize}
\end{figure}

The case is of a size typical for the first half of the 1980s (fig. \ref{fig:ManagerMousePic}, \ref{fig:ManagerMouseSize}). Based on the strong visual similarity, it can be assumed that the case shape was borrowed over the next two years by the Taiwanese company KYE Systems for the first generation of their own Genius mice (starting with the GM-3 and ending with the GM-5 ``old'' design models).

Manager Mouse versions with an RS-232 interface (both a 9-pin and a 25-pin connector variants were produced) operate using the Mouse Systems protocol. A variant equipped with an adapter that plugs into the system bus is also mentioned in \cite{maynard}.

In terms of ergonomics, the GM-5 mouse does not have many advantages. The user experience obviously suffers from the severe rectangularity of the body, given that it has a significant height and cannot provide much support for the palm (fig. \ref{fig:ManagerMouseHand}). Furthermore, as noted in \cite{managerwheels}, the wheel design does have some drawbacks: they occasionally slip (especially on smooth surfaces), and diagonal cursor movement, despite the tilted wheel axes, still appears uneven.

\begin{figure}[h]
    \centering
    \includegraphics[width=\textwidth]{1985_torrington_manager_mouse/hand_30.jpg}
    \caption{Manager Mouse with a human hand model}
    \label{fig:ManagerMouseHand}
\end{figure}

The internal structure of the mouse is shown in fig. \ref{fig:ManagerMouseInside}. As you can see, it is a pure mechanical design, and the mechanical encoder disks are mounted directly on the wheel axles. The manufacturer's advertising \cite{buxton} claims that this is a ``patented independent suspension system that has already brought a wide acceptance to the rest of Torrington's Manager Mouse\texttrademark~ line'', which ``keeps the mouse tracking smoothly on virtually any surface, at any angle'', and furthermore, it is reported that the polyurethane wheels ``actually push debris out of the way as they track mouse motion''. According to \cite{buxton}, this wheel-based mouse indeed accumulated slightly less debris than mechanical and optomechanical ball-based mice. In addition, it required less power because of the mechanical encoder, compared to the optomechanical one: all early optomechanical mice and trackballs with an RS-232 interface required an additional power source until low-power LEDs appeared in 1986.

A review \cite{buxton} also mentions that ``two tiny plastic wheels set at right angles to each other, instead of an optical detector or trackball'' with the resolution of 100 DPI (which was, for example, half the resolution of ball-based Logitech and Microsoft mice).

 \begin{figure}[h]
    \centering
    \includegraphics[width=\textwidth]{1985_torrington_manager_mouse/inside_30.jpg}
    \caption{Manager Mouse disassembled}
    \label{fig:ManagerMouseInside}
\end{figure}

~

\begin{thebibliography}{9}
\bibitem {torrington} The Torrington Company / International Directory of Company Histories. ed. by T. Grant. Vol. 13. - St.James Press, USA. 1996. P. 521 \url{https://archive.org/details/internationaldir0013unse/page/520/mode/2up}

\bibitem {admaynard} This Little Fella Means Business [adv.] // PC Magazine, Vol. 5, No. 1, January 14, 1986. P. 37. \url{https://archive.org/details/PC-Mag-1986-01-14/page/n71/mode/2up}

\bibitem {adnumonics} Don't let your mouse get hung-up on itself [adv.] // PC Magazine, Vol. 7, No. 6, March 29, 1988. P. 282 // \url{https://archive.org/details/PC-Mag-1988-03-29/page/n277/mode/2up}

\bibitem {maynard} Hart G. Building a better mouse interface. Maynard mouse // PC Magazine, Vol. 5, No. 4, February 25, 1986. - p. 170--172. \url{https://archive.org/details/PC-Mag-1986-02-25/page/n177/mode/2up}

\bibitem {managerwheels} Numonics corp. Manager mouse, Manager mouse cordless. //  PC Magazine, V. 7, No. 3, February 14, 1989. - p. 268. \url{https://archive.org/details/PC-Mag-1989-02-14/page/n267/mode/2up}

\bibitem {buxton} Manager Mouse Cordless \url{https://www.billbuxton.com/inputTimelineAssets/TorringtonBrochure.pdf}
\end{thebibliography}
\end{document}
