\documentclass[11pt, a4paper]{article}
\input{preamble.tex}

\begin{document}

\title{1999 "--- Trackerball ProTrack 60i trackball}
\date{}
\maketitle

ProTrack 60i произведен фирмой Trackerball "--- обретшим в 1995 году самостоятельность
бывшим подразделением компании Marconi - разработчика самого первого трекбола для радарной установки в 1940-х годах для управления экранным курсором военных радаров \cite{history}.

\begin{figure}[h]
    \centering
    \includegraphics[scale=0.3]{1999_protrack_60i/monstr1_30.jpg}
    \caption{Трекбол ProTrack 60i}
    \label{fig:ProTrack60i}
\end{figure}

Трекбол выпускался в двух модификациях: R60 (ProTrack 60) с характерной расцветкой "--- белый корпус с черным шаром и кнопками, и R60i (ProTrack 60i) с полупрозрачным корпусом и синим шаром с подсветкой, предназначенной для использования в слабоосвещенных помещениях, которая показанна на рис. \ref{fig:ProTrack60i}. Наиболее распространенными интерфейсами подключения являются PS/2 и USB, однако на сайте компании Cursor Controls, в состав которой Trackerball вошла в 2000 году,  также упоминаются интерфейсы RS232, SUN и DEC \cite{cursorcontrols}.

Заметим, что данное устройство является чрезвычайно габаритным трекболом (рис. \ref{fig:ProTrack60iSize}). Диаметр шара составляет 63,5 мм (2.5 дюйма).

\begin{figure}[h]
    \centering
    \includegraphics[scale=0.3]{1999_protrack_60i/monstr2.jpg}
    \caption{Изображение ProTrack 60i на размерном коврике с шагом сетки 1~см}
    \label{fig:ProTrack60iSize}
\end{figure}

Трекбол оснащён тремя кнопками, расположенными с противоположной от пользователя стороны, за шаром.
Согласно предположению, сделанному в \cite{trackballfan}, такое расположение кнопок минимизирует вероятность их случайного нажатия при перемещении шара, что очевидно является важным при использовании трекбола в индустриальных либо военных установках, но несколько затрудняет операцию перетаскивания (не характерную для подобного применения).
Трекбол не имеет колеса прокрутки, однако документация производителя упоминает возможность скроллинга шаром, которая активируется и деактивируется нажатиями на среднюю кнопку.

\begin{figure}[h]
    \centering
    \includegraphics[scale=0.3]{1999_protrack_60i/raz_monstr_60.jpg}
    \caption{Изображение ProTrack 60i с моделью руки человека}
    \label{fig:ProTrack60iHand}
\end{figure}

На рис. \ref{fig:ProTrack60iTopBottom} можно видеть верхнюю и нижнюю стороны трекбола.
На верхней части корпуса присутствует рельефная надпись ProTrack 60i, а также эмблема с надписью <<TRACKERBALL>>.

\begin{figure}[h]
    \centering
    \includegraphics[scale=0.3]{1999_protrack_60i/monstr3_60.jpg}
    \includegraphics[scale=0.3]{1999_protrack_60i/monstr4_60.jpg}
    \caption{ProTrack 60i, вид сверху и снизу}
     \label{fig:ProTrack60iTopBottom}
\end{figure}

На нижней стенке корпуса присутствуют резиновые ножки, фиксирующие надёжное положение на поверности стола, и маркировка ProTrack 60i. Страной изготовления устройства является Великобритания.

\begin{figure}[h]
    \centering
    \includegraphics[scale=0.6]{1999_protrack_60i/razobr3_60.jpg}
    \caption{ProTrack 60i в разобранном состоянии}
    \label{fig:ProTrack60iInside}
\end{figure}

 Внутреннее устройство данного трекбола показано на рис. \ref{fig:ProTrack60iInside}. Как можно видеть,  трекбол выполнен по традиционной оптомеханической схеме, с дополнительным светодиодом, обеспечивающим подсветку шара. Также следует отметить, что ролики реализованы с использованием подшипников и валов из нержавеющей стали, предназначенных для того, чтобы обеспечить максимальную надежность и долговечность конструкции.
По всей видимости, извлечение шара для чистки трекбола невозможно без разборки корпуса.

\begin{thebibliography}{9}
\bibitem{history} History of Cursor Controls, innovators in trackball manufacturing \url{https://www.cursorcontrols.com/about/history/}
\bibitem{cursorcontrols} Cursor Controls Protrack 60 Industrial \url{https://web.archive.org/web/20010311155954/http://www.cursorcontrols.com/protrack60i.htm}
\bibitem{trackerball} The Trackerball Company - Products \url{https://web.archive.org/web/19990128124617/http://www.trackerball.com/PRODUCT.HTM}
\bibitem{trackballfan} Trackball Fan! Cursor Controls ProTrack 60i (R60i) \url{http://www.hykw.com/tbfan/reviews/protrack.shtml}
\end{thebibliography}
\end{document}
