\documentclass[11pt, a4paper]{article}
\input{preamble.tex}
\switchlang{en}
\begin{document}

\title{1984 -- Mindset Joystick}
\date{}
\maketitle
\selectlanguage{english}

Mindset Joystick shown on the figure  \ref{fig:MindsetJoystickPic} was developed as an additional accessory for the Mindset personal computer. The Mindset computer was released in 1984 by Mindset Corporation, and was on sale for only a year. In technical terms, it was partially compatible with the IBM PC, based on the Intel 80186 processor and a non-standard graphics subsystem that had enhanced capabilities, including hardware acceleration of some typical graphics operations \cite{wiki, byteMagazine}. 
Both the joystick and mouse were meant to be connected to one of two ports located on the sides of the keyboard \cite{adv}. The real manufacturer of the joystick, as well as the mouse, was the Japanese company ALPS.

\begin{figure}[h]
   \centering
    \includegraphics[scale=0.8]{1984_mindset_joystick/pic_30.jpg}
    \caption{Mindset Joystick}
    \label{fig:MindsetJoystickPic}
\end{figure}

The joystick has a milky-white body with black and red accents, similar to other parts of the Mindset computer. The computer's name can be seen on the top of the casing (fig. \ref{fig:MindsetJoystickPic}). The buttons are round, red, and almost completely recessed into the casing. The metal stick of the joystick has a plastic tip that matches the color of the body. It is as small as everything else, and judging by its size, it's designed to be held with two fingers.

\begin{figure}[h]
    \centering
    \includegraphics[scale=0.81]{1984_mindset_joystick/top_30.jpg}
    \includegraphics[scale=0.81]{1984_mindset_joystick/bottom_30.jpg}
    \caption{Mindset Joystick, top and bottom views}
    \label{fig:MindsetJoystickTopAndBottom}
\end{figure}

The joystick is a miniature device (fig. \ref{fig:MindsetJoystickSize}), so its creators most likely did not rely on one hand usage too much. It's also worth noting that miniaturization and minimalist design have resulted in a complete lack of adjustment elements: the device's design leaves no room for trimmers, which would eliminate drift by setting zero voltage on the X and Y outputs when the stick is in a vertical position, and it also lacks adjustment screws that would enable or disable the stick's automatic return to a vertical position.

\begin{figure}[h]
    \centering
    \includegraphics[scale=0.42]{1984_mindset_joystick/size_30.jpg}
    \caption{Mindset Joystick on a graduated pad with a grid step of 1~cm}
    \label{fig:MindsetJoystickSize}
\end{figure}

The buttons are duplicating each other and are located on the sides of the body, which isn't the best ergonomic solution for one-handed operation shown in the fig. \ref{fig:MindsetJoystickHand}: the miniature body would shift at the stick or buttons presses. The user was likely intended to hold the joystick in the palm of one hand and control the stick with the fingers of the other (fig. \ref{fig:MindsetJoystickHand2}).

\begin{figure}[h!]
    \centering
    \includegraphics[scale=0.42]{1984_mindset_joystick/hand_30.jpg}
    \caption{Mindset Joystick with a human hand model}
    \label{fig:MindsetJoystickHand}
\end{figure}

\begin{figure}[h!]
    \centering
    \includegraphics[scale=0.51]{1984_mindset_joystick/hand2_30.jpg}
    \caption{Mindset Joystick, two-handed operation}
    \label{fig:MindsetJoystickHand2}
\end{figure}

The internal structure of the joystick can be seen in fig. \ref{fig:MindsetJoystickInside}. As you can see, it is a typical analog joystick design based on two crossed beams, only the mechanical parts are smaller than usual.

As mentioned before, the real manufacturer of the joystick, as well as the mouse, was the Japanese company ALPS -- a contract manufacturer of mice and joysticks for a number of well-known companies, including the first Japanese mouse, the MZ-1X10 mouse, released in 1983, and the first Microsoft mouse, known as the ``green-eyed mouse'' due to the color of the buttons.

 \begin{figure}[h!]
    \centering
    \includegraphics[width=\textwidth]{1984_mindset_joystick/inside_30.jpg}
    \caption{Mindset Joystick disassembled}
    \label{fig:MindsetJoystickInside}
\end{figure}

\begin{thebibliography}{9}
\bibitem {wiki} Mindset (computer) - Wikipedia \url{https://en.wikipedia.org/wiki/Mindset_(computer)}
\bibitem {byteMagazine} Wadlow T. The Mindset Personal Computer // Byte Magazine, Vol. 10, No. 6. June, 1985. - P. 324-232 \url{https://archive.org/details/byte-magazine-1985-06/page/n331/mode/2up}
\bibitem{adv} Mindset Personal Computer System \url{https://archive.org/details/bitsavers_mindsetBrore_3744143/page/n1/mode/2up}
\end{thebibliography}
\end{document}
