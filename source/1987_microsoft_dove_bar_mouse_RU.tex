\documentclass[11pt, a4paper]{article}
\input{preamble.tex}
\switchlang{ru}
\begin{document}

\title{1987 "--- Microsoft Dove Bar Mouse}
\date{}
\maketitle
\selectlanguage{russian}

Мышь, получившая от пользователей прозвище <<Dove Bar Mouse>> в честь мыла похожей формы, появилась в продаже в 1987 году, став третьим поколением мышей Microsoft. Дизайн был разработан Microsoft в сотрудничестве с компанией Matrix Design (впоследствии объединившейся с Hovey-Kelley, разработавшей дизайн мыши Apple Lisa, в компанию IDEO). Производство, как и в случае предыдущих мышей Microsoft, было доверено японской компании Alps. Вероятно, это первая мышь, разработчики которой ставили своей главной задачей именно эргономику. В первую очередь это сказалось на форме корпуса, которая была позаимствована у шлифовального бруска, чтобы автоматически обеспечить удобное положение руки, отработанное на многих поколениях людей \cite{doveBarDesign1, atkinson}.

\begin{figure}[h]
   \centering
    \includegraphics[scale=0.5]{1987_microsoft_dove_bar_mouse/pic_30.jpg}
    \caption{Microsoft Dove Bar Mouse}
    \label{fig:MicrosoftDoveBarPic}
\end{figure}

Помимо нового подхода к выбору формы, третье поколение мышей Microsoft получило и другие улучшения "--- как в плане эстетики и эргономики, так и по части конструкции. Мышь имеет молочно-белый глянцевый корпус (рис. \ref{fig:MicrosoftDoveBarTopAndBottom}); кнопки значительно увеличены в размерах по сравнению с предыдущим поколением, занимают переднюю треть корпуса и полностью вписаны в его форму. На нижней части корпуса присутствуют низкофрикционные накладки, табличка с техническими данными и сдвижное кольцо-защелка, необходимое, чтобы извлечь шар для чистки.

\begin{figure}[h]
    \centering
    \includegraphics[scale=0.5]{1987_microsoft_dove_bar_mouse/top_30.jpg}
    \includegraphics[scale=0.5]{1987_microsoft_dove_bar_mouse/bottom_30.jpg}
    \caption{Microsoft Dove Bar Mouse, вид сверху и снизу}
    \label{fig:MicrosoftDoveBarTopAndBottom}
\end{figure}

Шар смещен в переднюю часть мыши и находится практически в зоне расположения пальцев пользователя. Такая модификация должна была ощутимо улучшить точность позиционирования курсора по сравнению с мышами предыдущих поколений \cite{atkinson}, и очевидно улучшение действительно имело место "--- по крайней мере по сравнению с более ранними конструкциями производства ALPS, в которых шар располагался у ближнего к пользователю края корпуса, освобождая больше места для печатной платы и кнопок. Размеры мыши вполне типичны для второй половины 80-х годов, и не в последнюю очередь определяются размерами типового узла ALPS (рис. \ref{fig:MicrosoftDoveBarSize}).

\begin{figure}[h]
    \centering
    \includegraphics[scale=0.5]{1987_microsoft_dove_bar_mouse/size.jpg}
    \caption{Microsoft Dove Bar Mouse на размерном коврике с шагом сетки 1~см}
    \label{fig:MicrosoftDoveBarSize}
\end{figure}

Корпус Dove Bar Mouse является симметричным, за исключением кнопок, имеющих разный размер. Помимо того, что форма шлифовального бруска обеспечила достаточно удобное положение руки на корпусе (рис. \ref{fig:MicrosoftDoveBarHand}), нужно отметить также удобное положение пальцев на кнопках (очень большой главной и меньшей, но всё равно достаточно крупной правой). Ранее кнопки, вписанные в форму корпуса, пробовали использовать и другие компании: Logitech в мыши, выпущенной для Hewlett Packard в 1984 и Atari для мыши своих компьютеров 1985 года выпуска. Однако до появления Dove Bar Mouse такое решение было скорее исключением из правил, и кнопки всегда имели одинаковый размер: очевидно, производители мышей 80-х годов опасались, что пользователь не сможет уверенно определить границу между ними без взгляда на мышь. Опасалась этого также и Microsoft, поэтому между левой и правой кнопками можно заметить барьер, позволяющий различать их на ощупь. Также переключатели мембранного типа, которые использовались для кнопок в предыдущих моделях Microsoft, в данной модели они были заменены на микропереключатели, обладавшие меньшим ходом и лучшим откликом, чтобы минимизировать усилие нажатия \cite{doveBarMousePcMag3, doveBarDesign2}.

\begin{figure}[h]
    \centering
    \includegraphics[scale=0.45]{1987_microsoft_dove_bar_mouse/hand.jpg}
    \caption{Microsoft Dove Bar Mouse с моделью руки человека}
    \label{fig:MicrosoftDoveBarHand}
\end{figure}

Мышь выпускалась в модификациях: InPort Mouse с шинным интерфейсом (<<InPort>> "--- попытка Microsoft стандартизировать интерфейс подключения квадратурных мышей, соответсвующие адаптеры и переходники для них) и Serial Mouse с подключением к последовательному порту. Помимо подключения к специальной плате-адаптеру, смонтированной в системном блоке, для InPort Mouse выпускался также внешний переходник-конвертер в форме вытянутого параллелипипеда (он назывался Mouse Interface), позволявший подключать мышь к последовательному порту (рис. \ref{fig:MicrosoftDoveBarPic}). Кроме того, встречается более поздний вариант мыши с последовательным интерфейсом, названный Microsoft <<Serial -- PS/2 Compatible Mouse>> (название указывает на комплектный  конвертер RS-232 -- PS/2) \cite{doveBarDesign2}.


\begin{figure}[h]
    \centering
    \includegraphics[scale=0.6]{1987_microsoft_dove_bar_mouse/inside1_30.jpg}
    \caption{Microsoft Dove Bar Mouse в разобранном виде}
    \label{fig:MicrosoftDoveBarInside}
\end{figure}

Внутреннее устройство мыши образца 1987 года показано на рис. \ref{fig:MicrosoftDoveBarInside}. Компания Alps обычно предоставляла фирмам-заказчикам решения на основе своих типовых конструкций мышей. В частности, данная мышь конструктивно совпадает (включая узел преобразования движения на основе закрытых механических энкодеров и  массивные металлические ролики с подшипниками) с мышью IBM PS/2 mouse, появившейся на рынке в том же 1987 году. Однако, из-за размещения шара в передней части корпуса здесь наблюдается обратное расположение компонентов: механическая часть сдвинута вплотную к кнопкам, которые соединены с ней и с печатной платой гибким шлейфом, а сама печатная плата расположена в задней части мыши.

\begin{figure}[h]
    \centering
    \includegraphics[scale=0.6]{1987_microsoft_dove_bar_mouse/inside2_60.jpg}
    \caption{Оптомеханическая версия Microsoft Dove Bar Mouse в разобранном виде}
    \label{fig:MicrosoftDoveBarInside2}
\end{figure}

Вдохновленная успехом нового дизайна мыши,  в 1991 году Microsoft выпустила на рынок мышь в идентичном корпусе но с обновленной внутренней конструкцией (рис. \ref{fig:MicrosoftDoveBarInside2}), рекламировавшуюся как <<Contour Microsoft mouse>>. Производителем данного варианта Dove Bar Mouse выступила уже не компания ALPS, а Mitsumi. В мыши был использован более современный оптомеханический способ регистрации движения, экономная механическая конструкция на базе пластиковых роликов, а также характерный блестящий металлический диск оптического прерывателя (введенный в обиход компанией Depraz, и встречающийся также в ранних мышах Mitsumi), обеспечивавший, согласно рекламным материалам, разрешение 400 точек на дюйм \cite{doveBarMouseOldMouses}.

\begin{thebibliography}{9}
\bibitem{doveBarDesign1} Why Microsoft Resurrected A 15-Years-Old Mouse -- Fast Company. \url{https://www.fastcompany.com/90151927/why-we-still-love-using-mice#:~:text=By%20the%20time,the%20soap}
\bibitem{atkinson} Atkinson P. The best laid schemes o’ mice and men : the evolution of the computer mouse // Design and Evolution : Proceedings of Design History Society Conference 2006. Delft, Netherlands, Delft University of Technology, p. 1-20. \url{https://shura.shu.ac.uk/8659/}
\bibitem{doveBarMousePcMag1} The new Microsoft Mouse // PC Magazine, v. 7, No. 2, January, 1988, p. 310--311. \url{https://archive.org/details/PC-Mag-1988-01-26/page/n303/mode/2up}
\bibitem{doveBarMousePcMag2} Stanton T. Microsoft Mouse // PC Magazine, v. 8, No. 3, February 1989, p. 258. \url{https://archive.org/details/PC-Mag-1989-02-14/page/n257/mode/2up}
\bibitem{doveBarMousePcMag3} Stanton T. Microsoft Bus Mouse and Microsoft Serial Mouse // PC Magazine, v. 7, № 3, February 1988, p. 211--217. \url{https://archive.org/details/PC-Mag-1988-02-16/page/n209/mode/2up}
\bibitem{doveBarDesign2} Microsoft Mouse (3rd gen) - Deskthority wiki. \url{https://deskthority.net/wiki/Microsoft_mouse_(3rd_gen)}
\bibitem{doveBarMouseOldMouses} Microsoft ``Dove Bar'' Mouse  -- oldmouse.com  \url{https://web.archive.org/web/20210417224625/http://oldmouse.com/mouse/microsoft/dovebar.shtml}
\end{thebibliography}
\end{document}
