\documentclass[11pt, a4paper]{article}
\input{preamble.tex}
\switchlang{ru}
\begin{document}

\title{1987 "--- Microsoft Dove Bar Mouse}
\date{}
\maketitle
\selectlanguage{russian}

Данная мышь, появившаяся в продаже в 1987 году, стала третьим поколением мышей Microsoft. 




Производством мыши, как и в случае предыдущих поколений мышей Microsoft, занималась японская компания Alps, а дизайн был  разработан Microsoft в сотрудничестве с компанией Matrix Design (впоследствии объединившейся с Hovey-Kelley, разработавшей дизайн мыши Apple Lisa, в компанию IDEO)

\begin{figure}[h]
   \centering
    \includegraphics[scale=0.5]{1987_microsoft_dove_bar_mouse/pic_30.jpg}
    \caption{Microsoft Dove Bar Mouse}
    \label{fig:MicrosoftDoveBarPic}
\end{figure}

Третье поколение получило существенные улучшения как в конструкции, так и в плане эргономики. Отличительной особенностью мыши является смещенный вперед шар. Это сделано для того, чтобы датчик и центр тяжести мыши оказались в зоне расположения пальцев пользователя .

the form of this mouse
was closely based on a sanding block to get the hand feel right, and also included
major changes to the size and shape of the buttons. These became much larger,
stretching right across the surface of the front of the mouse, and were gently
indented. The left hand button was larger than the right, as this was the primary
button, with a small ridge added to its right hand edge to let users feel the boundary
between the buttons. 



корпус мыши практически лишен каких-либо выступающих элементов (присутствует только продолговатый выступ у внутреннего края главной кнопки, по замыслу разработчиков помогающий пользователю определить наощупь, где заканчивается большя по размеру левая кнопка и начинается меньшая правая.
     На нижней части мыши присутствуют низкофрикционные накладки, табличка с техническими данными и сдвижное кольцо-защелка, необходимое чтобы извлечь шар для чистки (рис. \ref{fig:MicrosoftDoveBarTopAndBottom}). 
     Сам шар смещен в переднюю часть мыши и находится практически в зоне расположения пальцев пользователя на корпусе. Такая модификация должна была ощутимо улучшить точность позиционирования курсора по сравнению с мышами предыдущих поколений \ref{atkinson}, и очевидно улучшение действительно имело место "--- по крайней мере по сравнению с более ранними конструкциями производства ALPS, в которых шар располагался у ближнего к пользователю края корпуса чтобы освободить больше места для печатной платы и кнопок.

В предыдущих моделях мышей Microsoft для кнопок использовались переключатели мембранного типа; в данной модели они были заменены на микропереключатели, обладавшие меньшим ходом и лучшим откликом, что минимизировало необходимость прикладывать усилия для выполнения нажатия \cite{doveBarDesign2}. Как отмечалось в \cite{doveBarMousePcMag3}, отклик кнопки четкий и прямой, сопровождаемый приглушенным, но заметным щелчком. 

\begin{figure}[h]
    \centering
    \includegraphics[scale=0.55]{1987_microsoft_dove_bar_mouse/top_30.jpg}
    \includegraphics[scale=0.55]{1987_microsoft_dove_bar_mouse/bottom_30.jpg}
    \caption{Microsoft Dove Bar Mouse, вид сверху и снизу}
    \label{fig:MicrosoftDoveBarTopAndBottom}
\end{figure}

Как можно заметить (рис. \ref{fig:MicrosoftDoveBarTopAndBottom}), корпус мыши практически лишен каких-либо выступающих элементов (присутствует только продолговатый выступ у внутреннего края главной кнопки, по замыслу разработчиков помогающий пользователю определить наощупь, где заканчивается левая кнопка и начинается правая. На нижней части мыши присутствуют низкофрикционные накладки, табличка с техническими данными и сдвижное кольцо-защелка, необходимое чтобы извлечь шар для чистки.

\begin{figure}[h]
    \centering
    \includegraphics[scale=0.5]{1987_microsoft_dove_bar_mouse/size.jpg}
    \caption{Microsoft Dove Bar Mouse на размерном коврике с шагом сетки 1~см}
    \label{fig:MicrosoftDoveBarSize}
\end{figure}

Размеры мыши являются типичными для 80-х годов (рис. \ref{fig:MicrosoftDoveBarSize}) и определяются размерами типового узла ALPS.

Кнопки имеют большую площадь, занимая переднюю треть корпуса, и полностью вписаны в его форму. Это не первая модель мыши с таким положением кнопок: аналогичный вариант пробовала Logitech в мыши, выпущенной для Hewlett Packard в 1984 и Atari для мыши одноименных компьютеров 1985 года выпуска; однако до появления Dove Bar Mouse такое решени было скорее исключением из правил. Очевидно, производители мышей 80-х годов опасались, что пользователь не сможет уверенно определить где находится какая кнопка; опасалась этого также и Microsoft, поэтому между левой и правой кнопками можно заметить барьер, позволяющий различать их на ощупь.

Благодаря кнопкм большой площади пользователь получал больше свободы в расположении пальцев на корпусе. 

\begin{figure}[h]
    \centering
    \includegraphics[scale=0.5]{1987_microsoft_dove_bar_mouse/hand.jpg}
    \caption{Microsoft Dove Bar Mouse с моделью руки человека}
    \label{fig:MicrosoftDoveBarHand}
\end{figure}

Мышь выпускалась в модификациях: InPort Mouse с шинным интерфейсом (<<InPort>> "--- попытка Microsoft стандартизировать интерфейс подключения квадратурных мышей, соответсвующие адаптеры и переходники для них) и Serial Mouse с подключением к последовательному порту. Помимо подключения к специальной плате-адаптеру, смонтированной в системном блоке, для InPort Mouse выпускался также внешний переходник-конвертер в форме вытянутого параллелипипеда (он назывался Mouse Interface), позволявший подключать мышь к последовательному порту (рис. \ref{fig:MicrosoftDoveBarPic}). Кроме того, встречается более поздний вариант мыши с последовательным интерфейсом, названный Microsoft <<Serial -- PS/2 Compatible Mouse>>, название которого указывает на комплектный  конвертер RS-232 -- PS/2.


\begin{figure}[h]
    \centering
    \includegraphics[scale=0.5]{1987_microsoft_dove_bar_mouse/inside1_30.jpg}
    \caption{Microsoft Dove Bar Mouse в разобранном виде}
    \label{fig:MicrosoftDoveBarInside}
\end{figure}

Внутреннее устройство мыши образца 1987 года показано на рис. \ref{fig:MicrosoftDoveBarInside}. Очевидно, реальным изготовителем мыши была японская компания Alps, часто предоставлявшая решения на основе своих типовых конструкций  мышей для других фирм. В частности, данная мышь конструктивно совпадает (включая узел преобразования движения на основе закрытых механических энкодеров и  массивные металлические ролики с подшипниками) с мышью IBM PS/2 mouse, появившейся на рынке в 1987 году. Однако, из-за смещения шара в переднюю часть корпуса, здесь наблюдается обратное расположение компонентов: механическая часть сдвинута вплотную к кнопкам, которые соединены с ней и с печатной платой гибким шлейфом, а сама печатная плата расположена в задней части мыши.

Вдохновленная успехом нового дизайна мыши, Microsoft в 1991 году выпустила на рынок мышь в идентичном корпусе но с обновленной конструкцией, рекламировавшуюся как <<Contour Microsoft mouse>>. Производителем данного варианта Dove Bar Mouse выступила уже не компания ALPS, а Mitsumi. В мыши используется оптомеханический способ регистрации движения с достаточно типичной конструкцией начала 90-х, экономная механическая конструкция на базе пластиковых роликов, а также металлический диск оптического прерывателя, обеспечивавший (согласно рекламным материалам) разрешение 400 точек на дюйм. Изображение поздней версии Microsoft Mouse в разобранном виде представлено на рисунке \ref{fig:MicrosoftDoveBarInside2}.

\begin{figure}[h]
    \centering
    \includegraphics[scale=0.5]{1987_microsoft_dove_bar_mouse/inside2_60.jpg}
    \caption{Оптомеханическая версия Microsoft Dove Bar Mouse в разобранном виде}
    \label{fig:MicrosoftDoveBarInside2}
\end{figure}



\begin{thebibliography}{9}
\bibitem{doveBarMouseOldMouses} Microsoft "Dove Bar" Mouse \url{https://web.archive.org/web/20210417224625/http://oldmouse.com/mouse/microsoft/dovebar.shtml}
\bibitem{atkinson} Atkinson P. The best laid schemes o’ mice and men : the evolution of the computer mouse // Design and Evolution : Proceedings of Design History Society Conference 2006. Delft, Netherlands, Delft University of Technology, p. 1-20. \url{https://shura.shu.ac.uk/8659/}
\bibitem{doveBarMousePcMag1} The new Microsoft Mouse // PC Magazine, v. 7, No. 2, January, 1988, p. 310--311. \url{https://archive.org/details/PC-Mag-1988-01-26/page/n303/mode/2up}
\bibitem{doveBarMousePcMag2} Stanton T. Microsoft Mouse // PC Magazine, v. 8, No. 3, February 1989, p. 258. \url{https://archive.org/details/PC-Mag-1989-02-14/page/n257/mode/2up}
\bibitem{doveBarMousePcMag3} Stanton T. Microsoft Bus Mouse and Microsoft Serial Mouse // PC Magazine, v. 7, № 3, February 1988, p. 211--217 [Электронный ресурс]. - Режим доступа: \url{https://archive.org/details/PC-Mag-1988-02-16/page/n209/mode/2up}
\bibitem{doveBarDesign1} Why Microsoft Resurrected A 15-Years-Old Mouse -- Fast Company. \url{https://www.fastcompany.com/90151927/why-we-still-love-using-mice#:~:text=By%20the%20time,the%20soap}
\bibitem{doveBarDesign2} Microsoft Mouse (3rd gen) - Deskthority wiki. \url{https://deskthority.net/wiki/Microsoft_mouse_(3rd_gen)}
\end{thebibliography}
\end{document}
