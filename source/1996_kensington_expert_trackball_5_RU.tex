\documentclass[11pt, a4paper]{article}
\input{preamble.tex}
\switchlang{ru}
\begin{document}

\title{1996 "--- Kensington Expert Mouse Trackball 5.0}
\date{}
\maketitle
\selectlanguage{russian}
В 1996 году Kensington выпустила пятую по счёту модель Expert Mouse Trackball. Трекбол претерпел существенный редизайн \cite{KensingtonPC}. В то время как предыдущие модели были достаточно классическими двухкнопочными устройствами, Expert Mouse Trackball 5.0 оказался оснащен шаром большего диаметра и четырьмя крупными кнопками, симметрично расположенными вокруг шара по аналогии с лепестками цветка (рис. \ref{fig:ExpertMousePic}).

\begin{figure}[h]
    \centering
    \includegraphics[scale=0.4]{1996_kensington_expert_trackball_5/pic_60.jpg}
    \caption{Kensington Expert Trackball}
    \label{fig:ExpertMousePic}
\end{figure}

Для Macintosh была выпущена аналогичная модель с предсказуемым названием Turbo Mouse 5.0 \cite{KensingtonMac} и интерфейсом ADB, в то время как Expert Mouse комплектовался сменными кабелями для подключения к последовательному интерфейсу и к порту PS/2 (также отдельно выпускалась шинная версия с ISA-адаптером). Визуально Turbo Mouse была идентична, воспроизводя тот же самый <<цветочный>> стиль (рис. \ref{fig:ExpertMouseTopBottom}).

\begin{figure}[h]
    \centering
    \includegraphics[scale=0.4]{1996_kensington_expert_trackball_5/top_30.jpg}
    \includegraphics[scale=0.4]{1996_kensington_expert_trackball_5/bottom_30.jpg}
    \caption{Kensington Expert Mouse Trackball вид сверху}
    \label{fig:ExpertMouseTopBottom}
\end{figure}

Несмотря на большой шар и крупные кнопки, Expert Mouse Trackball является не самым габаритным устройством (рис. \ref{fig:ExpertMouseSize}): фактически, полость для шара и кнопки занимают большую часть корпуса. Необычным является также то, что шар никак не закреплён и свободно лежит в сферической полости (поэтому, помимо флористических ассоциаций внешний вид трекбола также наводит на мысли о яйце в птичьем гнезде). Это значительно облегчает чистку устройства (но осложняет его переноску с места на место, поскольку  отклонение от горизонтального положения корпуса чревато падением шара).

\begin{figure}[h]
    \centering
    \includegraphics[scale=0.3]{1996_kensington_expert_trackball_5/size_30.jpg}
    \caption{Kensington Expert Mouse Trackball на размерном коврике с шагом сетки 1~см}
    \label{fig:ExpertMouseSize}
\end{figure}

Безусловно, шар такого диаметра рассчитан на прецизионное перемещение курсора, востребованное в САПР и графических редакторах (учитывая нестандартную дизайнерскую концепцию, можно предположить, что расчет делался в значительной степени на эту последнюю категорию). В плане эргономики необычное расположение кнопок оказывается достаточно удобным: при работе пользователь накрывает шар пальцами и, в зависимости от положения кисти, в состоянии дотянуться не перемещая руку либо до двух ближних, либо до одной ближней и двух дальних кнопок (рис. \ref{fig:ExpertMouseHand}). По умолчанию две ближние к пользователю (и имеющие наибольшую площадь) кнопки играют роль левой и правой кнопок мыши, что безусловно упрощает работу тем, кому дополнительные кнопки не требуются либо требуются редко. Положительной характеристикой эргономики устройства является и характерная для Kensington  симметричность корпуса, делающая его одинаково удобным как для левшей, так и для правшей.

\begin{figure}[h]
    \centering
    \includegraphics[scale=0.3]{1996_kensington_expert_trackball_5/hand_30.jpg}
    \caption{Изображение Kensington Expert Mouse Trackball с моделью руки человека}
    \label{fig:ExpertMouseHand}
\end{figure}

Изучение внутреннего устройства (рис. \ref{fig:ExpertMouseInside}) показывает, что устройство использует  оптомеханическое преобразование и высоконадежные металлические ролики. При этом оптомеханическое преобразование выполнено нестандартно, по запатентованной технологии <<оптического рычага>> (англ. optical levering) \cite{eu}: вместо используемой в большинстве устройств регистрирации света, прошедшего через прорези вращающегося диска, здесь регистрируется отражение от радиальных металлических <<ребер>>, расположенных на торце ролика "--- фактически, на боковой стороне подшипника, приводимого в движение вращением шара. Очевидно, плотная набивка ребер позволила разработчикам отказаться от диска, уменьшая тем самым габариты устройства без ущерба для его разрешающей способности. 

\begin{figure}[h]
    \centering
    \includegraphics[scale=0.65]{1996_kensington_expert_trackball_5/inside_60.jpg}
    \caption{Kensington Expert Mouse Trackball в разобранном виде}
    \label{fig:ExpertMouseInside}
\end{figure}

\begin{thebibliography}{9}

\bibitem {KensingtonPC} Kensington: Expert Mouse 5.0 "--- \url{https://web.archive.org/web/19970106170305/http://www.kensington.com/prod/mice/mice3b.html}
\bibitem {KensingtonMac} Kensington: Turbo Mouse 5.0 "--- \url{https://web.archive.org/web/19970106170317/http://www.kensington.com/prod/mice/mice3a.html}
\bibitem {eu} Kensington -- Trackballs.EU "--- \url{https://web.archive.org/web/20210423041707/https://forum.trackballs.eu/viewtopic.php?f=14&t=43}
\end{thebibliography}

\end{document}
