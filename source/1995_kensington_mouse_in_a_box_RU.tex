\documentclass[11pt, a4paper]{article}
\input{preamble.tex}
\switchlang{ru}
\begin{document}

\title{1995 "--- Kensington Mouse}
\date{}
\maketitle
\selectlanguage{russian}

Мышь Kensington Mouse, продававшаяся и рекламировавшаяся как <<Mouse in a Box>> из-за упаковки в форме куба, была выпущена  американской компанией Kensington Computer Products Group в составе линейки из нескольких похожих по форме мышей \cite{kensingtonfamily}. В 1995 году компания-производитель, к этому времени уже десять лет использовавшая слово <<mouse>> в названиях своих трекболов (Turbo Mouse и Expert Mouse), расширила линейку продукции устройствами, которые были бы мышами не только по названию. Как и трекболы компании, Kensington Mouse выпускалась в версиях для PC (рис. \ref{kensingtonmousepic}) и для компьютеров Apple с интерфейсом ADB \cite{mouseinabox, denver}.
   
    \begin{figure}[h]
        \centering
    \includegraphics[scale=0.7]{1995_kensington_mouse_in_a_box/pic_30.jpg}
        \caption{Внешний вид мыши Kensington Mouse}
        \label{kensingtonmousepic}
    \end{figure}

Дизайн мыши следует концепции минимализма и активно использует геометрическую абстрацию. Корпус мыши выполнен из белого пластика и имеет в основании овоид (симметричен относительно продольной оси и слегка заужен к передней, дальней от пользователя, стороне). На передней части корпуса присутствует пара кнопок (у версии мыши с интерфейсом ADB они обычно объединены в одну, однако компания активно экспериментировала с этими моделями, поэтому двухкнопочные экземпляры с интерфейсом ADB также встречаются \cite{lowendmac}). Кнопки симметричны, интегрированы в форму корпуса, занимают треть длины мыши, разделены визуально различимым  зазором, и так же отделены дугообразным зазором от остальной части, на которой нет никаких элементов кроме логотипа компании (рис. \ref{kensingtonmousetopbottom}). Логотип выполнен в тёмно-сером цвете, совпадающем с цветом кабеля и плоской шайбоподобной муфты, защищающей его от повреждения в месте выхода из корпуса. На нижней стороне мыши можно найти ярлык с техническими данными, пять круглых накладок из низкофрикционного материала, которые обеспечивают плавное скольжение мыши по поверхности стола, а также поворотное кольцо-защелку для извлечения шара и чистки мыши.

Изучение кода FCC ID по базе данных Федеральной комиссии по связи США позволяет датировать мышь 1995 годом.

    \begin{figure}[h]
        \centering
    \includegraphics[scale=0.75]{1995_kensington_mouse_in_a_box/top_15.jpg}
    \includegraphics[scale=0.75]{1995_kensington_mouse_in_a_box/bottom_15.jpg}
        \caption{Изображение Kensington Mouse вид сверху и снизу}
        \label{kensingtonmousetopbottom}
        \end{figure}

Мышь имеет средние размеры, шар расположен строго в центре корпуса и имеет достаточно небольшой диаметр (очевидно, его размеры ограничены высотой корпуса, также достаточно небольшой, как видно из рис. \ref{kensingtonmousesize}). Ее разработчики тщательно придерживались принципов математического дизайна, продекларированного также и на некоторых вариантах упаковки мыши, где можно видеть рисунок корпуса, снабженный дополнительными линиями построения, показывающими лежащие в его основе окружности и дуги \cite{lowendmac}.

     \begin{figure}[h]
        \centering
    \includegraphics[scale=0.45]{1995_kensington_mouse_in_a_box/size_15.jpg}
        \caption{Изображение Kensington Mouse на размерном коврике с шагом сетки 1 см}
        \label{kensingtonmousesize}
    \end{figure}


Благодаря симметричному корпусу, мышь Kensington Mouse одинаково подходит для левшей и для правшей (рис. \ref{kensingtonmousehand}). Согласно рекламе компании мышь подходит для ладоней любого размера, потому что <<лучше соответствует естественной форме человеческой руки>> \cite{mouseinabox}. Действительно, положение ладони на корпусе мыши выглядит и ощущается очень естественным (рис. \ref{kensingtonmousehand}).

Благодаря идущему в комплекте фирменному переходнику c брэндом Kensington, мышь могла подключаться как к последовательному порту (бывшему для неё <<родным>> интерфейсом), так и к порту PS/2.

Внутри Kensington Mouse можно увидеть классическую оптомеханическую систему, характерную для мышей первой половины 90-х годов (рис. \ref{kensingtonmouseinside}), с той разницей, что все электронные компоненты находятся на обратной стороне платы. Как можно заметить, надпись на плате сообщает, что мышь изготавливалась Mitsumi Electric, часто выступавшей в роли контрактного разработчика мышей для других компаний.

    \begin{figure}[h!]
        \centering
    \includegraphics[scale=0.65]{1995_kensington_mouse_in_a_box/hand_30.jpg}
        \caption{Изображение Kensington Mouse с муляжом руки}
        \label{kensingtonmousehand}
    \end{figure}

    \begin{figure}[h!]
        \centering
    \includegraphics[width=\textwidth]{1995_kensington_mouse_in_a_box/inside_30.jpg}
        \caption{Изображение Kensington Mouse в разобранном виде}
        \label{kensingtonmouseinside}
    \end{figure}

\begin{thebibliography}{9}
    \bibitem{kensingtonfamily} The first family in mice // PC Magazine, February 10, 1998 -- P. 270 \url{https://books.google.by/books?id=fFrjSBw0w14C&lpg=PA270&dq=kensington%20mouse%20in%20a%20box&hl=ru&pg=PA270#v=onepage&q&f=false}
    \bibitem {mouseinabox} Kensington: Mouse in a Box -- kensington.com. January 06, 1997 \url{https://web.archive.org/web/19970106170908/http://www.kensington.com/prod/mice/mice3d.html}
    \bibitem {denver} Kensinton 1 button ADB mouse -- Apple rescue of Denver \url{https://applerescueofdenver.com/products-page/macintosh-to-powerpc/keyboards-mice-joysticks-macintosh/kensinton-1-button-adb-mouse/} 
    \bibitem {lowendmac} Knight D. 2-Button Kensington Mouse ADB -- Low End Mac \url{https://lowendmac.com/1998/2-button-kensington-mouse-adb/}
\end{thebibliography}

\end{document}
