\documentclass[11pt, a4paper]{article}
\input{preamble.tex}
\switchlang{ru}
\begin{document}

\title{1995 "--- Kensington Mouse in a box}
\date{}
\maketitle
\selectlanguage{russian}

Мышь Kensington Mouse была выпущена компанией Kensington Computer Products Group в составе линейки из нескольких похожих по форме мышей. Kensington Computer Products Group "--- это американская компания, выпускающая периферийные устройства и аксессуары для персональных компьютеров, такие как док-станции, мыши и названный в честь самой компании защитный трос с замком Kensington Lock. Штаб-квартира Kensington находится в Бурлингейме, штат Калифорния, и является подразделением ACCO Brands.

    Компания была основана Филипом Дамиано как Kensington Microware в 1981 году, и их первый продукт назывался System Saver: это был аксессуар для компьютеров Apple IIe, который добавлял охлаждающий вентилятор и защиту от перенапряжения. В 1986 году компания была приобретена ACCO Brands. Первым средством управления курсором от Kensington стал трекбол Turbo Mouse для Macintosh, выпущенный в 1986 году, и с тех пор трекболы являются одним из основных предложений компании. Первый замок физической защиты для ноутбука Kensington Lock был выпущен в 1992 году. Kensington также выпускала аксессуары для портативных устройств, таких как iPod, в том числе Stereo Dock в 2005 году.

    В настоящее время компания Kensington в основном продает компьютерные аксессуары, такие как мыши и клавиатуры, а также блоки питания и противоугонные системы для компьютеров и электронных устройств. Они также выпускают чехлы для ноутбуков и другой потребительской электроники, аксессуары для стационарных и мобильных рабочих станций.
    Компания продолжает предлагать Kensington Lock, систему защиты от кражи для ноутбуков и других периферийных устройств, а также ряд индивидуальных замков с различными опциями для защиты компьютерных устройств, таких как ноутбуки, настольные компьютеры, проекторы, TFT-дисплеи и внешние жесткие диски. 
    
    В 1995 году линейка продуктов компании пополнилось семейством мышей (до того слово <<mouse>> в названиях своей продукции Kensington использовала для обозначения трекболов). Мышь Kensington mouse, показанная на рисунке \ref{kensingtonmousepic}, является представителем этого семейства и предназначена для IBM-совместимых компьютеров.
   
    \begin{figure}[h]
        \centering
    \includegraphics[scale=0.5]{1995_kensington_mouse_in_a_box/pic_30.jpg}
        \caption{Внешний вид мыши Kensington Mouse }
        \label{kensingtonmousepic}
    \end{figure}

    Дизайн мыши следует концепции минимализма. Корпус имеет овальную форму, симметричен относительно продольной оси, слегка заужен с передней (дальней от пользователя) стороне. На передней части корпуса присутствует пара кнопок.  Кнопки вписаны в форму корпуса, занимают треть длины мыши и отделены от остальной части и друг от друга визуально различимым  зазором, за которым присутствует логотип компании (рис. \ref{kensingtonmousetopbottom}).  На нижней стороне мыши можно найти логотип компании Kensington и информацию о модели и сертификации. Изучение кода FCC ID по базе данных Федеральной комиссии по связи США позволяет датировать мышь 1995 годом.
    Кроме того, на нижней части корпуса располагаются 5 круглых накладок из низкофрикционного материала, которые обеспечивают плавное скольжение мыши по поверхности стола.  поворотное кольцо-защелку, которое можно снять для извлечения шара и чистки мыши. Шар расположен в центре мыши и имеет достаточно небольшой диаметр (очевидно, его размеры ограничены высотой корпуса, также достаточно небольшой, как видно из рис. \ref{kensingtonmousepic}). 
    
    \begin{figure}[h]
        \centering
    \includegraphics[scale=0.6]{1995_kensington_mouse_in_a_box/top_15.jpg}
    \includegraphics[scale=0.6]{1995_kensington_mouse_in_a_box/bottom_15.jpg}
        \caption{Изображение Kensington Mouse вид сверху и снизу}
        \label{kensingtonmousetopbottom}
        \end{figure}


     \begin{figure}[h]
        \centering
    \includegraphics[scale=0.3]{1995_kensington_mouse_in_a_box/size_15.jpg}
        \caption{Изображение Kensington Mouse на размерном коврике с шагом сетки 1 см}
        \label{thinkingmousesize}
    \end{figure}


 Благодаря симметричному корпусу, мышь Kensington Mouse одинаково подходит для левшей и для правшей (рис. \ref{kensingtonmousehand}). Согласно рекламе компании мышь подходит для ладоней любого размера, потому что <<лучше соответствует естественной форме человеческой руки>> \cite{kensingtonmouse}.

     \begin{figure}[h]
        \centering
    \includegraphics[scale=0.5]{1995_kensington_mouse_in_a_box/hand_30.jpg}
        \caption{Изображение Kensington Mouse с муляжом руки на размерном коврике
с шагом сетки 1 см}
        \label{kensingtonmousehand}
    \end{figure}   
    
    Благодаря идущему в комплекте фирменному переходнику c брэндом Kensington, мышь могла подключаться как к последовательному порту (бывшему для неё <<родным>> интерфейсом), так и к порту PS/2. В комплекте с мышью поставлялось дополнительное программное обеспечение, в первую очередь "--- настраиваемое меню быстрого запуска.

    Внутри Kensington Mouse можно увидеть классическую оптомеханическую систему, характерную для мышей первой половины 90-х годов (рис. \ref{kensingtonmouseinside}), с той разницей, что все электронные компоненты находятся на обратной стороне платы. Как можно заметить, надпись на плате говорит, что мышь изготавливалась Mitsumi Electric. Mitsumi Electric "--- японский производитель потребительских электронных компонентов, основанный в 1954 году. Mitsumi известна прежде всего как OEM-производитель компьютерной периферии и устройств ввода, дисководов для гибких и оптических дисков, используемых в портативных компьютерах, настольных компьютерах, серверах и дисковой системе Famicom. Также компания часто выступала в роли контрактного разработчика мышей для других компаний.
   
    \begin{figure}[h]
        \centering
    \includegraphics[width=\textwidth]{1995_kensington_mouse_in_a_box/inside_30.jpg}
        \caption{Изображение Kensington Mouse в разобранном виде}
        \label{kensingtonmouseinside}
    \end{figure}

\begin{thebibliography}{9}

\end{thebibliography}

\end{document}
