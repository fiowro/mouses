\documentclass[11pt, a4paper]{article}
\input{preamble.tex}

\begin{document}

\title{1986 "--- Honeywell microLYNX trackball}
\date{}
\maketitle

Трекбол microLYNX, показанный на рис. \ref{fig:microLYNXPic}, выпускался в калифорнии компанией Honeywell "--- дочерним предприятием Disc Instruments.

%В плане технической реализации трекбол Honeywell оснащен механическими
%датчиками положения, то есть не использует наиболее часто встречающуюся в подобных устройствах оптомеханическую схему.

\begin{figure}[h]
    \centering
    \includegraphics[scale=0.4]{1986_honeywell_microlynx_trackball/pic_60.jpg}
    \caption{Внешний вид трекбола microLYNX}
    \label{fig:microLYNXPic}
\end{figure}

Устройство оснащено большим черным шаром в бежевом корпусе с вытянутой подставкой
под запястье \ref{fig:microLYNXTopBottom}. Перед шаром расположены три кнопки по форме соответствующие классической полноразмерной клавиатуре. Шар плотно прилегает к краям отверстия в корпусе, что неплохо защищает от попадания мусора внутрь трекбола, но делает извлечение шара для чистки невозможным без разборки корпуса.

\begin{figure}[h]
    \centering
    \includegraphics[scale=0.4]{1986_honeywell_microlynx_trackball/top_30.jpg}
    \includegraphics[scale=0.4]{1986_honeywell_microlynx_trackball/bottom_30.jpg}
    \caption{microLYNX, вид сверху и снизу}
    \label{fig:microLYNXTopBottom}
\end{figure}

Трекбол выпускался в нескольких модификациях, различающихся интерфейсом подключения. Модель comLYNX отличалась названием и тем, что подключалась к последовательному порту, а microLYNX использовал в качестве интерфейса подключения порт клавиатуры, фактически включаясь в разрыв клавиатурного кабеля.

При первом включении питания microLYNX переводится в <<текстовый режим>>, при котором поворот шара производит тот же эффект, что и нажатие клавиш курсора на клавиатуре: при движении шара трекбол генерирует и посылает в компьютер скан-коды курсорных клавиш. Кнопки трекбола
полностью программируются, и пользователь может настроить их на выдачу до 30 символов каждая в виде макросов. Также можно настроить
скорость перемещения курсора (частоту генерирования кодов курсорных клавиш) в текстовом режиме.

Для графических программ драйвер, которым комплектовался microLYNX, может эмулировать мышь Microsoft. В этом случае пользователь может использовать трекбол для перемещения курсора и перетаскивания при выделении текста или перемещении графического элемента. Для перетаскивания могут использоваться две внешние кнопки, расположенные перед шаром.

\begin{figure}[h]
    \centering
    \includegraphics[scale=0.3]{1986_honeywell_microlynx_trackball/size.jpg}
    \caption{Трекбол microLYNX на размерном коврике с шагом сетки 1 см}
    \label{fig:microLYNXSize}
\end{figure}

Однако, учитывая размеры устройства (рис. \ref{fig:microLYNXSize}), удерживать нажатой кнопку, пока происходит
перемещение шара "--- это в лучшем случае сложный маневр. Honeywell решила
эту проблему на родственном данному трекболу устройстве comLYNX, используя
среднюю кнопку в качестве <<защелки>> перетаскивания. Сначала выполняется нажатие средней кнопки, затем левой либо правой, и это
программно фиксирует выбранную кнопку в нажатом положении. Нажатие любой из трех кнопок повторно
отключает данный режим. Однако в microLYNX эта функция не поддерживается.

Оба трекбола поставляются с двумя наборами программ: драйвером для DOS, эмулирующим мышь Microsoft, и отдельной резидентной программой для DOS, которая показывает всплывающее меню, а также позволяет запрограммировать клавиатурные макросы для кнопок трекбола и перемещения курсора.

\begin{figure}[h]
    \centering
    \includegraphics[scale=0.4]{1986_honeywell_microlynx_trackball/hand_60.jpg}
    \caption{Трекбол microLYNX в комплекте с моделью руки человека}
    \label{fig:microLYNXHand}
\end{figure}

В отношении эргономики размеры трекбола, подставка под запястье, а также скругленные грани и углы создают достаточно комфортные условия для работы. Однако кнопки расположены достаточно далеко от шара, и существенно ниже по высоте, что лишает пользователя как возможности нажимать их одной рукой без перемещения кисти, так и двумя руками, поскольку при вращении шара кнопки закрыты ладонью (рис. \ref{fig:microLYNXHand}). Поэтому microLYNX трудно использовать в условиях, требующих интенсивной работы с постоянным чередованием перемещений курсора и нажатием кнопок.

\begin{figure}[h]
    \centering
    \includegraphics[scale=0.5]{1986_honeywell_microlynx_trackball/inside_60.jpg}
    \caption{microLYNX в разобранном виде}
    \label{fig:microLYNXInside}
\end{figure}

Внутреннее устройство данного трекбола показано на рис. \ref{fig:microLYNXInside}, что позволяет классифицировать его как оптомеханическое устройство. Также следует отметить, что ролики реализованы с использованием подшипников и валов из нержавеющей стали, что обеспечивает максимальную надежность и долговечность конструкции. Высокая надежность трекбола и эргономика, хорошо подходящая для ряда технических задач, сделали данную модель <<долгожителем>>: устройство не меньше десяти лет выпускалось для индустриальных применений под различными марками, полностью сохранив внешний вид и механическую часть конструкции.
 
\begin{thebibliography}{9}
\bibitem {mouses} Trackballs: Stationary mice // PC Magazine. August 1987, page 199-202
\end{thebibliography}
\end{document}
