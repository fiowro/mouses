\documentclass[11pt, a4paper]{article}
\input{preamble.tex}
\switchlang{ru}
\begin{document}

\title{1993 "--- Evergreen Diamond XL trackball}
\date{}
\maketitle
\selectlanguage{russian}
Трекбол Diamond XL выпущен калифорнийской компанией Evergreen Systems International, основанной в 1980 году и специализировавшейся на высококачественных трекболах для промышленного и военного применения.

Разработчик позиционировал трекболы Diamond как замену мыши в компьютерных системах, где требуются быстрые и точные функции наведения (CAD/CAM, графика, рабочие станции, киоски и т.д.), а также в промышленных и военных системах, где мышь не подходит для использования из-за нехватки места, суровых условий эксплуатации, среды или других факторов.

Первые упоминания данной модели в рекламных материалах датируются 1993 годом \cite{nasa}.

\begin{figure}[h]
    \centering
    \includegraphics[scale=0.3]{1993_evergreen_diamond_xl_trackball/pic_30.jpg}
    \caption{Трекбол Diamond XL}
    \label{fig:DiamondXL}
\end{figure}

На рис. \ref{fig:DiamondXLTopBottom} можно видеть верхнюю и нижнюю стороны трекбола.
На верхней части корпуса отсутствуют какие-либо надписи; корпус имеет гладкую матовую поверхность и ребристые боковые клавиши для их более легкой тактильной идентификации. На нижней части корпуса присутствуют резиновые ножки, обеспечивающие надежную фиксацию на поверхности стола, маркировка производителя и модели.

\begin{figure}[h]
    \centering
    \includegraphics[scale=0.35]{1993_evergreen_diamond_xl_trackball/top_30.jpg}
    \includegraphics[scale=0.35]{1993_evergreen_diamond_xl_trackball/bottom_30.jpg}
    \caption{Diamond XL, вид сверху и снизу}
     \label{fig:DiamondXLTopBottom}
\end{figure}

Данное устройство является весьма габаритным (рис. \ref{fig:DiamondXLSize}). Диаметр шара составляет 51 мм (2 дюйма).

\begin{figure}[h]
    \centering
    \includegraphics[scale=0.35]{1993_evergreen_diamond_xl_trackball/size.jpg}
    \caption{Изображение Diamond XL на размерном коврике с шагом сетки 1~см}
    \label{fig:DiamondXLSize}
\end{figure}

Трекбол симметричен и одинаково удобен при использовании как правой так и левой рукой. Основная (левая) кнопка мыши расположена под большим пальцем для удобства выбора и перетаскивания объектов (рис. \ref{fig:DiamondXLHand}), при этом остальные пальцы остаются свободными для позиционирования курсора. Роль средней кнопки играет пара наклонных клавиш, расположенных за шаром (в данной версии трекбола они имеют наклон к центру, а в упоминавшейся на сайте производителя пятикнопочной версии \cite{evergreen} за шаром расположен блок из трех таких клавиш, повернутых на 90 градусов и имеющих наклон к краю корпуса). Правая кнопка симметрична левой и нажимается мизинцем. В некоторых модификациях правостороннее или левостороннее управление выбирается с помощью переключателей конфигурации, расположенных в вырезе в нижней части корпуса, однако в данном экземпляре оно реализуется лишь на уровне драйвера.

Трекбол выпускался в модификациях c различными интерфейсами, что позволяло использовать их с компьютерами SUN, DEC, Hewlett Packard (шина HP/HIL), IBM, SGI и Macintosh с шиной USB. В частности, трекбол DTXL3 с шиной HP/HIL производился по лицензии Hewlett Packard, и позиционировался как прямая замена мыши HP/HIL, а также снятого с производства собственного трекбола Hewlett Packard HP M1309A \cite{dtxl3, hphil}. Следы такой ориентации Diamond XL можно заметить в дизайне корпуса: дальняя от пользователя стенка включает две технологические площадки, предназначенные для установки гнезд подключения к шине HP/HIL (рис. \ref{fig:DiamondXLHand}).

\begin{figure}[h]
    \centering
    \includegraphics[scale=0.4]{1993_evergreen_diamond_xl_trackball/hand_30.jpg}
    \caption{Diamond XL с моделью руки человека}
    \label{fig:DiamondXLHand}
\end{figure}

Внутреннее устройство данного трекбола показано на рис. \ref{fig:DiamondXLInside}. Как можно видеть,  трекбол выполнен по традиционной оптомеханической схеме. Также следует отметить, что ролики реализованы с использованием подшипников и металлических осей, предназначенных для того, чтобы обеспечить максимальную надежность и долговечность конструкции. На это делается упор и в описании производителя, где среди особенностей устройства подчеркивается полностью металлические механические части, валы повышенной твердости из азотированной стали с прецизионными шарикоподшипниками и переключатели с позолоченными пружинными контактами \cite{dtxl3}.

Извлечение шара для чистки трекбола невозможно без разборки корпуса.

\begin{figure}[h]
    \centering
    \includegraphics[scale=0.4]{1993_evergreen_diamond_xl_trackball/inside_30.jpg}
    \caption{Diamond XL в разобранном состоянии}
    \label{fig:DiamondXLInside}
\end{figure}

\begin{thebibliography}{9}
\bibitem{evergreen} Evergreen Systems International home page. \url{https://web.archive.org/web/19970102174426/http://trackballs.com:80/}
\bibitem{dtxl3} Keyton Computer -- Trackball \url{http://keyton.co.jp/products/UEVE/DTXL3.html}
\bibitem{hphil} Diamond XL HP-HIL trackball \url{https://web.archive.org/web/19970328230321/http://www.trackballs.com/xlhil.htm}
\bibitem{nasa} NASA Tech Briefs, November 1993. Volume 17, No. 11. -- p. 60 \url{https://archive.org/details/NASA_NTRS_Archive_20100030364/page/n59/mode/2up}
\end{thebibliography}
\end{document}
