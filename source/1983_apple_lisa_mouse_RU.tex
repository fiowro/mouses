\documentclass[11pt, a4paper]{article}
\input{preamble.tex}
\switchlang{ru}
\begin{document}

\title{1983 "--- Apple Lisa mouse}
\date{}
\maketitle
\selectlanguage{russian}
Мышь для компьютеров Apple Lisa была выпущена в 1983 году \cite{mouses} и считается одной из первых компьютерных мышей, доступных в свободной продаже (рис. \ref{fig:AppleLisaPic}).

\begin{figure}[h]
   \centering
    \includegraphics[scale=0.5]{1983_apple_lisa_mouse/applenorm_30.jpg}
    \caption{Apple Lisa mouse, вид спереди}
    \label{fig:AppleLisaPic}
\end{figure}

Мышь производилась Apple, но ее дизайном занималась сторонняя компания Hovey-Kelley (позже переименованная в IDEO). Отталкиваясь от дизайна мышей Xerox и Hawley Mouse House, команда разработчиков создала более дешевую и при этом лучшую в техническом плане механическую часть, и разработала сложную внутреннюю конструкцию корпуса, удерживающую ее части вместе. Большое внимание было уделено и другим ключевым компонентам мыши, вплоть до акустических и тактильных особенностей нажатия кнопки и прорезиненного покрытия на шарике \cite{ideo}.

\begin{figure}[h]
    \centering
    \includegraphics[scale=0.5]{1983_apple_lisa_mouse/appletop_60.jpg}
    \includegraphics[scale=0.5]{1983_apple_lisa_mouse/applebottom_60.jpg}
    \caption{Apple Lisa mouse, вид сверху и снизу}
    \label{fig:AppleLisaTopAndBottom}
\end{figure}

Корпус мыши Lisa представляет собой наклонную коробку бежевого цвета. Нижняя стенка корпуса мыши, кабель и кнопка окрашены в коричневый цвет. На корпусе присутствует рельефный логотип Apple (рис. \ref{fig:AppleLisaTopAndBottom}). Шар выполнен из металла и имеет резиновое покрытие для лучшего сцепления с поверхностью, в отличие от более ранних мышей, имевших либо резиновый либо гладкий стальной шар. В последствии подобная реализация шара, а также съёмное кольцо, позволяющее легко извлечь шар для удаления собравшегося мусора, станут стандартом. То же самое можно сказать и про базовую конструкцию ее механизма, которая в дальнейшем использовалась в абсолютном большинстве оптомеханических мышах.

\begin{figure}[h]
    \centering
    \includegraphics[scale=0.4]{1983_apple_lisa_mouse/applekovrik_60.jpg}
    \caption{Apple Lisa I на размерном коврике с шагом сетки 1~см}
    \label{fig:AppleLisaSize}
\end{figure}

Несмотря на то, что мышь имеет сравнительно небольшие размеры, впрочем, характерные для мышей 1980-х годов (рис. \ref{fig:AppleLisaSize}), исследования ее разработчиков не прошли даром, и она достаточно удобно лежит в руке. При этом единтвенная кнопка расположена ортогонально кабелю и рассчитана на нажатие скорее двумя пальцами, чем одним (рис. \ref{fig:AppleLisaHand}), а тактильность нажатия обеспечивается специально предусмотренным пружинным механизмом.

\begin{figure}[h]
    \centering
    \includegraphics[scale=0.4]{1983_apple_lisa_mouse/appleruka_60.jpg}
    \caption{Apple Lisa I с моделью руки человека}
    \label{fig:AppleLisaHand}
\end{figure}

Внутреннее устройство мыши показано на рис. \ref{fig:AppleLisaInside}. Можно отметить достаточно сложную внутреннюю конструкцию корпуса со значительным числом перегородок и дополнительных опор. Унаследовав конструкцию оптомеханическуого энкодера практически без изменений, производители последующих мышей постепенно все более упрощали внутреннюю конструкцию корпуса, достигнув к девяностым годам практически пустотелой конструкции устройства.

 \begin{figure}[h]
    \centering
    \includegraphics[scale=0.6]{1983_apple_lisa_mouse/appleraz_60.jpg}
    \caption{Apple Lisa I в разобранном виде}
    \label{fig:AppleLisaInside}
\end{figure}

\begin{thebibliography}{9}
\bibitem {mouses} Apple Lisa I Mouse \url{https://web.archive.org/web/20211019174310/https://www.oldmouse.com/mouse/apple/lisa.shtml}

\bibitem {ideo} Creating the First Usable Mouse \url{https://www.ideo.com/case-study/creating-the-first-usable-mouse}
\end{thebibliography}
\end{document}
