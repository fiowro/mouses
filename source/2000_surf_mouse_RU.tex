\documentclass[11pt, a4paper]{article}
\input{preamble.tex}
\switchlang{ru}
\begin{document}

\title{2000 "--- мышь SurfMouse}
\date{}
\maketitle
\selectlanguage{russian}
Мышь SurfMouse является характерным примером скевоморфного манипулятора "--- такого, который имитирует внешний вид другого объекта, никак не связанного с его основной функцией. Несложно заметить, что в данном случае копируется внешний вид доски для серфинга (рис. \ref{fig:SurfMousePic}).

\begin{figure}[h]
    \centering
    \includegraphics[scale=0.4]{2000_surf_mouse/pic_60.jpg}
    \caption{Мышь SurfMouse}
    \label{fig:SurfMousePic}
\end{figure}

Мышь имеет симметричный корпус, верхней части которого придана форма серфборда. Из-за этого она существенно выдаётся вперед и назад, нависая над основной частью, сохраняющей форму типичной оптомеханической мыши (рис. \ref{fig:SurfMouseTopBottom}). На верхней стороне корпуса присутствуют элементы орнамента с лого SurfMouse, а также две кнопки большого размера. Кнопки расположены ближе к центральной части, чем к передней части корпуса (на самом деле они занимают обычное для мышей положение, а их видимое смещение к центру вызвано исключительно выдающимся вперед носом серфборда).

\begin{figure}[h]
    \centering
    \includegraphics[scale=0.46]{2000_surf_mouse/top_60.jpg}
    \includegraphics[scale=0.46]{2000_surf_mouse/bottom_60.jpg}
    \caption{SurfMouse, вид сверху и снизу}
    \label{fig:SurfMouseTopBottom}
\end{figure}

Перевернув мышь, можно увидеть традиционные для мышей 90-х годов кольцо-защелку, закрывающее обрезиненный шарик, ножки для скольжения по поверхности, выполненные из низкофрикционного полимера, а также наклейку с маркировкой.

Из-за необходимости копировать форму доски, мышь оказывается крупнее типичных манипуляторов (рис. \ref{fig:SurfMouseSize}) за счет удлиненных передней и задней части.

\begin{figure}[h]
    \centering
    \includegraphics[scale=0.5]{2000_surf_mouse/size_30.jpg}
    \caption{Изображение SurfMouse на размерном коврике с шагом сетки 1~см}
    \label{fig:SurfMouseSize}
\end{figure}

Благодаря симметричности мышь одинаково подходит как для левшей, так и для правшей. Нельзя не заметить, что в целом рука лежит на мыши достаточно удобно (рис. \ref{fig:SurfMouseHand}). Ближняя к пользователю часть <<серфборда>> загнута вниз, образуя удобную горизонтальную подставку под запястье, а расположение кнопок и их форму также можно назвать достаточно удачным.

\begin{figure}[h]
    \centering
    \includegraphics[scale=0.5]{2000_surf_mouse/hand_30.jpg}
    \caption{Изображение SurfMouse с моделью руки человека}
    \label{fig:SurfMouseHand}
\end{figure}

Очевидно, разработчики полагались на хорошую эргономику за счет удобного положения руки на серфборде (специально для этой мыши разработчиками был создан отдельный веб-сайт \url{surfmouse.com} \cite{site}). Однако отсутствие колеса прокрутки, никак не вписывавшегося в стилистику доски для серфинга, в 2000 году уже было критичным недостатком в эргономике мыши.

\begin{figure}[h]
    \centering
    \includegraphics[scale=0.6]{2000_surf_mouse/inside_60.jpg}
    \caption{SurfMouse в разобранном состоянии}
    \label{fig:SurfMouseInside}
\end{figure}

Внутреннее устройство данного манипулятора, показанное на рис. \ref{fig:SurfMouseInside}, открывает типичную оптомеханическую конструкцию первой половины или середины девяностых годов. Более того, представляется вполне вероятным существование в 90-х годах мыши-донора, нижняя часть и наполнение которой послужили основой для SurfMouse.

\begin{thebibliography}{9}
    \bibitem {site} Welcome to SurfMouse.com -- Surf the Web in Style! \url{https://web.archive.org/web/20001204212200/http://www.surfmouse.com/}
\end{thebibliography}

\end{document}
