\documentclass[11pt, a4paper]{article}
\input{preamble.tex}

\begin{document}

\title{1997 "--- Fellowes Sphere Trackball}
\date{}
\maketitle
Fellowes Sphere Trackball — типичный представитель данного типа указательных устройств ввода информации для компьютера, наиболее характерных для первой половины 90-х годов (хотя и выпущен компанией Fellowes Computerware в 1997). Поскольку трекбол аналогичен мыши по принципу действия и по функциям, но появился раньше, чем мышь, широко распространено мнение, что мышь была изобретена путем переворачивания трекбола вверх дном и его перемещения по поверхности стола.

\begin{figure}[h]
    \centering
    \includegraphics[scale=0.2]{1997_fellowes_trackball/fellowes.jpg}
    \caption{Fellowes Trackball}
    \label{fig:pic}
\end{figure}

Конструктивно трекбол также похож на мышь: вращение шарика приводит в движение пару валиков, соединённых с механическими датчиками, либо, в варианте, появившемся позднее данного, движения шара сканируют размещённые в корпусе оптические датчики. 

\begin{figure}[h]
    \centering
    \includegraphics[scale=0.3]{1997_fellowes_trackball/fellowes2.jpg}
    \caption{Fellowes Trackball в разобранном виде}
    \label{fig:inside}
\end{figure}

Протокол обмена данными между трекболом и компьютером, как правило, также полностью соответствует протоколу для мыши. Поэтому с точки зрения компьютера трекбол является стандартным интерфейсным указательным устройством; в отсутствие специальных драйверов он воспринимаются операционной системой компьютера как стандартная мышь и нормально поддерживаются стандартным универсальным драйвером мыши. 

\begin{figure}[h]
    \centering
    \includegraphics[scale=0.3]{1997_fellowes_trackball/fellowsup.JPG}
    \includegraphics[scale=0.3]{1997_fellowes_trackball/fellowsdown.JPG}
    \caption{Fellowes Trackball, вид сверху и снизу}
    \label{fig:top}
\end{figure}

При работе с трекболом для операций перемещения курсора используется только кисть руки и движения пальцев. Поэтому от пользователя не требуется движений плеча и предплечья, тогда как те же операции с мышью требуют задействования практически всей руки. 
  Поэтому трекбол часто рекомендуется пользователям, испытывающим временные или постоянные проблемы, связанные сплечевым поясом или запястьем.

\begin{figure}[h]
    \centering
    \includegraphics[scale=0.3]{1997_fellowes_trackball/fellowset2.jpg}
    \caption{Fellowes Trackball с моделью руки человека}
    \label{fig:hand}
\end{figure}

В графических приложениях, где позиционирование является основной операцией, использование трекбола, по результатам некоторых исследований, приводит к существенно меньшей усталости и большей точности позиционирования курсора. 

С другой стороны, применение трекбола вместо мыши увеличивает количество движений пальцами, которые вращают шарик, что при активной работе может приводить уже к утомлению пальцев. Также есть сведения, что треболы с шаром, расположенным под большим палцем, способны при длительной и напряженной эксплуатации приводить к проблемам суставов большого пальца.


%\begin{figure}[h]
%    \centering
%    
%    \caption{Fellowes Trackball вид снизу}
%    \label{fig:bottom}
%\end{figure}

\begin{figure}[h]
    \centering
    \includegraphics[scale=0.25]{1997_fellowes_trackball/fellowsset.jpg}
    \caption{Fellowes Trackball на размерном коврике с шагом сетки 1~см}
    \label{fig:size}
\end{figure}

Трекбол не требует пространства на рабочем месте, превышающего собственные размеры, его даже можно жёстко закрепить (в том числе на негоризонтальной поверхности), гарантировав, что он случайно не переместится, не упадёт с рабочего места. В условиях ограниченного пространства или необходимости работать в неудобных положениях это может быть решающим аргументом.

Производитель трекбола в рекламных материалах делал упор на высококачественные кнопки большой площади, а также симметричный дизайн, одинаково удобный как для левой, так и для правой руки. В плане совместимости заявлена поддержка ОС Windows начиная с версии 3.1.

