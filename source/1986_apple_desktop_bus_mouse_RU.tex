\documentclass[11pt, a4paper]{article}
\input{preamble.tex}
\switchlang{ru}
\begin{document}

\title{1986 "--- Apple Desktop Bus Mouse}
\date{}
\maketitle
\selectlanguage{russian}
Мышь Apple Desktop Bus mouse (рис. \ref{fig:pic}) была представлена компанией Apple в 1986 году вместе с новой интерфейсной шиной для подключения периферийных устройств: Apple Desktop Bus (ADB). 
Это была первая мышь официально выполненная в рамках дизайн-языка <<Snow White>>, разработанного по заказу Apple в Frog Design -- компанией промышленного дизайнера Хартмута Эсслингера \cite{apple1, apple2}. 
В соответствии с концепцией <<Snow White>>, корпус мыши, включая единственную кнопку, выполнен в однотонном светло-сером (так называемом <<платиновом>>, а не белом, как можно было бы ожидать по названию) цвете, 
а кабели и разъемы выполнены в более темном <<дымчато-сером>> оттенке (рис. \ref{fig:pic}) \cite{frog}.

\begin{figure}[h]
    \centering
    \includegraphics[scale=0.5]{1986_apple_desktop_bus_mouse/pic_30.jpg}
    \caption{Apple Desktop Bus Mouse} \label{fig:pic}
\end{figure}

Корпус мыши Apple Desktop Mouse подчёркнуто прямоуголен, имеет клиновидный продольный профиль и выглядит достаточно неестественно (рис. \ref{fig:top}). 
Единственная кнопка мыши -- крупная, имеет форму прямоугольника. Она вписана в обводы корпуса, но при этом резко очерчена, чтобы облегчить пользователю поиск ее местоположения.
На нижней стороне корпуса находится поворотное кольцо контрастного черного цвета (в более поздних модификациях Apple Desktop Mouse оно выполнялось из пластика того же цвета, что и корпус), 
позволяющее извлечь шар для чистки мыши, ярлык производителя с техническими данными, а также низкофрикционные накладки, облегчающие перемещение мыши (при этом одна из двух накладок находится на поворотном кольце).

\begin{figure}[h]
    \centering
    \includegraphics[scale=0.65]{1986_apple_desktop_bus_mouse/top_30.jpg}
    \includegraphics[scale=0.65]{1986_apple_desktop_bus_mouse/bottom_30.jpg}
    \caption{Apple Desktop Bus Mouse, вид сверху и снизу} \label{fig:top}
\end{figure}

В целом эта модель сохранила прямоугольные очертания предыдущей мыши Apple, но получила меньшую высоту и уже упоминавшийся <<клин>> с ребром, расположенным перед ближней к пользователю третью корпуса (рис. \ref{fig:size}).

\begin{figure}[h]
    \centering
    \includegraphics[scale=0.4]{1986_apple_desktop_bus_mouse/size_30.jpg}
    \caption{Apple Desktop Bus Mouse на размерном коврике с шагом сетки 1~см} \label{fig:size}
\end{figure}

Эксцентричный технический дизайн сделал мышь мгновенно узнаваемой, но сказался на ее эргономике не лучшим образом.
Угловатый корпус не предоставляет достаточной опоры для ладони пользователя (рис. \ref{fig:hand}), и в целом единственным его эргономическим достоинством являются большие размеры кнопки.

\begin{figure}[h]
    \centering
    \includegraphics[scale=0.4]{1986_apple_desktop_bus_mouse/hand_30.jpg}
    \caption{Apple Desktop Bus Mouse с моделью руки человека} \label{fig:hand}
\end{figure}

Внутреннее устройство мыши показано на рис. \ref{fig:inside}. Как уже упоминалось, у мыши было несколько модификаций, которые производились несколькими компаниями (ALPS Electric, Logitech, Mitsumi).
В большинстве случаев мышь является оптомеханической, хотя известен также вариант на базе закрытых механических энкодеров ALPS \cite{apple3}. Данный экземпляр также произведен ALPS, но в его реализации 
использованы оптомеханические энкодеры в закрытом неразборном исполнении.

\begin{figure}[h]
    \centering
    \includegraphics[scale=0.6]{1986_apple_desktop_bus_mouse/inside_60.jpg}
    \caption{Apple Desktop Bus Mouse в разобранном виде} \label{fig:inside}
\end{figure}


\begin{thebibliography}{9}
    \bibitem {apple1} Fuller M. Apple Desktop Bus (ADB) Mouse (A9M0331, 1986) -- mattjfuller.com \url{http://mattjfuller.com/apple-desktop-bus-adb-mouse-a9m0331-1986/}
    \bibitem {frog} Frog Design -- Wikipeda. \url{https://en.wikipedia.org/wiki/Frog_Design}
    \bibitem {apple2} Phin C. When mice had balls: Remembering the Apple Desktop Bus Mouse -- macworld.com \url{https://www.macworld.com/article/225298/when-mice-had-balls-remembering-the-apple-desktop-bus-mouse.html}	
    \bibitem {apple3} Apple Desktop Bus Mouse Parts -- Apple Rescue of Denver \url{https://applerescueofdenver.com/products-page/macintosh-to-powerpc/keyboards-mice-joysticks-macintosh/apple-desktop-bus-mouse-parts/}
\end{thebibliography}

\end{document}
