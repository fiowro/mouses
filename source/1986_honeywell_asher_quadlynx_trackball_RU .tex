\documentclass[11pt, a4paper]{article}
\input{preamble.tex}
\switchlang{ru}
\begin{document}

\title{1986 "--- Honeywell/Asher quadLYNX trackball}
\date{}
\maketitle
\selectlanguage{russian}
Трекбол quadLYNX, показанный на рис. \ref{fig:quadLYNXPic}, является разработанной для компьютеров Apple Macintosh модификацией трекбола <<LX200>> (более известного своими разновидностями microLYNX и comLYNX), выпущенного в калифорнии компанией Honeywell, дочерним предприятием Disc Instruments. Трекбол <<LX200>> оказался долгожителем и в дальнейшем выдержал множество переизданий под разными брэндами, отличаясь интерфейсом подключения и блоком электроники \cite{lx200}. Рекламные материалы позволяют датировать модель quadLYNX под брэндом Honeywell 1986 годом \cite{honeywell}. Однако в 1988 году появляется похоже оформленная реклама данной модели как <<LX200-192-D1>> под брэндом Asher Engineering Corporation (с упоминанием Honeywell в качестве изначального разработчика устройства) \cite{asher}. В дополнение к путанице, в том же году ASHER Engineering Corporation рекламирует устройство Turbo Trackball --- исключительно под собственным брэндом, в корпусе другой формы, но с номером модели <<LX200-192-S3A>> и внутренними конструктивными решениями, позаимствованными у трекбола quadLYNX \cite{turbo}.

\begin{figure}[h]
    \centering
    \includegraphics[scale=0.45]{1986_honeywell_asher_quadlynx_trackball/pic_30.jpg}
    \caption{Внешний вид трекбола quadLYNX}
    \label{fig:quadLYNXPic}
\end{figure}

Трекбол выполнен в характерном для <<LX200>> дизайне, с большим черным шаром, утопленным в бежевом корпусе, и вытянутой подставкой под запястье \ref{fig:quadLYNXTopBottom}. Шар плотно прилегает к краям отверстия в корпусе, что неплохо защищает от попадания мусора внутрь трекбола, но делает извлечение шара для чистки невозможным без разборки корпуса. Перед шаром расположены две кнопки, копирующие внешним видом клавиши классической полноразмерной клавиатуры. Это главное визуальное отличие quadLYNX от других вариантов <<LX200>>, оснащенных тремя кнопками.

\begin{figure}[h]
    \centering
    \includegraphics[scale=0.4]{1986_honeywell_asher_quadlynx_trackball/top_30.jpg}
    \includegraphics[scale=0.4]{1986_honeywell_asher_quadlynx_trackball/bottom_30.jpg}
    \caption{quadLYNX, вид сверху и снизу}
    \label{fig:quadLYNXTopBottom}
\end{figure}



\begin{figure}[h]
    \centering
    \includegraphics[scale=0.4]{1986_honeywell_asher_quadlynx_trackball/size_30.jpg}
    \caption{Трекбол quadLYNX на размерном коврике с шагом сетки 1 см}
    \label{fig:quadLYNXSize}
\end{figure}

Однако, учитывая размеры устройства (рис. \ref{fig:quadLYNXSize}), удерживать нажатой кнопку, пока происходит
перемещение шара "--- это в лучшем случае сложный маневр. Honeywell решила эту проблему на трекболе comLYNX, в котором можно использовать среднюю кнопку в качестве <<защелки>> перетаскивания \cite{comlynx}. Сначала выполняется нажатие средней кнопки, затем левой либо правой, и это программно фиксирует выбранную кнопку в нажатом положении. Повторное нажатие любой из трех кнопок отключает данный режим. Однако в quadLYNX эта функция не поддерживается.

\begin{figure}[h]
    \centering
    \includegraphics[scale=0.4]{1986_honeywell_asher_quadlynx_trackball/hand_30.jpg}
    \caption{Трекбол quadLYNX в комплекте с моделью руки человека}
    \label{fig:quadLYNXHand}
\end{figure}

В отношении эргономики размеры трекбола, подставка под запястье, а также скругленные грани и углы создают достаточно комфортные условия для работы (рис. \ref{fig:quadLYNXHand}). Однако кнопки расположены далеко от шара и существенно ниже по высоте, что лишает пользователя как возможности нажимать их одной рукой без перемещения кисти, так и двумя руками, поскольку при вращении шара они закрыты ладонью. Особенно это затрудняет перетаскивание объектов, повсеместно использовавшееся в интерфейсе компьютеров Macintosh, для которых предназначался трекбол. Поэтому одна из кнопок (правая, меньшая по размеру кнопка на данном двухкнопочном экземпляре) работает в качестве <<защелки>> перетаскивания. Сначала выполняется нажатие кнопки-защёлки, затем кнопки, которую нужно <<зафиксировать>>, и в результате выбранная кнопка программно фиксируется в нажатом положении. Повторное нажатие любой из кнопок трекбола отключает данный режим \cite{bible}.

\begin{figure}[h]
    \centering
    \includegraphics[scale=0.52]{1986_honeywell_asher_quadlynx_trackball/inside_30.jpg}
    \caption{quadLYNX в разобранном виде}
    \label{fig:quadLYNXInside}
\end{figure}

Внутреннее устройство данного трекбола показано на рис. \ref{fig:quadLYNXInside}, что позволяет классифицировать его как устройство с механическим энкодером, в отличие от большинства вариантов <<LX-200>>, оснащавшихся оптомеханическим энкодером начания с 1986 года. Согласно информации, приведенной в \cite{lx200}, механический энкодер иногда встречается в ранних экземплярах <<LX-200>>;  кроме того, такой же реализацией энкодера оснащены трекболы Asher Turbo Mouse 1988 u  Энкодер 
Ролики реализованы с использованием подшипников и валов из нержавеющей стали, что обеспечивает максимальную надежность и долговечность конструкции. Высокая надежность трекбола и эргономика, хорошо подходящая для ряда технических задач, сделали данную модель <<долгожителем>>: устройство не меньше десяти лет выпускалось для индустриальных применений под различными марками, сохранив неизменными внешний вид и механическую часть конструкции.

\begin{thebibliography}{9}
\bibitem {comlynx} Trackballs: Stationary mice // PC Magazine. August 1987, page 199-202 \url{https://trackballs.eu/media/Fulcrum/PC%20Mag%20Aug-1987%20p199-202.pdf}
\bibitem {lx200} Disc Instruments LX200 \url{https://web.archive.org/web/20220501213039/https://forum.trackballs.eu/viewtopic.php?f=17&t=16}
\bibitem {honeywell} Try the new quadLYNX Trackball // Macworld, August 1986. P. 155  \url{https://archive.org/details/eu_Macworld-1986-08_OCR/page/n155/mode/2up}
\bibitem {asher} Try the new quadLYNX Trackball // Macworld, January 1988. P. 212 \url{https://archive.org/details/macworld00unse_oel/page/212/mode/2up}
\bibitem {turbo} New Turbo Trackball from Asher // MacUser, February, 1988. P. 320 
\url{https://archive.org/details/MacUser8802February1988/page/n323/mode/2up}
\bibitem {bible} A. Naiman (ed.) QuadLYNX Trackball, Turbo Mouse. The Macintosh Bible. Chapter 2 -- Basic Mac hardware. PP. 73-75. \url{https://archive.org/details/macintoshbibleth00naim/page/72/mode/2up}
\end{thebibliography}
\end{document}
