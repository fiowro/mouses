\documentclass[11pt, a4paper]{article}
\input{preamble.tex}
\switchlang{ru}
\begin{document}

\title{1992 "--- IBM <<Soap>> mouse}
\date{}
\maketitle
\selectlanguage{russian}
Данная модель мыши, официально известная под труднозапоминающимся названием <<модель 13H6690>> (и практически идентичная ей <<модель 33G5430>>) была по всей видимости выпущена на рынок в 1992 году, как обновление мыши IBM PS/2 1987 года выпуска.
Помимо очевидной ассоциации с обтекаемым куском мыла, эту модель можно встретить также под неофициальным названием <<IBM Fat Mouse>>.

\begin{figure}[h]
    \centering
    \includegraphics[scale=0.6]{1992_ibm_soap_mouse/pic_30.jpg}
    \caption{IBM Soap mouse}
    \label{fig:IBMSoapPic}
\end{figure}

Несмотря на широкую распространенность, эта мышь практически не представлена в обзорах в прессе. IBM комплектовала ей некоторые модели компьютеров PS/1 и PS/ValuePoint (однако, наряду с данной мышью можно встретить рекламные материалы этих компьютеров с предыдущей моделью IBM PS/2 Mouse).

\begin{figure}[h]
    \centering
    \includegraphics[scale=0.4]{1992_ibm_soap_mouse/top_60.jpg}
    \includegraphics[scale=0.4]{1992_ibm_soap_mouse/bottom_60.jpg}
    \caption{IBM Soap mouse, вид сверху и снизу}
    \label{fig:IBMSoapTopBottom}
\end{figure}

Мышь выпускалась в нескольких вариантах расцветки: однотонный корпус бежевого цвета, и двухцветные варианты с темными кнопками (рис. \ref{fig:IBMSoapTopBottom}) либо темной нижней частью корпуса \cite{hugold}. На верхней стороне присутствуют две большие кнопки и выгравированная эмблема компании IBM; в целом корпус минималистичен и кроме этого не содержит никаких дополнительных элементов. На нижней части можно видеть шар и поворотное кольцо, позволяющее извлечь его для чистки.

\begin{figure}[h]
    \centering
    \includegraphics[scale=0.34]{1992_ibm_soap_mouse/size_30.jpg}
    \caption{Изображение IBM Soap mouse на размерном коврике с шагом сетки 1~см}
    \label{fig:IBMSoapSize}
\end{figure}

Мышь компактна (рис. \ref{fig:IBMSoapSize}) и не предоставляет опоры для запястья.
Однако, как можно видеть на рис. \ref{fig:IBMSoapHand}, изогнутая форма корпуса позволяет комфортно опереться на неё ладонью, и нажимать кнопки при естественном положении кисти. При этом мышь симметрична и одинаково подходит для левшей и правшей. В дополнение к эргономичности, мышь получилась удачной и в плане надежности \cite{usage}, что по всей видимости сделало ее популярной и послужило причиной довольно продолжительного массового производства.

\begin{figure}[h]
    \centering
    \includegraphics[scale=0.34]{1992_ibm_soap_mouse/hand_30.jpg}
    \caption{Изображение IBM Soap mouse с моделью руки человека}
    \label{fig:IBMSoapHand}
\end{figure}

Внутреннее устройство IBM Soap mouse показано на рис. \ref{fig:IBMSoapInside}, что позволяет классифицировать мышь как устройство с оптомеханическим энкодером, изготовленное компанией Logitech. Диски энкодеров с частыми прорезями, незначительное число дискретных элементов на печатной плате и пластиковые ролики характерны скорее для мышей середины, чем начала 90-х годов, что позволяет отнести IBM Soap mouse к числу родоначальников этой тенденции.

\begin{figure}[h]
    \centering
    \includegraphics[scale=0.7]{1992_ibm_soap_mouse/inside_60.jpg} 
    \caption{IBM Soap mouse в разобранном состоянии}
    \label{fig:IBMSoapInside}
\end{figure}

\begin{thebibliography}{9}
\bibitem {usage} Wendt P.H. Mice \& other stuff \url{http://www.mcamafia.de/mycomp/mycomp06.htm}
\bibitem {hugold} Przytul grata. IBM Mouse PS/2. \url{http://hugold.pl/gratym0126/ibm33g5430.html}
\end{thebibliography}
\end{document}
