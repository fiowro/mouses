\documentclass[11pt, a4paper]{article}
\input{preamble.tex}
\switchlang{en}
\begin{document}

\title{1986 "--- NEC Crystal mouse}
\date{}
\maketitle
\selectlanguage{english}
In September 1986, NEC Corporation announced the EWS 4800 UNIX workstation. This computer was designed as a workstation for engineers to improve the efficiency of solving problems such as software development, computer-aided design, scientific and engineering calculations, and the collection and analysis of experimental data \cite{yt}. The computer was equipped with a graphical multi-window interface and a large 20-inch $1280 \times 1024 \times 256$ display. This powerful workstation came with the NEC Crystal Mouse (figure \ref{fig:NECCrystalPic}).

\begin{figure}[h]
    \centering
    \includegraphics[scale=0.7]{1986_nec_crystal_mouse/necNorm_30.jpg}
    \caption{NEC Crystal Mouse}
    \label{fig:NECCrystalPic}
\end{figure}

Like all early optical mice, this manipulator requires a reflective pad with a grid pattern. Unlike traditional Mouse Systems “mirror” pads with a grid of vertical and horizontal lines corresponding to two different wavelengths, the Nec Crystal Mouse's metal pad has a dark glossy surface (figure \ref{fig:NecCrystalPad}).

\begin{figure}[h]
    \centering
    \includegraphics[scale=0.6]{1986_nec_crystal_mouse/necPad_30.jpg}
    \caption{NEC Crystal Mouse on its pad}
    \label{fig:NecCrystalPad}
\end{figure}

The name of the mouse “Crystal Mouse” is highlighted on the upper side of the case with two triangular stylized mice in outlines and solid lines. The bottom side shows that this is an optical mouse (figure \ref{NecCrystalTopAndBottom}), which largely repeats the external design solutions of the Mouse Systems mice of the same period \cite{photo}.

\begin{figure}[h]
    \centering
    \includegraphics[scale=0.75]{1986_nec_crystal_mouse/nectop_60.jpg}
    \includegraphics[scale=0.75]{1986_nec_crystal_mouse/necbottom_60.jpg}
    \caption{NEC Crystal Mouse, top and bottom views}
    \label{NecCrystalTopAndBottom}
\end{figure}

In terms of size, the manipulator is an optical cursor control device typical for the 80s (figure \ref{fig:NecCrystalSize}).

\begin{figure}[h]
    \centering
    \includegraphics[scale=0.45]{1986_nec_crystal_mouse/necSize_30.jpg}
    \caption{NEC Crystal Mouse on a graduated pad with a grid step of 1~cm}
    \label{fig:NecCrystalSize}
\end{figure}

In terms of ergonomics, the exterior of the Crystal Mouse has a strong industrial design. At the same time, a large number of corners and flat edges are partly compensated by rounded joints of the edges in the part of the body closest to the user and convex long buttons conveniently located within the reach of the fingers (figure \ref{fig:NecCrystalHand}).

~

\begin{figure}[h]
    \centering
    \includegraphics[scale=0.4]{1986_nec_crystal_mouse/necHand_30.jpg}
    \caption{NEC Crystal Mouse with a human hand model}
    \label{fig:NecCrystalHand}
\end{figure}

The manipulator's internal structure is shown on figure \ref{fig:NecCrystalInside}, where you can see the unusual optical mouse design different from most optical mice of the 1980s, which copied the design of Mouse Systems. Light emitted by LEDs is reflected from the mouse pad, then re-reflected by mirrors inside the mouse (a pair of rectangular blocks with a blue outer surface in the center of the body), and finally reaches the lines of photodetectors, one located along the longitudinal axis and the other along the transverse axis. The LEDs (and, consequently, the mirrors) in Nec Crystal Mouse are positioned at different angles to the mouse pad's surface.

\begin{figure}[h]
    \centering
    \includegraphics[scale=0.75]{1986_nec_crystal_mouse/necraz_60.jpg}
    \caption{NEC Crystal Mouse disassembled}
    \label{fig:NecCrystalInside}
\end{figure}

The mousepad is a metal plate with a translucent coating that conceals the grid. The longitudinal and transverse lines of the grid are at different heights, and their visibility depends on the angle of the optical axis relative to the surface. The figure \ref{fig:NecCrystalPad} shows the Nec Crystal Mouse on a mousepad from a later mouse of similar design, the Kokuyo EAM-102; as can be judged from the photograph in \cite{yt}, the original Nec mousepad was larger and likely had a lower grid resolution.

Also, the figure \ref{fig:NecCrystalInside} allows us to conclude that the mouse is connected to the computer via some sort of a serial interface (this is evident from its four-wire cable, which, according to the markings on the printed circuit board, has two wires responsible for receiving and transmitting data bits, and two wires to provide power).

\begin{thebibliography}{9}
\bibitem {yt} NEC EWS4800 \url{http://museum.ipsj.or.jp/en/computer/work/0003.html}
\bibitem {photo} Graphics NEC EWS4800 \url{http://www.cs.ce.nihon-u.ac.jp/facility/exp-gazo.html}
\end{thebibliography}
\end{document}
