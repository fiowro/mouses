\documentclass[11pt, a4paper]{article}
\input{preamble.tex}

\begin{document}

\title{1998 "--- Apple Puck Mouse}
\date{}
\maketitle
Мышь Apple USB mouse, часто называемая <<шайбой>> (англ. <<puck>>)  из-за своей необычной формы, была разработана компанией Apple в 1998 году. Это была первая коммерчески выпущенная мышь Apple mouse, которая использовала формат подключения USB, а не шину Apple ADB. Многие обозреватели критиковали данную мышь за ее недостаточно эргономичный дизайн.

\begin{figure}[h]
    \centering
    \includegraphics[scale=0.8]{1998_apple_puck/apple60.jpg}
    \caption{Apple Puck Mouse}
    \label{fig:pic}
\end{figure}

В отличие от большинства манипуляторов, эта мышь имеет круглую форму, и у нее есть одна кнопка, расположенная вверху, как и у предыдущих мышей Apple. При этом мышь имеет зазор между кнопкой и корпусом, показывающий, куда именно пользователь должен нажимать \cite{Apple}.

\begin{figure}[h]
    \centering
    \includegraphics[scale=0.65]{1998_apple_puck/appleup60.jpg}
    \includegraphics[scale=0.65]{1998_apple_puck/appledown60.jpg}
    \caption{Apple Puck Mouse, вид сверху и снизу}
    \label{fig:top}
\end{figure}

Круглая форма мыши была признана сообществом неудобной из-за небольшого размера данного конкретного манипулятора и склонности вращаться при использовании.
Также из-за малого размера, перемещение мыши на самом деле требовало гораздо большего количества движений  пальцев и  меньшего количества движений запястья по сравнению с более крупными мышами (рис. \ref{fig:size}, \ref{fig:hand}).

\begin{figure}[h]
    \centering
    \includegraphics[scale=0.4]{1998_apple_puck/appleset60.jpg}
    \caption{Apple Puck Mouse на размерном коврике с шагом сетки 1~см}
    \label{fig:size}
\end{figure}

\begin{figure}[h]
    \centering
    \includegraphics[scale=0.4]{1998_apple_puck/appleset62.jpg}
    \caption{Apple Puck Mouse с моделью руки человека}
    \label{fig:hand}
\end{figure}

Это стало основной причиной успеха адаптеров-накладок, таких как iCatch Mouse Adapter \cite{icatch}, придававших мыши более продолговатую форму (рис. \ref{fig:addon}).

\begin{figure}[h]
    \centering
    \includegraphics[scale=0.45]{1998_apple_puck/apple63.jpg}
    \includegraphics[scale=0.45]{1998_apple_puck/appleup63.JPG}
    \includegraphics[scale=0.45]{1998_apple_puck/appledown63.JPG}
    \caption{Apple Puck Mouse, вид с накладкой}
    \label{fig:addon}
\end{figure}



В полупрозрачном пластике помещалась печатная плата и двухцветный шар, который можно было легко разглядеть. 


Однако идеально круглое тело часто приводило к ошибкам, так как пользователи предполагали, что мышь была в правильной ориентации, даже если это было не так. Позже Apple добавила ямочку на корпусе мыши, чтобы помочь пользователям почувствовать, в каком направлении указывала мышь.
\begin{figure}[h]
    \centering
    \includegraphics[scale=0.6]{1998_apple_puck/apple62.jpg}
    \caption{Apple Puck Mouse, в разобранном виде}
    \label{fig:inside}
\end{figure}

Внутреннее устройство данной мыши показано на рис. \ref{fig:inside}, что позволяет классифицировать  ее как оптомеханическую.

\begin{thebibliography}{9}

    \bibitem {Apple} An ode to the puck, Apple's first USB mouse "--- \url{https://thehouseofmoth.com/an-ode-to-the-puck-apples-first-usb-mouse/} 
    \bibitem{icatch} The iCatch Mouse Adapter \url{https://web.archive.org/web/20001024170633/http://www.macsensetech.com:80/Product/iCatch.html}
\end{thebibliography}

\end{document}
