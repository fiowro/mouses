\documentclass[11pt, a4paper]{article}
\input{preamble.tex}
\switchlang{ru}
\begin{document}

\title{2000 "--- IBM ScrollPoint II mouse}
\date{}
\maketitle
\selectlanguage{russian}
Модель MO09K, появившаяся в продаже в районе 2000 года, была вторым поколением мыши IBM ScrollPoint, дебютировавшей в 1998 году \cite{hist}. Особенностью этой мыши IBM была замена колеса прокрутки на миниатюрный тензометрический джойстик (TrackPoint), аналогичный средствам управления курсором, используемым в линейке ноутбуков ThinkPad. Благодаря интеграции трекпоинта, IBM ScrollPoint стала первой мышью, поддерживающей как горизонтальную, так и вертикальную прокрутку \cite{buxtonG1}. При этом направление скроллинга определяется направлением нажима на джойстик, а его скорость задается силой нажима.

\begin{figure}[h]
    \centering
    \includegraphics[scale=0.6]{2000_ibm_scrollpoint_ii_mouse/pic_30.jpg}
    \caption{IBM ScrollPoint II mouse}
    \label{fig:IBMScrollPointIIPic}
\end{figure}

Второе поколение мыши ScrollPoint получило корпус чуть более округлой формы, новую форму резинового наконечника джойстика, а также третью кнопку, расположенную непосредственно за ним (рис. \ref{fig:IBMScrollPointIITopBottom}).

\begin{figure}[h]
    \centering
    \includegraphics[scale=0.7]{2000_ibm_scrollpoint_ii_mouse/top_30.jpg}
    \includegraphics[scale=0.7]{2000_ibm_scrollpoint_ii_mouse/bottom_30.jpg}
    \caption{IBM ScrollPoint II mouse, вид сверху и снизу}
    \label{fig:IBMScrollPointIITopBottom}
\end{figure}

Замена традиционного для ноутбуков круглого наконечника трекпоинта на подобие вогнутой ребристой качельки было призвано облегчить горизонтальный скроллинг. Как известно, пальцы приспособлены к приложению усилия вбок меньше, чем вниз. По замыслу разработчиков, вертикальное нажатие средним пальцем вблизи от края вогнутой поверхности должно за счет эффекта рычага облегчать горизонтальную прокрутку \cite{buxtonG2}.

\begin{figure}[h]
    \centering
    \includegraphics[scale=0.54]{2000_ibm_scrollpoint_ii_mouse/size_30.jpg}
    \caption{Изображение IBM ScrollPoint II mouse на размерном коврике с шагом сетки 1~см}
    \label{fig:IBMPS2Size}
\end{figure}

Как можно видеть на рис. (рис. \ref{fig:IBMPS2Size}), мышь имеет типичные средние размеры. Изогнутая форма корпуса позволяет комфортно опереться на неё ладонью  \ref{fig:IBMScrollPointIIHand} и нажимать кнопки при естественном положении кисти. При этом корпус симметричен и одинаково подходит для левшей и правшей.


\begin{figure}[h]
    \centering
    \includegraphics[scale=0.6]{2000_ibm_scrollpoint_ii_mouse/hand_30.jpg}
    \caption{Изображение IBM ScrollPoint II mouse с моделью руки человека}
    \label{fig:IBMScrollPointIIHand}
\end{figure}

Внутреннее устройство IBM ScrollPoint II mouse показано на рис. \ref{fig:IBMPS2Inside}, что позволяет классифицировать мышь как типовое (помимо наличия тензометрического джойстика) устройство с оптомеханическим энкодером. Однако конструкция мыши демонстрирует отчетливые признаки консерватизма: характерным примером является защита шара, типичная для первой половины девяностых годов, но не слишком характерная для для манипуляторов 2000 года выпуска.

\begin{figure}[h]
    \centering
    \includegraphics[scale=0.7]{2000_ibm_scrollpoint_ii_mouse/inside_30.jpg} 
    \caption{IBM ScrollPoint II mouse в разобранном состоянии}
    \label{fig:IBMPS2Inside}
\end{figure}

\begin{thebibliography}{9}
    \bibitem {buxtonG1} Bill Buxton. TrackPoint Mouse G1 -- Buxton Collection. \url{https://www.microsoft.com/buxtoncollection/detail.aspx?id=120}
    \bibitem {buxtonG2} Bill Buxton. TrackPoint Mouse G2 -- Buxton Collection. \url{https://www.microsoft.com/buxtoncollection/detail.aspx?id=121}
    \bibitem {hist} IBM ScrollPoint Information and Software -- ibmfiles.com // \url{http://www.ibmfiles.com/pages/scrollpoint.htm}
\end{thebibliography}

\end{document}
