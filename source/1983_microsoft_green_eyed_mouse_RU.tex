\documentclass[11pt, a4paper]{article}
\input{preamble.tex}
\switchlang{ru}
\begin{document}

\title{1983 "--- Microsoft Green Eyed Mouse}
\date{}
\maketitle
\selectlanguage{russian}

Первая мышь Microsoft была выпущена в 1983 году и стала одновременно первым продуктом подразделения по разработке аппаратного обеспечения, которое компания открыла годом ранее. Из-за двух зелёных кнопок эта модель получила известность под названием <<зеленоглазой мыши>>.

Поскольку у Microsoft на тот момент было недостаточно опыта в разработке и изготовлении аппаратного обеспечения, реальным производителем мыши выступила японская компания Alps "--- производитель первой японской мыши, MZ-1X10 mouse, представленной в том же 1983 году.

\begin{figure}[h]
   \centering
    \includegraphics[scale=0.6]{1983_microsoft_green_eyed_mouse/pic_30.jpg}
    \caption{Microsoft Green Eyed Mouse}
    \label{fig:MicrosoftGreenEyedPic}
\end{figure}

В конструктивном плане первая мышь Microsoft и MZ-1X10 имеют чрезвычайно много общего. Однако если корпус мыши MZ-1X10 представляет собой паралеллепипед со слегка скругленными гранями и парой прямоугольных кнопок на верхней стороне корпуса, то корпус мыши Microsoft имеет более сложную форму, а кнопки смещены на наклонную переднюю стенку (рис.  \ref{fig:MicrosoftGreenEyedPic}).

\begin{figure}[h]
    \centering
    \includegraphics[scale=0.55]{1983_microsoft_green_eyed_mouse/top_60.jpg}
    \includegraphics[scale=0.55]{1983_microsoft_green_eyed_mouse/bottom_60.jpg}
    \caption{Microsoft Green Eyed Mouse, вид сверху и снизу}
    \label{fig:MicrosoftGreenEyedTopAndBottom}
\end{figure}

Очевидно, что компания Microsoft придавала дизайну своей мыши большое значение. Корпус слегка кремового оттенка выполнен в стиле минимализма, единственными элементами являются две контрастные зеленые кнопки и едва заметное название компании, вытесненное у ближнего к пользователю края корпуса (рис. \ref{fig:MicrosoftGreenEyedTopAndBottom}). Движение регистрируется тяжелым стальным шаром, расположенным ближе к задней части мыши, а еще три маленьких гладко отполированных шарика играют роль ножек для минимизации трения. Также в нижней стенке корпуса предусмотрено съемное кольцо, позволяющее извлечь шар для удаления собравшегося мусора; однако вариант кольца на защелках еще не был придуман, поэтому его требуется отвинчивать с помощью отвертки.

\begin{figure}[h]
    \centering
    \includegraphics[scale=0.5]{1983_microsoft_green_eyed_mouse/size_30.jpg}
    \caption{Microsoft Green Eyed Mouse на размерном коврике с шагом сетки 1~см}
    \label{fig:MicrosoftGreenEyedSize}
\end{figure}

Несмотря на малые размеры мыши (рис. \ref{fig:MicrosoftGreenEyedSize}), она довольно тяжелая. Очевидно, что внимание Microsoft к форме корпуса не могло не сказаться положительно на эргономике. В сравнении с ближайшим родственником мышью MZ-1X10, ладонь и пальцы располагаются на корпусе в более естественном положении (рис. \ref{fig:MicrosoftGreenEyedHand}).

\begin{figure}[h]
    \centering
    \includegraphics[scale=0.5]{1983_microsoft_green_eyed_mouse/hand_30.jpg}
    \caption{Microsoft Green Eyed Mouse с моделью руки человека}
    \label{fig:MicrosoftGreenEyedHand}
\end{figure}

Ранние версии мыши Microsoft имели шинный интерфейс и комплектовались специальной платой-адаптером для установки в системный блок. Позже появились версии с последовательным интерфейсом и 25 либо 9-контактным разъемом \cite{mouses}. Разрешение мыши составляло всего 100 DPI \cite{review}.

Внутреннее устройство мыши показано на рис. \ref{fig:MicrosoftGreenEyedInside}. В мыши использованы закрытые контактные энкодеры. При сравнении с мышью MZ-1X10 обнаруживается почти полная идентичность конструкции: отличия наблюдаются в конфигурации печатной платы и связаны с перемещением кнопок на переднюю стенку корпуса.

 \begin{figure}[h]
    \centering
    \includegraphics[scale=1]{1983_microsoft_green_eyed_mouse/inside_30.jpg}
    \caption{Microsoft Green Eyed Mouse в разобранном виде}
    \label{fig:MicrosoftGreenEyedInside}
\end{figure}



\begin{thebibliography}{9}
\bibitem{mouses} Microsoft Green Eyed Mouse \url{https://web.archive.org/web/20211205011010/https://www.oldmouse.com/mouse/microsoft/greeneyed.shtml}
\bibitem{review} Hart G. Building a better mouse interface // PC Magazine, February 25, 1986. -- pp. 167-170. \url{https://archive.org/details/PC-Mag-1986-02-25/page/173/mode/2up}
\end{thebibliography}
\end{document}
