\documentclass[11pt, a4paper]{article}
\input{preamble.tex}
\switchlang{ru}
\begin{document}

\title{1986 "--- TPX Mouse}
\date{}
\maketitle
\selectlanguage{russian}

TPX Mouse (рис. \ref{fig:TPXPic}) "--- это устройство управления курсором, выпущенное бразильской компанией Tropic Inform\'atica в 1986 году для компьютера TK90X (бразильского клона ZX Spectrum) и доработанное  годом позже для компьютеров архитектуры MSX (версия с интерфейсом MSX продавалась компанией Input Digital под именем <<Input Mouse>>) \cite{tpx, msxwiki}. Компьютер TK90X, для которого первоначально предназначалась мышь TPX, был бразильским клоном ZX Spectrum, созданным в 1985 году компанией Microdigital Electr\^onica: он оснащался 16 либо 48 кБ оперативной памяти, процессором с тактовой частотой 3.58 МГц и 15-цветным видеорежимом с разрешением $256 \times 192$ пиксела \cite{tk90x}.

\begin{figure}[h]
   \centering
    \includegraphics[scale=0.75]{1986_tpx_mouse/pic_60.jpg}
    \caption{TPX Mouse}
    \label{fig:TPXPic}
\end{figure}

Верхняя часть корпуса TPX Mouse имеет сравнительно сложную по меркам 80-х годов форму: выпуклые боковые стенки и изогнутая верхняя сторона в зоне стыковки с нижней частью переходят в строгий прямоугольник, стыкующийся с полностью прямоугольной нижней частью корпуса. Мышь имеет наклонную переднюю (дальнюю от пользователя) сторону, на которой расположены две кнопки синего цвета (рис. \ref{fig:TPXTopAndBottom}). Верх корпуса воспроизводит стиль первой мыши Microsoft, выпущенной в 1983 году и известной за зеленый цвет кнопок как <<Зеленоглазая мышь>>. Форма в данном случае воспроизводит форму зеленоглазой мыши лишь настолько, насколько это реализуемо с учетом прямоугольной нижней части корпуса. Нижняя часть корпуса вплоть до мельчайших деталей совпадает с нижней частью корпуса мыши Torrington Manager Mouse 1985 года выпуска. Как и у мыши Torrington, на нижней стороне можно увидеть наклейку с информационными данными, а также два маленьких колеса в форме усечённого конуса, предназначенные для регистрации движения вместо более традиционного для механических мышей шара. Это решение восходит к первым мышам Дугласа Энгельбарта: колеса расположены ортогонально друг другу, чтобы при продольном и поперечном движении одно из них вращалось, а другое проскальзывало. Но, по сравнению с мышью Энгельбарта, колеса TPX Mouse (вслед за ее прототипом Manager Mouse) существенно уменьшены в диаметре, а их оси расположены под углом к горизонтальной плоскости, что должно обеспечивать регистрацию диагональных перемещений. Как и у Manager Mouse, кабель TPX Mouse  не снабжён ограничительной втулкой, которая защищала бы его от механических повреждений в месте выхода из корпуса.

\begin{figure}[h]
    \centering
    \includegraphics[scale=0.8]{1986_tpx_mouse/top_60.jpg}
    \includegraphics[scale=0.8]{1986_tpx_mouse/bottom_60_30.jpg}
    \caption{TPX Mouse, вид сверху и снизу}
    \label{fig:TPXTopAndBottom}
\end{figure}

Размеры мыши являются средними по меркам 1980х  (рис. \ref{fig:TPXSize}) "--- в первую очередь они определяются размерами основания, позаимствованного у мыши Torrington, а также, отчасти, формой мыши Microsoft. Однако благодаря отсутствию стального шара и более миниатюрному узлу регистрации движения TPX Mouse получилась существенно более легкой и также более приземистой, чем зеленоглазая мышь "--- лишь чуть выше мыши Torrington.

\begin{figure}[h]
    \centering
    \includegraphics[scale=0.55]{1986_tpx_mouse/size_30.jpg}
    \caption{TPX Mouse на размерном коврике с шагом сетки 1~см}
    \label{fig:TPXSize}
\end{figure}

Форма верхней части и расположение кнопок на наклонной грани, призванное обеспечить комфортное положение ладони на корпусе мыши, дают TPX Mouse определенное преимущество в эргономике по сравнению с другими мышами Spectrum-совместимых компьютеров (рис. \ref{fig:TPXHand}). Однако при этом кнопки достаточно неудобны в использовании из-за большого хода и отсутствия щелчка при нажатии. К недостаткам дизайна можно отнести также и более высокую вероятность случайно сдвинуть мышь назад при нажатии кнопок (в случае Microsoft Mouse этот эффект был менее выражен за счет большого веса мыши).

Внутреннее устройство мыши показано на рис. \ref{fig:TPXInside}. Как можно видеть, она представляет собой механическую конструкцию, в которой диски механического энкодера закреплены непосредственно на осях колес. Внутренняя компоновка корпуса, форма печатных плат, узел регистрации движения и фиксирующая оси накладка, полностью совпадают с мышью Torrington "--- вплоть до использования достаточно дорогих подшипников в качестве втулок, прижимаемых накладкой для удержания осей под нужными углами \cite{retrofit}.

\begin{figure}[h]
    \centering
    \includegraphics[scale=0.62]{1986_tpx_mouse/hand_30.jpg}
    \caption{TPX Mouse с моделью руки человека}
    \label{fig:TPXHand}
\end{figure}

Этот последний компонент контрастирует с дешевизной и малой надежностью остальных элементов конструкции, однако объясняется тем, что компания Torrington к моменту выпуска мыши была крупным международным производителем подшипников. Дополнительно, это можно считать убедительным свидетельством использования в составе TPX Mouse готового узла производства компании Torrignton.


 \begin{figure}[h!]
    \centering
    \includegraphics[width=\textwidth]{1986_tpx_mouse/inside_60.jpg}
    \caption{TPX Mouse в разобранном виде}
    \label{fig:TPXInside}
\end{figure}


При этом электронные компоненты имеют мало общего с Manager Mouse, и к тому же их большая часть перенесена в разъем мыши. Такое решение достаточно типично для подключения мышей к ZX Spectrum и Spectrum-совместимым компьютерам: из-за отсутствия у них специализированного интерфейса, квадратурные мыши подключались к достаточно габаритному разъему системной шины и дополнительно оснащались контроллером параллельного интерфейса, обеспечивающим взаимодействие с центральным процессором на уровне прерываний \cite{pio}.

\begin{thebibliography}{9}
    \bibitem {tpx} Mouse TPX para MSX -- Tabajara Labs [in Portugal] \url{https://web.archive.org/web/20180510093756/https://www.tabalabs.com.br/msx/mouse_tpx/index.htm}
    \bibitem {msxwiki} TPX Mouse -- MSX Wiki \url{https://www.msx.org/wiki/TPX_Mouse}
    \bibitem {tk90x} TK90X -- Wikipedia \url{https://en.wikipedia.org/wiki/TK90X}
    \bibitem {retrofit} Sturaro L. Mouse TPX, fazendo um retrofit -- The MSX Hardware Page [in Portugal] \url{https://www.msxpro.com/tpx_retrofit.html}
    \bibitem {pio} Z8420 datasheet -- Zilog, Inc. \url{https://www.alldatasheet.com/html-pdf/78374/ZILOG/Z8420/124/1/Z8420.html}
\end{thebibliography}
\end{document}
