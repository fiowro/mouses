\documentclass[11pt, a4paper]{article}
\input{preamble.tex}
\switchlang{en}
\begin{document}

\title{1985 -- Torrington Manager Mouse}
\date{}
\maketitle
\selectlanguage{english}

The Manager Mouse was released by a Taiwanese company KYE Systems (owner of the Genius brand) for use with Commodore 64 computers (the first mass-produced home computers, which were on the market from 1982 to 1992). The mouse can be dated based on KYE Systems advertising that mentions the GM-2 (a mouse produced by Z-Nix in 1986 and sold under several brands), the GM-3 (the first mouse produced by KYE itself, in a "new" case relative to the GM-2, with a serial interface and external power supply), and the similar GM-4 without additional power supply \cite{YourComputer}. The GM-6 model, in an even newer case with wide flat buttons, which became the first truly mass-produced Genius mouse (and also the first Genius mouse with a registered FCC ID), dates back to 1987, and its modification GM-6000 dates back to 1988. Thus, the GM-5 mouse is actually a slightly later modification of the GM-3, made for Commodore computers \cite{armadale}, and could have appeared either in late 1986 or, more likely, in 1987.

\begin{figure}[h]
   \centering
    \includegraphics[scale=0.9]{1985_torrington_manager_mouse/pic_30.jpg}
    \caption{Manager Mouse}
    \label{fig:ManagerMousePic}
\end{figure}

The mouse is made in a beige case, expressing strict industrial design and minimalism. The case has an almost rectangular shape, if you do not count the forward-slanted part of the upper side, on which there are three elongated rectangular gray buttons. The mouse cable is not provided with protection from mechanical damage at the point where it exits the case. On the bottom (fig. \ref{fig:ManagerMouseTopAndBottom}) there is a removable ring made of contrasting gray plastic, designed to remove the ball and clean the mouse (it is attached with a screw -- a design typical of mice from the first half of the 80s). Also on the bottom there are three additional metal balls, which act as supports with a low coefficient of friction.

\begin{figure}[h]
    \centering
    \includegraphics[scale=0.8]{1985_torrington_manager_mouse/top_30.jpg}
    \includegraphics[scale=0.8]{1985_torrington_manager_mouse/bottom_30.jpg}
    \caption{Manager Mouse, top and bottom views}
    \label{fig:ManagerMouseTopAndBottom}
\end{figure}

The case is of a size typical for the first half of the 1980s and does not show any identification of the mouse model. Only a metal plate located on the side closest to the user has the inscription ``Genius Mouse'' with the Genius logo inscribed (fig. \ref{fig:ManagerMousePic}, \ref{fig:ManagerMouseSize}). It should also be noted that the GM-5 mouse was produced for quite a long time, and there are known 3-button and even 2-button GM-5 samples in a newer case -- the one in which the GM-6 and GM-6000 \cite{commodore, atari} models appeared. The replacement of the ``old'' case with the ``new'' one in Genius mice occurred gradually: at least the GM-3a (a serial mouse that receives additional power from the keyboard) and GM-4 models can also be found in the new case. The lack of clear identification of the mouse model on the cases allowed KYE Systems to use such case variants as they could, and even alternate them.

\begin{figure}[h]
    \centering
    \includegraphics[scale=0.56]{1985_torrington_manager_mouse/size_30.jpg}
    \caption{Manager Mouse on a graduated pad with a grid step of 1~cm}
    \label{fig:ManagerMouseSize}
\end{figure}

In terms of ergonomics, the GM-5 mouse does not have many advantages. The user experience obviously suffers from the severe rectangularity of the body, given that it has a significant height and cannot provide much support for the palm (fig. \ref{fig:ManagerMouseHand}).

\begin{figure}[h]
    \centering
    \includegraphics[scale=0.51]{1985_torrington_manager_mouse/hand_30.jpg}
    \caption{Manager Mouse with a human hand model}
    \label{fig:ManagerMouseHand}
\end{figure}

The FCC ID code reveals that this mouse was released in 1985 by The Torrington Company in Connecticut.

The internal structure of the mouse is shown in fig. \ref{fig:ManagerMouseInside}. As you can see, it is an optomechanical design typical of the second half or end of the 80s: with a massive plastic frame covering most of the printed circuit board, and fairly durable brass rollers.

 \begin{figure}[h]
    \centering
    \includegraphics[scale=0.65]{1985_torrington_manager_mouse/inside_30.jpg}
    \caption{Manager Mouse disassembled}
    \label{fig:ManagerMouseInside}
\end{figure}

\begin{thebibliography}{9}
\bibitem {torrington} The Torrington Company / International Directory of Company Histories. ed. by T. Grant. Vol. 13. - St.James Press, USA. 1996. P. 521 \url{https://archive.org/details/internationaldir0013unse/page/520/mode/2up}

\bibitem {admaynard} This Little Fella Means Business [adv.] // PC Magazine, Vol. 5, No. 1, January 14, 1986. P. 37. \url{https://archive.org/details/PC-Mag-1986-01-14/page/n71/mode/2up}

\bibitem {adnumonics} Don't let your mouse get hung-up on itself [adv.] // PC Magazine, Vol. 7, No. 6, March 29, 1988. P. 282 // \url{https://archive.org/details/PC-Mag-1988-03-29/page/n277/mode/2up}

\bibitem {maynard} Hart G. Building a better mouse interface. Maynard mouse // PC Magazine, Vol. 5, No. 4, February 25, 1986. - p. 170. \url{https://archive.org/details/PC-Mag-1986-02-25/page/n177/mode/2up}

\bibitem {managerwheels} Numonics corp. Manager mouse, Manager mouse cordless. //  PC Magazine, V. 7, No. 3, February 14, 1989. - p. 268. \url{https://archive.org/details/PC-Mag-1989-02-14/page/n267/mode/2up}

\bibitem {buxton} Manager Mouse Cordless \url{https://www.billbuxton.com/inputTimelineAssets/TorringtonBrochure.pdf}
\end{thebibliography}
\end{document}
