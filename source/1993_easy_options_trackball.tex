\documentclass[11pt, a4paper]{article}
\input{preamble.tex}

\begin{document}

\title{1993 "--- Easy Options trackball}
\date{}
\maketitle

Наиболее заметной визуальной особенностью трекбола Easy Options является дизайн в стиле бытовых приборов 60-х годов (рис. \ref{fig:EasyOptionsTopBottom}).

\begin{figure}[h]
    \centering
    \includegraphics[scale=0.5]{1993_easy_options_trackball/pic_60.jpg}
    \caption{Трекбол Easy Options}
    \label{fig:EasyOptionsSize}
\end{figure}

Перевернув мышь, можно увидеть код FCC ID (рис. \ref{fig:EasyOptionsTopBottom}), исследование которого показывает, что трекбол производился на Тайване начиная с 1993 года.

\begin{figure}[h]
    \centering
    \includegraphics[scale=0.6]{1993_easy_options_trackball/top_60.jpg}
    \includegraphics[scale=0.6]{1993_easy_options_trackball/bottom_60.jpg}
    \caption{Easy Options, вид сверху и снизу}
    \label{fig:EasyOptionsTopBottom}
\end{figure}

Как можно видеть на рис. \ref{fig:EasyOptionsSize}, этот манипулятор имеет сравнительно небольшие размеры.

\begin{figure}[h]
    \centering
    \includegraphics[scale=0.35]{1993_easy_options_trackball/size_30.jpg}
    \caption{Easy Options на размерном коврике с шагом сетки 1 см}
    \label{fig:EasyOptionsSize}
\end{figure}

В плане эргономики можно отметить закруглённые формы корпуса, практически не имеющего острых углов, а также левую и правую кнопки, занимающие всю его длину, что достаточно хорошо согласуется с анатомическим строением кисти (рис. \ref{fig:EasyOptionsHand}).

\begin{figure}[h]
    \centering
    \includegraphics[scale=0.35]{1993_easy_options_trackball/hand_30.jpg}
    \caption{Easy Options в комплекте с моделью руки человека}
    \label{fig:EasyOptionsHand}
\end{figure}

В Easy Options также используется специальная кнопка, которая позволяет включить либо отключить все остальные кнопки во избежиание случайного нажатия и для удобного перемещения курсора мыши соответственно. Еще одной отличительной особенностью манипулятора является двусторонний разъём, позволяющий подключаться к последовательному порту, оснащенному гнездом как с 9, так и с 25 контактами.

\begin{figure}[h]
    \centering
    \includegraphics[scale=0.6]{1993_easy_options_trackball/inside_60.jpg}
    \caption{Easy Options в разобранном виде}
    \label{fig:EasyOptionsInside}
\end{figure}

Рисунок \ref{fig:EasyOptionsInside} показывает, что манипулятор выполнен по классической оптомеханической схеме. Из особенностей конструкции можно отметить четыре пружины, обеспечивающие равномерный ход вытянутым в длину левой и правой клавишам, а также подпружиненную центральную клавишу.

\end{document}
