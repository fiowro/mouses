\documentclass[11pt, a4paper]{article}
\input{preamble.tex}
\switchlang{ru}
\begin{document}

\title{1985 "--- Microsoft Gray-eyed Mouse}
\date{}
\maketitle
\selectlanguage{russian}

Данная мышь, появившаяся в продаже в 1985 году, стала вторым поколением мышей Microsoft. Компания просто называла свои ранние модели <<мышь Microsoft>>, иногда уточняя ещё способ подключения к компьютеру. Поэтому официальное название данного экземпляра <<Microsoft Serial Mouse>> скорее запутывало, и мышь получила известность среди пользователей под названием <<Cероглазой мыши Microsoft>> (чтобы дифференцировать ее от мыши первого поколения с парой зеленых кнопок, подаривших ей название <<Зеленоглазая мышь>>). Также иногда эту мышь упоминают под названием <<Microsoft Mouse 5.0>> \cite{mouses}, очевидно из-за вариантов мышей первого поколения с разными интерфейсами. Производством мыши, как и в случае первого поколения, занималась японская компания Alps.

\begin{figure}[h]
   \centering
    \includegraphics[scale=0.4]{1985_microsoft_gray_eyed_mouse/pic_30.jpg}
    \caption{Microsoft Gray-eyed Mouse}
    \label{fig:MicrosoftGrayEyedPic}
\end{figure}

Второе поколение получило несколько существенных улучшений как в конструкции, так и в плане эргономики. Материалом корпуса стал классический бежевый пластик, а сам корпус стал более выпуклым. Две контрастные серые кнопки по-прежнему расположены на наклонной передней грани, но заходят также и на верхнюю стенку корпуса (рис.  \ref{fig:MicrosoftGrayEyedPic}). Иногда встречаются также экземпляры <<сероглазой>> мыши с кнопками красного или коричневого цвета (рис.  \ref{fig:MicrosoftGrayEyedRedPic}).

\begin{figure}[h]
   \centering
    \includegraphics[scale=0.35]{1985_microsoft_gray_eyed_mouse/pic_red_30.jpg}
    \caption{Microsoft Gray-eyed Mouse}
    \label{fig:MicrosoftGrayEyedRedPic}
\end{figure}

Нижняя часть содержит стальной шар с резиновым покрытием и фиксирующее кольцо, которое можно сдвинуть в сторону, чтобы извлечь шар для чистки мыши (рис. \ref{fig:MicrosoftGrayEyedTopAndBottom}).

\begin{figure}[h]
    \centering
    \includegraphics[scale=0.55]{1985_microsoft_gray_eyed_mouse/top_30.jpg}
    \includegraphics[scale=0.55]{1985_microsoft_gray_eyed_mouse/bottom_30.jpg}
    \caption{Microsoft Gray-eyed Mouse, вид сверху и снизу}
    \label{fig:MicrosoftGrayEyedTopAndBottom}
\end{figure}

Вокруг фиксирующего кольца наблюдается контрастная кольцеобразная накладка из низкофрикционного материала, являющаяся отличительной чертой многих мышей 80-х годов, разработанных в Японии. Резиновое покрытие шара дало мыши существенное преимущество по сравнению с обычным стальным шаром зеленоглазой мыши: оно одновременно уменьшало проскальзывание шара и обеспечивало тихую работу мыши на твердых поверхностях.

\begin{figure}[h]
    \centering
    \includegraphics[scale=0.5]{1985_microsoft_gray_eyed_mouse/size_30.jpg}
    \caption{Microsoft Gray-eyed Mouse на размерном коврике с шагом сетки 1~см}
    \label{fig:MicrosoftGrayEyedSize}
\end{figure}

Размеры мыши являются типичными для 80-х годов (рис. \ref{fig:MicrosoftGrayEyedSize}) и определяются размерами типового узла ALPS. В рекламных материалах Microsoft упоминается, что <<огибающие корпус командные кнопки спроектированы таким образом, чтобы естественно помещаться в ладони любого размера>> \cite{mouses}. Часть критики в адрес эргономики мышей первого поколения касалась возможности сдвинуть мышь, нажимая кнопки на передней стенке корпуса, и такое двойное положение кнопок, очевидно, решало данную проблему, позволяя нажимать на них сверху тем, кому это удобнее (рис. \ref{fig:MicrosoftGrayEyedHand}). Вслед за Microsoft, эта угловая форма кнопок появилась в некоторых других мышах. Больше всего сходства, вплоть до похожей формы корпуса, демонстрируют мыши ProCorp Serial Mouse 1988 года выпуска и Vatek Сolor Mouse 1989 года.

\begin{figure}[h]
    \centering
    \includegraphics[scale=0.5]{1985_microsoft_gray_eyed_mouse/hand_30.jpg}
    \caption{Microsoft Gray-eyed Mouse с моделью руки человека}
    \label{fig:MicrosoftGrayEyedHand}
\end{figure}

Также рекламные материалы отмечали <<вдвое большее разрешение, чем у большинства других мышей, "--- 200 точек на дюйм>> (под <<большинством других мышей>> в данном случае имелось в виду первое поколение мышей Microsoft).

Данный экземпляр мыши имеет последовательный интерфейс подключения. Кроме того эта модель выпускалась как квадратурная мышь в двух вариантах \cite{guide}: с шинным интерфейсом (в комплекте со специальной платой-адаптером для установки в системный блок) и с интерфейсом InPort (ставшим попыткой Microsoft стандартизировать интерфейс подключения квадратурных мышей,  соответсвующие адаптеры и переходники для них). При этом у первых двух поколений мышей Microsoft с последовательным интерфейсом совпадает код FCC ID, поэтому на рис. \ref{fig:MicrosoftGrayEyedTopAndBottom} можно видеть FCC ID <<зелоноглазой>> мыши, зарегистрированный в 1983 году \cite{zero}.

\begin{figure}[h]
    \centering
    \includegraphics[scale=0.6]{1985_microsoft_gray_eyed_mouse/inside_30.jpg}
    \caption{Microsoft Gray-eyed Mouse в разобранном виде}
    \label{fig:MicrosoftGrayEyedInside}
\end{figure}

Внутреннее устройство мыши показано на рис. \ref{fig:MicrosoftGrayEyedInside}. Очевидно, реальным изготовителем мыши была японская компания Alps, часто выступавшая подрядчиком в разработке мышей по заказу других фирм (услугами Alps в частности пользовались такие компании, как IBM, Sharp, NeXT, Siemens, Intergraph). Alps обычно изготавливала для каждого заказчика дизайн мыши с уникальной формой корпуса и расположением кнопок, но использовала одно и то же конструктивное исполнение в нескольких устройствах. В частности, данная мышь совпадает (включая нижнюю часть корпуса, массивные металлические ролики с подшипниками, узел регистрации движения на основе закрытых контактных энкодеров, подключенный T-образным гибким шлейфом к двусторонней печатной плате) с продававшейся в 1986 году мышью Panasonic FS-JM650 и с IBM PS/2 Mouse, выпущенной в 1987 году.

\begin{thebibliography}{9}
\bibitem{mouses} Microsoft Gray-eyed Mouse -- oldmouse.com \url{https://web.archive.org/web/20210417233303/http://oldmouse.com/mouse/microsoft/greyeyed.shtml}
\bibitem{guide} Microsoft Mouse User Guide. Microsoft, 1986. \url{https://minuszerodegrees.net/manuals/Microsoft/Microsoft%20Mouse%20-%20User's%20Guide%20-%201986.pdf}
\bibitem{zero} Microsoft Mouse. Minus zero degrees \url{https://www.minuszerodegrees.net/msmouse/Microsoft%20mouse.htm}
\end{thebibliography}
\end{document}
