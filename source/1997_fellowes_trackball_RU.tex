\documentclass[11pt, a4paper]{article}
\input{preamble.tex}

\begin{document}

\title{1997 "--- Fellowes Sphere Trackball}
\date{}
\maketitle

Fellowes Sphere Trackball, показанный на рис. \ref{fig:FellowesTrackballPic} — типичный представитель данного типа указательных устройств ввода информации для компьютера, наиболее характерных для первой половины 90-х годов (хотя и выпущен компанией Fellowes Computerware в 1997).

\begin{figure}[h]
    \centering
    \includegraphics[scale=0.5]{1997_fellowes_trackball/pic_30.jpg}
    \caption{Fellowes Trackball}
    \label{fig:FellowesTrackballPic}
\end{figure}

Этот трекбол имеет симметричный дизайн, подходящий как для правшей, так и для левшей (рис. \ref{fig:FellowesTrackballTopBottom}). 

\begin{figure}[h]
    \centering
    \includegraphics[scale=0.45]{1997_fellowes_trackball/top_60.jpg}
    \includegraphics[scale=0.45]{1997_fellowes_trackball/bottom_60.jpg}
    \caption{Fellowes Trackball, вид сверху и снизу}
    \label{fig:FellowesTrackballTopBottom}
\end{figure}

Трекбол является достаточно крупным (рис. \ref{fig:FellowesTrackballSize}); в описании, приведенном в каталоге производителя, в качестве отличительных особенностей устройства упоминается прецизионное управление, достигаемое за счёт шара большого диаметра, а также легкое извлечение шара для чистки \cite{advertising}.

\begin{figure}[h]
    \centering
    \includegraphics[scale=0.3]{1997_fellowes_trackball/size_30.jpg}
    \caption{Fellowes Trackball на размерном коврике с шагом сетки 1~см}
    \label{fig:FellowesTrackballSize}
\end{figure}

При работе с трекболом для операций перемещения курсора используется только кисть руки и движения пальцев. Поэтому от пользователя не требуется движений плеча и предплечья, тогда как те же операции с мышью требуют задействования практически всей руки. Поэтому трекбол часто рекомендуется пользователям, испытывающим временные или постоянные проблемы, связанные сплечевым поясом или запястьем.

\begin{figure}[h]
    \centering
    \includegraphics[scale=0.3]{1997_fellowes_trackball/hand_30.jpg}
    \caption{Fellowes Trackball с моделью руки человека}
    \label{fig:FellowesTrackballHand}
\end{figure}

В плане эргономики можно отметить удобную форму корпуса и кнопки большого размера, однако расположение кнопок нельзя назвать оптимальным (рис. \ref{fig:FellowesTrackballHand}). Использование всего двух кнопок, расположенных симметрично по бокам устройства, делает внешний вид эстетически привлекательным, однако это заставляет пользователя расставлять пальцы максимально широко ~--- особенно при необходимости их одновременного нажатия, которое, в теории, могло бы использоваться для имитации отсутствующей третьей кнопки либо для использования шара для скроллинга.

\begin{figure}[h]
    \centering
    \includegraphics[scale=0.7]{1997_fellowes_trackball/inside_60.jpg}
    \caption{Fellowes Trackball в разобранном виде}
    \label{fig:FellowesTrackballInside}
\end{figure}

Внутреннее устройство данного трекбола показано на рис. \ref{fig:FellowesTrackballInside}. Как можно видеть,  трекбол выполнен по традиционной оптомеханической схеме. Помимо симметричного дизайна, одинаково удобного как для левой, так и для правой руки, производитель в рекламных материалах делал упор на высококачественные кнопки большой площади. В плане совместимости заявлена поддержка ОС Windows начиная с версии 3.1 \cite{advertising}.

\begin{thebibliography}{9}
\bibitem{advertising} Fellowes on-line catalog \url{https://docs.rs-online.com/046b/0900766b8024b417.pdf}
\end{thebibliography}
\end{document}
