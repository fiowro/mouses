\documentclass[11pt, a4paper]{article}
\input{preamble.tex}
\switchlang{ru}
\begin{document}

\title{1994 "--- Elecom Lunaris Mouse}
\date{}
\maketitle
\selectlanguage{russian}
Мышь Elecom Lunaris была представлена на рынке в 1994 году компанией Elecom -- японским производителем электроники, основанным восемью годами ранее \cite{elecom} и к этому моменту уже успевшим представить на рынке две компьютерные мыши.
Мышь Lunaris была выпущена в нескольких вариантах, отличавшихся интерфейсом подключения (достаточно часто встречаются также экземпляры с интерфейсом PS/2, USB, варианты для подключения к компьютерам стандарта NEC PC-98) и наличием либо отсутствием возможности
регулировки разрешающей способности. Данный экземпляр (рис. \ref{luna}), имеющий номер модели M-L98MD, оснащён шиной ADB и таким образом предназначен для подключения к компьютерам Apple.

\begin{figure}[h]
    \centering
    \includegraphics[scale=0.5]{1994_elecom_lunaris_mouse/pic_30.jpg}
    \caption{Elecom Lunaris M-L98MD mouse} \label{luna}
\end{figure}

Мышь отличается явным дизайнерским исполнением, что не является случайным: ее внешний вид был разработан компанией Frog Design известного промышленного дизайнера  Хартмута Эсслингера \cite{frog} -- создателем дизайн-языка <<Snow Wihite>>,
применявшегося в линейке компьютеров Apple. Пристальное внимание к дизайну отмечавшееся с начала 90-х годов, позиционировалось Elecom как преимущество перед конкурентами, разрабатывавшими свои продукты с основным упором на технические характеристики
и цену \cite{elecom_timeline}. Олицетворением такой политики компании в 90-х годах стала линейка продуктов Lunaris, начатая с данной компьютерной мыши, и далее дополненная другими компьютерными аксессуарами.

Усилия Elecom в отношении дизайна не остались незамеченными, и в 1994 году Lunaris mouse выиграла престижную японскую премию в области промышленного дизайна «Good Design Award» \cite{award} (вслед за предыдущей моделью мышей компании,
Elecom Graio 200 mouse, получившей такую же премию годом ранее).

\begin{figure}[h]
    \centering
    \includegraphics[scale=0.65]{1994_elecom_lunaris_mouse/top_30.jpg}
    \includegraphics[scale=0.65]{1994_elecom_lunaris_mouse/bottom_30.jpg}
    \caption{Elecom Lunaris M-L98MD mouse, вид сверху и снизу} \label{topandbottomluna}
\end{figure}

Корпус мыши Lunaris выполнен из бежевого пластика матовой текстуры (иногда встречаются также экземпляры светло- или тёмно-серого цвета; кроме того, они могут быть глянцевыми).
Он имеет симметричную форму, одинаково удобную для управления левой и правой рукой (рис. \ref{topandbottomluna}). Как можно заметить, в основу дизайна мыши положено сочетание четких геометрических форм (круг, дуга, прямоугольник) и их комбинаций.
В основании корпуса находится кольцо поворотного переключателя, с помощью которого выполняется регулировка  разрешающей способности мыши. Внутри кольца на нижней стороне можно видеть прорезиненный шар небольшого диаметра, поворотное кольцо-защелку,
позволяющее выполнить чистку мыши, а также полукруглый ярлык с техническими данными мыши. На верхней стороне можно видеть эмблему линейки продуктов Lunaris и две крупные кнопки, которые вписаны в обводы корпуса и разделены вертикальной переборкой
(в первой половине 90-х годов считалось, что без четкой тактильной идентификации пользователям будет трудно определить, где заканчивается левая кнопка и начинается правая). Основная часть корпуса мыши имеет в основании прямоугольник с выпуклыми
передней и задней гранями. При этом диаметр кольца поворотного переключателя превышает ширину прямоугольника, и сверху оно закрыто парой выступов, образующих сегмент сферы, скрытой в толщине основной части корпуса (рис. \ref{sizeluna}).

\begin{figure}[h]
    \centering
    \includegraphics[scale=0.55]{1994_elecom_lunaris_mouse/size_30.jpg}
    \caption{Elecom Lunaris M-L98MD mouse на размерном коврике с шагом сетки 1~см} \label{sizeluna}
\end{figure}

Вращение кольца ступенчатого переключателя против часовой стрелки приводит к уменьшению разрешающей способности (ускорению перемещения курсора), а поворот по часовой стрелке дает противоположный эффект. Кольцо обеспечивает ступенчатое перемещение и
проградуировано от 50 до 1600 отсчетов на дюйм. Варианты данной мыши без возможности регулировки разрешающей способности сохраняют кольцо переключателя как декоративный элемент, однако оно не проградуировано, а на месте слова <<variable>> на нижней
стороне корпуса указана разрешающая способность мыши в DPI.

\begin{figure}[h]
    \centering
    \includegraphics[scale=0.6]{1994_elecom_lunaris_mouse/hand_30.jpg}
    \caption{Elecom Lunaris M-L98MD mouse с моделью руки человека} \label{handluna}
\end{figure}

Выпуклый дугообразный профиль верхней стороны мыши и вытянутая форма делают её достаточно удобной в качестве опоры для ладони (рис. \ref{handluna}).

\begin{figure}[h]
    \centering
    \includegraphics[scale=0.7]{1994_elecom_lunaris_mouse/inside_60.jpg}
    \caption{Elecom Lunaris M-L98MD mouse в разобранном виде} \label{insideluna}
\end{figure}

Узел отслеживания движения Elecom Lunaris mouse представляет собой достаточно типичную для начала 90-х годов оптомеханическую конструкцию с черным пластиковым кожухом, закрывающим оптопары, и металлическими роликами (рис. \ref{insideluna}).
Перевернутое положение печатной платы, поднятой на стойках над дном корпуса, является не самым типичным, т.к. вызвано встроенным в дно корпуса поворотным переключателем.

\begin{thebibliography}{9}
    \bibitem {elecom} Elecom -- Wikipeda \url{https://en.wikipedia.org/wiki/Elecom}    
    \bibitem {frog} Frog Design -- Wikipeda \url{https://en.wikipedia.org/wiki/Frog_Design}
    \bibitem{elecom_timeline} Elecom Sustainability Report 2025 \url{https://www.elecom.co.jp/ir/sustainability/pdf/i-2025_eng.pdf}    
    \bibitem {award} LUNARIS-MOUSE M-L98MD, M-LMA, M-LP2 -- g-mark.org \url{https://www.g-mark.org/en/gallery/winners/9cdee7d5-803d-11ed-862b-0242ac130002}
\end{thebibliography}

\end{document}
