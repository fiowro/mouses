\documentclass[11pt, a4paper]{article}
\input{preamble.tex}

\begin{document}

\title{1983 "--- Hawley Mark II X063X Mouse}
\date{}
\maketitle

Мышь Mark II X063X Mouse (рис. \ref{fig:HawleyMarkIIPic}) стала первой собственной разработкой компании Hawley Mouse House \cite{hawley,mouses}, созданной Джеком Хоули, соразработчиком мыши для компьютеров Xerox Alto и одним из авторов патента Xerox 1973 года на мышь с двумя наклонными колесами \cite{pat}.

\begin{figure}[h]
   \centering
    \includegraphics[scale=0.6]{1983_hawley_mark_ii/pic_60.jpg}
    \caption{Hawley Mark II X063X Mouse, вид спереди}
    \label{fig:HawleyMarkIIPic}
\end{figure}

Мышь Mark II X063X выполнена в доведенном до крайней степени индустриальном дизайне: корпус представляет собой паралеллепипед, на котором расположены три прямоугольные кнопки цвета, максимально контрастного по отношению к корпусу (рис. \ref{fig:HawleyMarkIITopAndBottom}). Очевидно, такой дизайн был призван подчеркнуть целевое назначение мыши, ориентированной на инженеров и пользователей различных САПР (включая твердотельное моделирование и архитектуру).

Данный экземпляр является бежевым с черными кнопками (на кабеле можно заметить остатки черной изоляции, утратившей со временем эластичность и рассыпавшейся на части). Распространенным также был вариант с противоположным сочетанием цветов корпуса и кнопок, а также известно еще несколько вариантов расцветки. Реклама Hawley дает представление о возможных сочетаниях цветов \cite{brochure}, однако нет информации о том, сколько вариантов расцветки из этой рекламной фотографии было реализовано на практике.

\begin{figure}[h]
    \centering
    \includegraphics[scale=0.5]{1983_hawley_mark_ii/top_60.jpg}
    \includegraphics[scale=0.5]{1983_hawley_mark_ii/bottom_60.jpg}
    \caption{Hawley Mark II X063X Mouse, вид сверху и снизу}
    \label{fig:HawleyMarkIITopAndBottom}
\end{figure}



Нижняя стоорона целиком выполнена из металла (рис. \ref{fig:HawleyMarkIITopAndBottom}). Вращение регистрируется гладким стальным шаром в центре, а еще два шарика меньшего размера играют роль ножек для минимизации трения. Съемное кольцо, позволяющее извлечь шар для удаления собравшегося мусора, в данной модели еще не предусмотрено, поэтому для чистки необходима полная разборка.

\begin{figure}[h]
    \centering
    \includegraphics[scale=0.5]{1983_hawley_mark_ii/size_30.jpg}
    \caption{Hawley Mark II на размерном коврике с шагом сетки 1~см}
    \label{fig:HawleyMarkIISize}
\end{figure}

Мышь имеет небольшие размеры, характерные для мышей 1980-х годов (рис. \ref{fig:HawleyMarkIISize}). Очевидно, это хотя бы немного уменьшает негативное влияние корпуса с ортогональными гранями на эргономику, поскольку рука может опереться на корпус лишь в незначительной степени (рис. \ref{fig:HawleyMarkIIHand}).

\begin{figure}[h]
    \centering
    \includegraphics[scale=0.5]{1983_hawley_mark_ii/hand_60.jpg}
    \caption{Hawley Mark II с моделью руки человека}
    \label{fig:HawleyMarkIIHand}
\end{figure}

Внутреннее устройство мыши показано на рис. \ref{fig:HawleyMarkIIInside}. Можно отметить съемную глухую защиту шара, требующую дополнительных операций разборки для удаления мусора. В мыши испольованы контактные энкодеры (с четырьмя контактами для большей надежности), включающие в себя металлический контактный барабан, вместо более распространенного в последующих моделях диска.

 \begin{figure}[h]
    \centering
    \includegraphics[scale=0.8]{1983_hawley_mark_ii/inside_60.jpg}
    \caption{Hawley Mark II в разобранном виде}
    \label{fig:HawleyMarkIIInside}
\end{figure}

\begin{thebibliography}{9}
\bibitem{hawley} Hawley Mouse House \url{https://oldmouse.com/mouse/hawley/}

\bibitem{mouses} Hawley Mark II X063X Mouses \url{https://oldmouse.com/mouse/hawley/X063X.shtml}

\bibitem{pat} Transducer for a display-oriented pointing device \url{https://patents.google.com/patent/US3892963A/en}

\bibitem{brochure} Mouse House MK II Brochure \url{https://www.microsoft.com/buxtoncollection/a/pdf/Mouse%20House%20MK%20II%20Brochure.pdf}
\end{thebibliography}
\end{document}
