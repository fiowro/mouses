\documentclass[11pt, a4paper]{article}
\input{preamble.tex}
\switchlang{en}
\begin{document}

\title{1994 "--- Memorex trackball}
\date{}
\maketitle
\selectlanguage{english}
The Memorex trackball shown in figure \ref{fig:MemorexPic} was manufactured by Memtek Products inc., California. At the same time, the actual production, in a typical way for the mid-90s, was carried out in Taiwan.

\begin{figure}[h]
    \centering
    \includegraphics[scale=0.55]{1994_memorex_trackball/pic_30.jpg}
    \caption{Memorex trackball}
    \label{fig:MemorexPic}
\end{figure}

The body of the trackball is asymmetrical, made in a minimalist style of glossy white plastic with a brown insert.

\begin{figure}[h]
    \centering
    \includegraphics[scale=0.5]{1994_memorex_trackball/top_30.jpg}
    \includegraphics[scale=0.5]{1994_memorex_trackball/bottom_30.jpg}
    \caption{Memorex trackball, top and bottom views}
    \label{fig:MemorexTopBottom}
\end{figure}

As you can see in figure \ref{fig:MemorexSize}, this manipulator is relatively small.

The trackball packaging and user manual use the “Stationary Mouse” subtitle, apparently borrowed from the Logitech TrackMan Stationary Mouse trackball released a year earlier, and emphasize its ease of use in a limited workspace.

\begin{figure}[h]
    \centering
    \includegraphics[scale=0.41]{1994_memorex_trackball/size_30.jpg}
    \caption{Memorex trackball on a graduated pad with a grid step of 1~cm}
    \label{fig:MemorexSize}
\end{figure}

The trackball is right-handed and the shape is largely inspired by the Logitech another product - LOGiTECH, which is also designed for a horizontal position of the hand and the rotation of the ball with the thumb (figure \ref{fig:MemorexHand}). According to Logitech, this shape is better suited to the anatomical structure of the hand (in advertising, the shape of classic symmetrical trackballs was opposed to a similar shape as a “device for aliens” \cite{adv}).

\begin{figure}[h]
    \centering
    \includegraphics[scale=0.2]{1994_memorex_trackball/hand_30.jpg}
    \caption{Memorex trackball with a human hand model}
    \label{fig:MemorexHand}
\end{figure}

Trackball internals are shown on figure \ref{fig:MemorexInside}. It is an opto-mechanical device typical for the beginning of the 90s, in both the encoder implementation and the whole internal layout, which leaves a significant amount of empty space inside the case.

\begin{figure}[h]
    \centering
    \includegraphics[scale=0.6]{1994_memorex_trackball/inside_30.jpg}
    \caption{Memorex trackball disassembled}
    \label{fig:MemorexInside}
\end{figure}

\begin{thebibliography}{9}
\bibitem {adv} Not every kind of pointing device fits your kind of hand (LOGiTECH TrackMan tadvertising). // PC Magazine. V.~8, No.~19. November, 1989, pp. 360-361. \url{https://archive.org/details/PC-Mag-1989-11-14/page/n361}
\end{thebibliography}
\end{document}
