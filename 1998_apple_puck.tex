\documentclass[11pt, a4paper]{article}
\input{preamble.tex}

\begin{document}

\title{1998 "--- Apple Puck Mouse}
\date{}
\maketitle
Мышь Apple USB mouse, часто называемая <<шайбой>> (англ. <<puck>>)  из-за своей необычной формы, была разработана компанией Apple в 1998 году. Это была первая коммерчески выпущенная мышь Apple mouse, которая использовала формат подключения USB, а не шину Apple ADB. Многие обозреватели критиковали данную мышь за ее недостаточно эргономичный дизайн.

\begin{figure}[h]
    \centering
    \includegraphics[scale=0.3]{1998_apple_puck/apple.jpg}
    \caption{Apple Puck Mouse}
    \label{fig:pic}
\end{figure}

В отличие от большинства манипуляторов, мышь <<шайба>> имеет круглую форму, и у нее есть одна кнопка мыши, расположенная вверху, как и у предыдущих мышей Apple. При этом мышь имеет зазор между кнопкой и корпусом, показывающий, куда именно пользователь должен нажимать.

\begin{figure}[h]
    \centering
    \includegraphics[scale=0.5]{1998_apple_puck/appleup.JPG}
    \includegraphics[scale=0.5]{1998_apple_puck/appledown.JPG}
    \caption{Apple Puck Mouse, вид сверху и снизу}
    \label{fig:top}
\end{figure}

Круглая форма мыши была признана сообществом неудобной из-за небольшого размера данного конкретного манипулятора и склонности вращаться при использовании. 
Это стало основной причиной успеха адаптеров Griffin iMate ADB для USB, поскольку они позволяли использовать с компьютерами iMac более старую и удобную мышь ADB, а также пластиковых накладок, придававших мыши более продолговатую форму. 

\begin{figure}[h]
    \centering
    \includegraphics[scale=0.23]{1998_apple_puck/appp.jpg}
    \caption{Apple Puck Mouse, вид с накладкой}
    \label{fig:addon}
\end{figure}

Также из-за малого размера, перемещение мыши на самом деле требовало гораздо большего количества движений  пальцев и  меньшего количества движений запястья по сравнению с более крупными мышами.

\begin{figure}[h]
    \centering
    \includegraphics[scale=0.3]{1998_apple_puck/appleset.jpg}
    \caption{Apple Puck Mouse на размерном коврике с шагом сетки 1~см}
    \label{fig:size}
\end{figure}

В полупрозрачном пластике помещалась печатная плата и двухцветный шар, который можно было легко разглядеть. 
\begin{figure}[h]
    \centering
    \includegraphics[scale=0.25]{1998_apple_puck/appleset2.jpg}
    \caption{Apple Puck Mouse с моделью руки человека}
    \label{fig:hand}
\end{figure}

Однако идеально круглое тело часто приводило к ошибкам, так как пользователи предполагали, что мышь была в правильной ориентации, даже если это было не так. Позже Apple добавила ямочку на корпусе мыши, чтобы помочь пользователям почувствовать, в каком направлении указывала мышь.
\begin{figure}[h]
    \centering
    \includegraphics[scale=0.3]{1998_apple_puck/apple2.jpg}
    \caption{Apple Puck Mouse, в разобранном виде}
    \label{fig:inside}
\end{figure}

Внутреннее устройство данной мыши показано на рисунке 2.18, что позволяет классифицировать  ее как оптомеханическую.

\begin{thebibliography}{9}

    \bibitem {Apple} An ode to the puck, Apple's first USB mouse "---
    \url{https://thehouseofmoth.com/an-ode-to-the-puck-apples-first-usb-mouse/}

\end{thebibliography}


\end{document}
