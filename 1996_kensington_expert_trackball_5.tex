\documentclass[11pt, a4paper]{article}
\input{preamble.tex}

\begin{document}

\title{1996 "--- Kensington Expert Mouse Trackball 5.0}
\date{}
\maketitle
В 1996 году с выходом пятой по счёту модели Expert Mouse Trackball претерпел существенный редизайн \cite{KensingtonPC}. Устройство оснащено крупным шаром и четырьмя крупными кнопками, расположенными вокруг него как лепестки цветка (рис. \ref{fig:pic}).

\begin{figure}[h]
    \centering
    \includegraphics[scale=0.4]{1996_kensington_expert_trackball_5/king.jpg}
    \caption{Kensington Expert Trackball}
    \label{fig:pic}
\end{figure}

Аналогично выглядевшая версия для Macintosh с предсказуемым названием Turbo Mouse 5.0 (\cite{KensingtonMac}) отличалась интерфейсом ADB, в то время как Expert Mouse комплектовался сменными кабелями для подключения к последовательному интерфейсу и к порту PS/2 (также отдельно выпускалась шинная версия с ISA-адаптером). Визуально устройства не отличались.

\begin{figure}[h]
    \centering
    \includegraphics[scale=0.5]{1996_kensington_expert_trackball_5/kingup.JPG}
    \includegraphics[scale=0.5]{1996_kensington_expert_trackball_5/kingdown.JPG}
    \caption{Kensington Expert Trackball вид сверху}
    \label{fig:top}
\end{figure}



%\begin{figure}[h]
%    \centering
%    \includegraphics[scale=0.2]{1996_kensington_expert_trackball_5/234.JPG}
%    \caption{Kensington Expert Trackball вид снизу}
%    \label{fig:bottom}
%\end{figure}


\begin{figure}[h]
    \centering
    \includegraphics[scale=0.3]{1996_kensington_expert_trackball_5/kingset.jpg}
    \caption{Kensington Expert Trackball на размерном коврике с шагом сетки 1~см}
    \label{fig:size}
\end{figure}


\begin{figure}[h]
    \centering
    \includegraphics[scale=0.3]{1996_kensington_expert_trackball_5/kingset2.jpg}
    \caption{Изображение Kensington Expert Trackball с моделью руки человека}
    \label{fig:hand}
\end{figure}

Это удобно, например для изготовления снимков области экрана или других прецизионных действий — обеспечивается точность до пикселя. Поначалу немного неудобно вращать шар, не отпуская кнопку (ну, например, для выделения текста или объектов на экране), но это дело привычки. 

\begin{figure}[h]
    \centering
    \includegraphics[scale=0.4]{1996_kensington_expert_trackball_5/king2.jpg}
    \caption{Kensington Expert Trackball в разобранном виде}
    \label{fig:inside}
\end{figure}

Внутреннее устройство данного трекбола показано на рисунке 2.21, что позволяет классифицировать трекбол как оптомеханический.

\begin{thebibliography}{9}

\bibitem {KensingtonPC} Kensington: Expert Mouse 5.0 "--- \url{https://web.archive.org/web/19970106170305/http://www.kensington.com/prod/mice/mice3b.html}
\bibitem {KensingtonMac} Kensington: Turbo Mouse 5.0 "--- \url{https://web.archive.org/web/19970106170317/http://www.kensington.com/prod/mice/mice3a.html}
\end{thebibliography}

\end{document}
