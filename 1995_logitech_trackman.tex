\documentclass[11pt, a4paper]{article}
\input{preamble.tex}

\begin{document}

\title{1995 "--- Трекбол Logitech TrackMan Marble}

\maketitle

Данный трекбол имеет 3 клавиши, отвечаюие за стандартные функции кнопок мыши, и шар, предназначенный для вращения большим пальцем правой руки (рис. \ref{fig:trackman}). Драйвер позволял использовать для прокрутки вращение шара с зажатой средней кнопкой (при нажатии эта кнопка выполняет привычную функцию), но следует отметить, что сущестовала также модификация этого трекбола с традиционным колесом прокрутки в вырезе третьей кнопки.
Трекбол подключается к компьютеру по интерфейсу PS/2.

Корпус трекбола имеет постоянный наклон вправо, благодаря чему запястье лежащей на трекболе руки находится в более естественном положении (рис. \ref{fig:trackmanHand}). В то время, как шар прокручивается большим пальцем, остальные пальцы работают так же, как при пользовании обычной мышью, что делает конструкцию более привлекательной для пользователя, привыкшего к мыши или попеременно работающего мышью и трекболом. Но она имеет и свои недостатки: подвижность большого пальца несколько меньше, что отражается на быстроте и точности позиционирования, к тому же такая конструкция, в отличие от «классической», совершенно непригодна для левшей.

\begin{figure}[h]
    \centering
    \includegraphics[scale=0.3]{1995_logitech_trackman/2.15.JPG}
    \caption{Изображение Logitech TrackMan}
    \label{fig:trackman}
\end{figure}

\begin{figure}[h]
    \centering
    \includegraphics[scale=0.5]{1995_logitech_trackman/2.14.JPG}
    \caption{Изображение Logitech TrackMan с моделью руки человека}
    \label{fig:trackmanHand}
\end{figure}

Из особенностей можно отметить нестандартный рисунок шара, на котором нанесен регулярный узор из тёмных точек. Разбор трекбола (рис. \ref{fig:trackmanInside}) показывает, что причиной этой расцветки является отказ фирмы Logitech от традиционной схемы оптомеханической мыши в пользу аналога оптической мыши, считывающей изменения яркости с помощью специального коврика с нанесенной на нём сеткой. Только в данном случае роль коврика играет рисунок на вращающемся шаре. По заверениям разработчика, распознавание движения реализовано системой на основе искусственной нейронной сети \cite{marbleAdv}.

\begin{figure}[h]
    \centering
    \includegraphics[scale=0.5]{1995_logitech_trackman/201.JPG}
    \caption{Изображение Logitech TrackMan изнутри}
    \label{fig:trackmanInside}
\end{figure}

\begin{figure}[h]
    \centering
    \includegraphics[scale=0.5]{1995_logitech_trackman/2.16.JPG}
    \includegraphics[scale=0.5]{1995_logitech_trackman/2.17.JPG}
    \caption{Изображение Logitech TrackMan, вид сверху и снизу}
    \label{fig:trackmanTopAndBottom}
\end{figure}
    
    Маркировка на нижней части трекбола содержит код FCC ID (рис. \ref{fig:trackmanTopAndBottom}).
    Проверка кода по базе данных Федеральной комиссии по коммуникациям США показывает, что трекбол был разработан компанией Logitech в 1995 году.

%\begin{figure}[h]
%    \centering
%    \includegraphics[scale=0.5]{1995_logitech_trackman/2.17.JPG}
%    \caption{Изображение Logitech TrackMan вид снизу}
%    \label{fig:trackmanBottom}
%\end{figure}

\begin{thebibliography}[9]
\bibitem{marbleAdv} Melissa J. Perenson. New & improved. News of announced products and upgrades. // PC Magazine, Vol. 14, No. 22. -- December 19, 1995. -- p. 61 -- 66.
\end{thebibliography}

\end{document}
