\documentclass[11pt, a4paper]{article}
\input{preamble.tex}

\begin{document}

\title{1990 "--- Трекбол Kraft TopTrak}

\maketitle

Трекбол TopTrak имеет средние размеры и корпус с закруглёнными углами в стиле дизайна бытовых приборов 60-х годов (рис. \ref{fig:TopTrakTop}). Устройство снабжено кабелем длиной 2,5 м, что заметно больше типового расстояния между пользователем и системным блоком. Трекбол имеет последовательный интерфейс сопряжения с компьютером.

\begin{figure}[h]
    \centering
    \includegraphics[scale=0.55]{1990_kraft_toptrack/2.9.jpg}
    \caption{TopTrak, вид сверху}
    \label{fig:TopTrakTop}
\end{figure}

Дополнительно в комплекте с трекболом идёт стальная ножная педаль (рис. \ref{fig:TopTrakPedal}), которая служит альтернативой левой кнопке мыши, имеет еще один аналогичный кабель и добавляет дополнительные полкилограмма веса устройству.

\begin{figure}[h]
    \centering
    \includegraphics[scale=0.55]{1990_kraft_toptrack/2.7.jpg}
    \caption{Изображение педали для мыши TopTrak}
    \label{fig:TopTrakPedal}
\end{figure}

%\begin{figure}[h]
%    \centering
%    \includegraphics[scale=0.4]{1990_kraft_toptrack/2.8.jpg}
%    \caption{Изображение педали для мыши TopTrak с моделью руки человека}
%    \label{fig:TopTrakPedalHand}
%\end{figure}

Левая и правая кнопки TopTrak полностью занимают верхние углы.

\begin{figure}[h]
    \centering
    \includegraphics[scale=0.35]{1990_kraft_toptrack/2.6.jpg}
    \caption{Изображение TopTrak на размерном коврике с шагом сетки 1~см}
    \label{fig:TopTrakSize}
\end{figure}


TopTrak может быть подходящим устройством для настольного компьютера, но он слишком громоздкий, чтобы его всерьез рассматривать для портативных компьютеров.

\begin{figure}[h]
    \centering
    \includegraphics[scale=0.45]{1990_kraft_toptrack/2.5.jpg}
    \caption{Изображение TopTrak с моделью руки человека}
    \label{fig:TopTrakHand}
\end{figure}

Как можно видеть, маркировка TopTrak содержит код FCC ID (рис. \ref{fig:TopTrakBottom}).

\begin{figure}[htpb]
    \centering
    \includegraphics[scale=0.4]{1990_kraft_toptrack/2.10.jpg}
    \caption{TopTrak, вид снизу}
    \label{fig:TopTrakBottom}
\end{figure}

Проверка кода по базе данных Федеральной комиссии по коммуникациям США показывает, что трекбол был разработан американской компанией Kraft Systems в 1990 году.

Изучение разобранного трекбола (рис. \ref{fig:TopTrakInside}) показывает, что он выполнен по стандартной оптомеханической схеме, а массивные металлические ролики с подшипниками качения показывают, что трекбол был задуман как достаточно долговечное устройство, не относящееся к нижнему ценовому диапазону.

\begin{figure}[h]
    \centering
    \includegraphics[scale=0.5]{1990_kraft_toptrack/200.jpg}
    \caption{TopTrak изнутри}
    \label{fig:TopTrakInside}
\end{figure}

\end{document}
