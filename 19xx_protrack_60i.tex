\documentclass[11pt, a4paper]{article}
\input{preamble.tex}

\begin{document}

\title{199x "--- Trackerball ProTrack 60i trackball}
\date{}
\maketitle

    ProTrack 60i является чрезвычайно габаритным трекболом. Диаметр шара составляет 2,5 дюйма (63,5 мм), шар размещается на опоре с использованием подшипников и валов из нержавеющей стали. Для извлечения шара и чистки трекбола требуется разборка.
    
    \begin{figure}[h]
        \centering
    \includegraphics[scale=0.3]{19xx_protrack_60i/raz_monstr.JPG}
        \label{m4-hand}
        \caption{Изображение ProTrack 60i с моделью руки человека}
    \end{figure}
    \begin{figure}[h]
        \centering
    \includegraphics[scale=0.3]{19xx_protrack_60i/monstr2.jpg}
        \label{m4-size}
        \caption{Изображение ProTrack 60i на размерном коврике с шагом сетки 1~см}
    \end{figure}
    \newpage
Перед шаром трекбола расположены три кнопки. На верхней части корпуса присутствует рельефная надпись ProTrack 60i, а также эмблема с надписью <<TRACKERBALL>>.
    \begin{figure}[h]
        \centering
    \includegraphics[scale=0.3]{19xx_protrack_60i/monstr3.JPG}
        \label{m4-top}
        \caption{ProTrack 60i, вид сверху}
    \end{figure}
    
На нижней стенке корпуса присутствуют резиновые ножки, фиксирующие надёжное положения на поверности стола и маркировка ProTrack 60i. Как можно видеть, страной изготовления данного устройства является Великобритания.
Изучение интернет-архива показывает, что компания Trackerball позиционировала себя как произвоидетеля исходного устройства Trackerball, разработанного в 1940-х годах для управления экранным курсором военных радаров, и в последствии разрабатывала высоконадёжные манипуляторы для управления курсором в первую очередь в военных и индустриальных сферах применения.  Модель R60, разновидностью которой является данное устройство, была выпущена в конце 90-х годов.

    \begin{figure}[h]
        \centering
    \includegraphics[scale=0.3]{19xx_protrack_60i/monstr4.JPG}
        \label{m4-bottom}
        \caption{ProTrack 60i, вид снизу}
    \end{figure}
    
    \begin{figure}[h]
        \centering
        \includegraphics[scale=0.4]{19xx_protrack_60i/razobr3.jpg} 
        \label{Wacom-2}
        \caption{ProTrack 60i в разобранном состоянии}
    \end{figure}

 Внутреннее устройство данного трекбола показано на рисунке 2.14, что позволяет классифицировать трекбол как оптомеханический. Подключение данного устройства к компьюетру осуществляется через интерфейс PS/2.
 
\end{document}
