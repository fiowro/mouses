\documentclass[11pt, a4paper]{article}
\input{preamble.tex}

\begin{document}

\title{Трекбол}
\author{Общая информация}
\date{}
\maketitle

    Трекбол используется для тех же целей, что и мышь. Его внутренняя конструкция почти идентична мыши и может рассматриваться как перевернутая мышь, находящаяся в неподвижном положении. Трекбол исторически предшествовал мыши, и существует версия, что концепция мыши была придумана в ходе переворачивания трекбола вверх ногами и перемещения его по поверхности стола. Основные характеристики трекбола показаны на рисунке \ref{fig:theoryTrackballGeneric}. Трекбол представляет собой металлический или пластиковый шар, который монтируется в раме так, что только небольшая его часть выступает через отверстие в верхней части рамы. Шар поддерживается двумя ортогональными роликами-стержнями, так что, когда шар поворачивается влево или вправо, вращается один ролик, а когда он поворачивается вперед или назад, вращается другой.
    
    \begin{figure}[h]
        \centering
    \includegraphics[width=0.5\linewidth]{theory_track/2.3.png}
        \caption{Трекбол}
        \label{fig:theoryTrackballGeneric}
    \end{figure}
    
    Шар полностью свободно вращается в своем гнезде. Он управляется ладонью руки, и движения, принимаемые шаром, находящимся в контакте с двумя роликами, преобразуются внутри корпуса так же, как у механической мыши. Движения роликов так же детектируются путем измерения вращения дисков, прикрепленных к их концам. Это детектирование может выполняться с помощью электрических контактов или светодиодов и фотоприемников.
    
    Как и мышь, трекбол обычно включает в себя несколько кнопок, до которых можно дотянуться кончиками пальцев, пока ладонь лежит на шаре. Для большинства целей мышь более популярна, чем трекбол, но в ситуациях, когда недостаточно свободного места или нет подходящей поверхности, трекбол оказывается более предпочтительным.

\end{document}
